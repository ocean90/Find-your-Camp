%!TEX root = ../expose.tex

Reisende, die während ihres Aufenthalts an mehreren Orten Halt machen, werden oftmals vor dem Problem gestellt, eine geeignete Unterkunft für die Nacht zu finden. Vorhandene Systeme ermöglichen bereits eine zufriedenstellende Suche und gewährleisten eine Übersicht bekannter Hostels, Campingplätze oder Hotels.\\
Ein wesentliches Problem, das bei dieser „Art des Reisens“ auftritt ist jedoch der Kostenfaktor, der für viele ein Hindernis und ein gewisses Risiko darstellen kann.\\
Die Entwicklung der vergangen Jahre zeigte beispielsweise am Couchsurfing, dass es einen Bedarf an alternative Übernachtungsformen gibt, die grundsätzlich auf sozialen Aspekten sowie dem Kostenfaktor begründet liegen und großen Zuspruch erhalten.


\subsubsection{Zielsetzung}
Das Ziel ist die Entwicklung eines Vermietsystems für private Grundstücke als Aufenthaltsplatz für Reisende, in Anlehnung an das Share Economy Konzept.\\
Das verteilte System soll vorhandene Angebote aufzeigen, die Kommunikation zwischen Mieter und Vermieter ermöglichen und den Mietprozess möglichst sicher sowie erfolgreich organisieren und abschließen können.\\
Die Motivation der Mieter liegt darin, kostengünstige Alternativen gegenüber herkömmlichen Aufenthaltsmöglichkeiten zu finden, dabei möglichst mobil zu sein sowie kurzfristige Optionen ermöglichen. Der Vermieter soll zum einen sozialen Nutzen daraus ziehen, dabei aber auch einen finanziellen Gewinn für die Vermietung erzielen. Was diese Variante deutlich vom Couchsurfing System abhebt ist der finanzielle Gewinn des Vermieters, das Ansprechen naturbegeisterter Reisende sowie der Aspekt, dass der Vermieter das „Eindringen“ in den Privatraum selbst regulieren kann.


\subsubsection{Zielgruppe}
Primäre Benutzer der Anwendung sind (langzeit) Reisende in der Tätigkeit des Mietenden und Grundstückbesitzer als Vermieter. Auswirkungen kann das System auch auf die Stadt haben, indem der Tourismus beeinflusst wird und Campingplätze oder Hotels potentielle Kunden verlieren. Als weiterer Stakeholder lassen sich Systemverantwortliche (Administratoren, Support, etc.) identifizieren, die am Erfolg des Systems Interesse haben und mögliche Werbepartner,die eingebunden werden könnten.


\subsubsection{Marktrecherche und Alleinstellungsmerkmal}
Ähnliche Ansätze verfolgen Webseiten wie freagle.org oder campinmygarden.com. Beide ermöglichen das Suchen solcher Angebote, setzen dabei auf ein Reputationssystem und bieten eine gewissen Sicherheit durch notwendige Verifikation. Keine der bekannten Webseiten bieten dabei jedoch eine mobile Variante, die speziell für Reisende eine deutliche Erleichterung bieten kann. Existierende Apps wie der ADAC Campingplatz Finder setzen den Fokus lediglich auf öffentliche Campingplätze, die aufgrund des Preises, für die identifizierte Zielgruppe oftmals nicht in Frage kommen.\\
Die Gestaltung als Applikation unterstützt die natürliche Flexibilität eines Reisenden und ermöglicht ihm neben dem Finden eines Angebotes auch die Abwicklung der Kommunikation sowie gegebenfalls das Orientieren durch die GPS Funktion. Der Einsatz eines QR Code oder NFC Systems zur Identifikation kann zudem die Sicherheit deutlich steigern.


\subsubsection{Risiken}
\begin{itemize}
   \item Beteiligte benötigen Schutz: Verifikation, mögliches QR Code System, Bewertungen
   \item Aktivität der Nutzer: Möglichst vielen Interessenten zugänglich machen und langfristig halten. (Erfolg/Gewinn durch Reputationssystem)
   \item Geschäftsmodell: Finanzielle Absicherung muss gewährleistet sein. (Appkosten, Kaufabwicklung über App, Werbung)
\end{itemize}
