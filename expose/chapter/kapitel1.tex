%!TEX root = ../expose.tex

Jeder Mensch wird mal vor einer Situation im Alltag gestellt, in der er für eine einmalige Tätigkeit ein materielles Gut benötigt, dieses jedoch selbst nicht besitzt. Infolgedessen wird zunächst im Bekanntenkreis nach diesem Gut nachgefragt, um es sich ausleihen zu können.
Aufgrund eines eingeschränkten Bekanntenkreises schlägt dieser Aufruf häufig fehl, sodass nur noch der Kauf des Gutes als Option in Frage kommt - der jedoch in keiner angemessenen Kosten-Nutzen-Relation steht.


\subsubsection{Zielsetzung}
Die Entwicklung eines Verleihsystems für materielle Gegenstände, dass den Bekanntenkreis auf die Nachbarschaft erweitert, um so die potentielle Möglichkeit zu erhöhen, etwas Notwendiges zeitgerecht und kostengünstig zu erhalten, in Anlehnung an das Share Economy Konzept.
Share Economy ist ein Begriff, der durch den Wirtschaftswissenschaftler Martin Weitzman geprägt wurde und sagt aus, dass sich der Wohlstand aller erhöhen wird, je mehr unter allen Marktteilnehmern geteilt wird.\footnote{The Share Economy: Conquering Stagflation, Harvard University Press, 1984}\\
Das verteilte System soll die Kommunikation zwischen Leiher und Verleiher ermöglichen und den Leihprozess möglichst erfolgreich organisieren und abschließen können.\\
Die Motivation zur Nutzung soll darin liegen, schnell und unkompliziert Kontakt zu Besitzern von benötigten Güter aufzubauen und die ansonsten auftretenden Kosten möglichst gering zu halten.


\subsubsection{Zielgruppe}
Volljährige und voll geschäftsfähige Benutzer, die das System direkt in der Tätigkeit als Leiher oder Verleiher benutzen, sowie Personen, die nicht direkt mit dem System interagieren, aber Anteil am Leihprozess haben.\\
Als weiterer Stakeholder lassen sich Händler identifizieren, welche potentielle Kunden verlieren könnten, sowie Systemverantwortliche (Administratoren, Support, etc.), die am Erfolg des Systems Interesse haben.


\subsubsection{Marktrecherche und Alleinstellungsmerkmal}
Ähnliche Ansätze verfolgen Webseiten wie Leihdirwas.de oder erento.de. Beide vernachlässigen jedoch den direkten Bezug zur Umgebung und führen teilweise zu hohen Kosten sowie zeitliche Einschränkungen  durch Versand. Die Auswahl an mobilen Applikationen ist sehr eingeschränkt. Existierende Apps wie Why own it setzen den Fokus lediglich auf den eigenen Freundeskreis und bieten keine direkte Möglichkeit den potentiellen Besitzerkreis zu erweitern.\\
Aspekte die das neue System zu vorhandenen Optionen abheben, wären der verstärkte Bezug zur direkten Umgebung und infolgedessen die Zeitoptimierung zwischen Anfrage und Erhalt.\\
Das System kann/soll außerdem die Kommunktionshemme zu neuen Kontakten überwinden, durch den soziale Kontakt die Sicherheit der Beteiligten  erhöhen und besitzt das Potential neue Bekanntschaften zu schließen.


\subsubsection{Risiken}
\begin{itemize}
   \item Händler verlieren Kunden: Händler können sich ins System „einkaufen”.
   \item Verleiher benötigen Schutz: Kann durch Kaution, Verifikation, Richtlinien gegeben werden.
   \item Einbruchspotential: Die Adresse wird erst kurz vor Vertragsabschluss bekannt gegeben.
   \item Aktivität der Nutzer: Bei kleinen Dörfern oder Gebieten mit einem hohen Altersdurchschnitt kann das System nicht das volle Potential bieten. (Geringer Anwenderkreis.)
   \item Eingeschränkte Benutzervielfalt: Gehören im angefragten Umkreis zu viele Benutzer einer Benutzergruppe an, ist die Wahrscheinlich geringer, dass ein Gut vorhanden ist.
   \item Geschäftsmodel: Finanzielle Absicherung muss gewährleistet sein. (Für Kunden und Betreiber)
\end{itemize}

