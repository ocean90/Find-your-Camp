%!TEX root = ../konzept.tex

\section{Zielsetzung}
Was sind die Entwicklungsziele für das interaktive System?
allgemeinen Zielen (higher-level goals), etwa den Geschäftszielen und darauffolgend  
den System-Zielen (lower- level goals), also den konkreteren Arbeitszielen.
 
Das Ziel des Projektes  ist die Entwicklung eines Vermietsystems für private Grundstücke als Aufenthaltsplatz für Reisende, auf Grundlage des Share Economy Konzeptes.\\
Das verteilte System soll vorhandene Angebote aufzeigen, die Kommunikation zwischen Mieter und Vermieter ermöglichen und den Mietprozess möglichst sicher sowie erfolgreich organisieren und abschließen können.\\
Die Motivation der Mieter liegt darin, kostengünstige Alternativen gegenüber herkömmlichen Aufenthaltsmöglichkeiten zu finden, dabei möglichst mobil zu sein sowie kurzfristige Optionen zu ermöglichen. Der Vermieter kann zum einen sozialen Nutzen daraus ziehen, dabei aber auch einen finanziellen Gewinn für die Vermietung erzielen. Was diese Variante deutlich vom Couchsurfing System abhebt ist der finanzielle Gewinn des Vermieters, das Ansprechen naturbegeisterter Reisende sowie der Aspekt, dass der Vermieter das „Eindringen“ in den Privatraum selbst regulieren kann.
 
\subsection{Zielhierachie}

\subsubsection{Strategische Ziele}
Langfristig

\subsubsection{Taktisches Ziele}
Mittelfristig

\subsubsection{Operative Ziele}
Kurzfristig

\subsubsection{Optionale Ziele}

\subsubsection{Minimale Ziele}

