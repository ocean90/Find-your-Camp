%!TEX root = ../konzept.tex

\section{Zielhierachie}

\subsubsection{Strategische Ziele}
Langfristig gesehen ist das oberste Ziel das System, eine Etablierung des Shared Economy Prinzips auf die Domäne des Grundstückverleihs und der Aufbau eines Netzwerks aus zahlreichen Interessenten.\\
Mit der fortschreitenden Entwicklung diverser Sharing Angebote wäre zudem der Ausbau einer globalen Alternative solcher Anbieter möglich. Wie derzeit gängige Unternehmen Hotels, Campingplätze und Hostels gewerblich anbieten und über Suchmaschinen zu finden sind, könnte solch ein Netzwerk eine weitere Zielgruppe bedienen. Priorität hätte dabei der Aufbau sozialer Kontakte und das Nutzen neue Reisemöglichkeiten.


\subsubsection{Taktisches Ziele}
Mittelfristig ist es das Ziel einen aktiven Anwenderkreis aufzubauen und diese durch erfolgreiche, sichere und zufriedenstellende Erfahrungen beim Reisen mit dem System langfristig zu gewinnen.
Finanziell würde eine Entwicklung hinsichtliche Kooperationen mit Werbepartnern und Eventveranstaltern\footnote{Dazu mehr im Punkt 2.6 Geschäftsmodell} angestrebt werden, um das System auf Dauer gewinnbringend zu betreiben und weiter zu entwickeleln. \\
Funktional liesen sich mit dem technischen Fortschritt neue Ansätze einbauen (z.B. bei der Bezahlung, Navigation, Augmented Reality) und die Vorteile der Smartphones weiter ausnutzen. \\
In kürzerer (mittelfirstiger) Sicht das Unterstützen mehrer Betriebssysteme und Hersteller oder der Ausbau des Angebots auf Webpräsenz und weitere Geräte.


\subsubsection{Operative Ziele}
Ziel der kurzfristigen Entwicklungsphase ist die Entwicklung eines Vermietsystems für private Grundstücke als Aufenthaltsplatz für Reisende, auf Grundlage des Share Economy Konzeptes.\\
Das verteilte System soll vorhandene Angebote aufzeigen, die Kommunikation zwischen Mieter und Vermieter ermöglichen und den Mietprozess möglichst sicher sowie erfolgreich organisieren und abschließen können.\\
Die Motivation der Mieter liegt darin, kostengünstige Alternativen gegenüber herkömmlichen Aufenthaltsmöglichkeiten zu finden, dabei möglichst mobil zu sein sowie kurzfristige Optionen zu ermöglichen. Der Vermieter kann zum einen sozialen Nutzen daraus ziehen, dabei aber auch einen finanziellen Gewinn für die Vermietung erzielen. 


%\subsubsection{Optionale Ziele}


%\subsubsection{Minimale Ziele}

