%!TEX root = ../konzept.tex

\chapter{Zielhierachie}

TODO mehr Bezug zum Projekt
Das Projekt unterliegt einer gewissen Zielhierachie, die im Rahmen der Zielsetzung aufgestellt wurde. Dabei wurden zum einen high-level Ziele betrachtet, die auch über den Projektrahmen hinausgehen und eine mögliche Entwicklung im Kontext beschreiben, sowie konkretere Ziele, die sich direkt auf das Projekt anwenden lassen.

\subsubsection{Strategische Ziele}
Konkretisiert man den Betrachtungsraum auf den Projektrahmen, so ist das strategische Ziel die prototypische Entwicklung eines Verleihsystems für private Grundstücke als Campingplatz für Reisende. Schwerpunkt des Projektes liegt dabei auf der Konzipierung und Dokumentierung des Entwicklungsprozess und die Realisierung der Minimalziele.

\subsubsection{Taktisches Ziele}
Für das Projekt speziell bedeutet dies, das Aufstellen eines Interaktionsplans zwischen Vermieter und Mieter während eines Mietprozesses und das Herausstellen der einzelnen Aktivitätsschritte, die zur Aufgabenbewältigung notwendig sind.
Daraus einhergehend die funktionale Unterstützung dieser. Grob umfasst dies, die Angebotssuche und Mietanfrage des Mieters (Reisenden), den Kommunikationsaufbau zwischen Mieter und Vermieter und den Datenaustausch der Beteiligten, gefolgt vom Kommunikationsabschluss. (Zum Beispiel durch Bewertung und Bezahlung)

\subsubsection{Operative Ziele}
Operativ bedeutet dies, dass die einzelnen Aktivitätsschritte genau untersucht werden und darauf hin eine mögliche Realisierung geplant wird. Die Gestaltung einer möglichen Lösung der taktischen Schritte anhand verschiedener Technologien, die in diesem Zusammenhang ermittelt, abgewogen und umgesetzt werden sollten.
Genau bedeutet dies, mit Hilfe der Verteiltheit des Systems, einzelne Kommunikationsbereiche genauer zu betrachten.

\subsubsection{Minimal Ziele}
Schwerpunkt des Projektes liegt auf der Kommunikation der Mieter und Vermieter. Das Minimal Ziel ist das Finden eines eingetragenen Vermieters, das Stellen einer Anfrage und die daraus resultierende Kontaktaufnahme. Zusätzlich der damit verbundene Austausch der Benutzerdaten. Die Anfrage soll anhand von Matchingkriterien nur an relevante Anbieter weitergeleitet werden.

\subsubsection{Optionale Ziele}
Optionales Projektziel ist die Vertiefung des Matchingalgorithmen. Die grundlegende Funktionalität sollte das Minimalziel sein, die Feinheit und Komplexität der Kriterien, kann jedoch (voraussichtlich) nur in einem einfacheren Rahmen betrachtet werden.



