%!TEX root = ../konzept.tex

\chapter{Zielhierachie}
Das Projekt unterliegt einer gewissen Zielhierachie, die im Rahmen der Zielsetzung aufgestellt wurde. Dabei wurden high-level Ziele betrachtet, die sich über den gesamten Projektrahmen erstrecken und eine mögliche Entwicklung im Kontext beschreiben, sowie konkretere low-level Ziele, die sich direkt auf einzelne Aspekte des Projekt beziehen.

\subsubsection{Strategische Ziele}
Strategisches Ziel des Projektes, ist die prototypische Entwicklung eines Verleihsystems für private Grundstücke als Campingplatz für Reisende. Der Schwerpunkt liegt auf der Unterstützung der Anwender, indem Angebotssuche und -anfrage effektiver und einheitlicher gestaltet werden, als es vorhandene System ermöglichen. Zudem soll die Relevanz der Suchanfrage gesteigert werden und damit den Aufwand für Mieter und Vermieter verringern. 

\subsubsection{Taktisches Ziele}
Um die Kommunikation der Anwender zu vereinfachen, soll eine Smartphoneanwendung entwickelt werden, welche die Aktivitäten der Reisenden und Grundstücksbesitzer gleichermaßen unterstützt. Der Einsatz des Produktes soll auf die vorhandenen Umstände während einer Reise (in der Regel kein Laptop) ausgelegt werden und speziell auf die Anwendungsdomäne zugeschnitten sein. Der Einsatz hardwarespezifischer Funktionen (GPS und lokaler Speicher), soll dabei die ortsgebundene Suche übernehmen und sensible Benutzerdaten zur Wiederverwendung lokal absichern. Um geeignete Anwender miteinander in Verbindung zu bringen, übernimmt das System Arbeitsschritte des Reisenden und filtert Anfragen auf Grundlage der individuellen Benutzerinformationen.

\subsubsection{Operative Ziele}
Domänenspezifische Anforderungen werden im Rahmen einer ersten Konzeptphase ermittelt und anschließend vertiefend behandelt. Dies soll unter Verwendung geeigner Methoden der Mensch Computer Interaktion geschehen. Die einheitliche Anwendung soll als Smartphoneapplikation für Geräte mit Android Betriebssystem entwickelt werden. In dem Zusammenhang erfolgt die Konzipierung und anschließende Umsetzung einer passenden Softwarearchitektur. Nach der ersten theoretischen Auseinandersetzung innerhalb der Konzeptphase, sollen angedachte Funktionalitäten innerhalb der Proof of Concepts getestet werden. Während dieser Zeit werden Anforderungen durch MCI Modelle genauer spezifiziert und Gestaltungslösungen zur Umsetzung erarbeitet. Um eine optimierte Variante der Anwendung zu erhalten, sollen Iterationen innerhalb des Prototypings, sowie Evaluationen mit realen Benutzern durchgeführt werden (sofern sich geeignete Personen finden lassen). 

\subsubsection{Minimal Ziele}
Schwerpunkt des Projektes liegt auf der Kontaktaufnahme der Mieter und Vermieter. Das Minimal Ziel ist es daher, Reisende und Vermieter untereinander zu vermitteln, sofern sie geeignete Kriterien erfüllen, die einen erfolgreichen Mietvorgang ermöglichen. Im ersten Schritt bedeutet dies, dass die Position und Zielumgebung des Reisenden aufgenommen wird und darauf aufbauend geeignete lokale Anbieter findet. Anschließend wird die Kompatibilität des Reisenden und Grundstücksbenutzers bestimmt und führt gegebenfalls zur Mietanfrage. Die daraus resultierende Kommunikation soll initiiert werden und einen nachrichtenbasierten Austausch ermöglichen.

\subsubsection{Optionale Ziele}
Optionales Projektziel ist die Vertiefung des Matchingalgorithmus, der dafür zuständig sein soll, die Informationen der Benutzer zu vergleichen und  relevante Suchanfragen zu ermitteln. Die grundlegende Funktionalität sollte das Minimalziel sein, die Feinheit und Komplexität der Kriterien, kann jedoch (voraussichtlich) nur in einem einfacheren Rahmen betrachtet werden.
Dazu kommt die Möglichkeit einer Userbewertung. Die Anbindung an einen Zahlungsanbieter wird innerhalb dieses Projekt nicht realisiert werden können, diese Schritte werden lediglich prototypisch eingebunden.
