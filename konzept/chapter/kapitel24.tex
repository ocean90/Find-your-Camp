%!TEX root = ../konzept.tex

\section{Marktanalyse}
     http://freagle.org
    Freagle basiert auf Gemeinschaft und Gastfreundschaft. Deshalb musst Du zunächst ein Camp hinzufügen, in dem andere Mitglieder ihr Zelt für eine Nacht aufschlagen können.
      Die Verifizierung geschieht per Post und dient Deiner eigenen Sicherheit. Außerdem bekommst Du auch deine persönliche Freagle-Card zugeschickt, die Du zum Zelten benötigst.
     http://campinmygarden.com
      Anzeige der örtlichen Events
      “Response Rate”
      Couchsurfing
      Meldung über Suchende in der Umgebung
      ADAC Camping- und Schnellplatzführer 2013
      Nur gewerbliche Zeltplätze
       
      Ähnliche Ansätze verfolgen Webseiten wie freagle.org oder campinmygarden.com. Beide ermöglichen das Suchen solcher Angebote, setzen dabei auf ein Reputationssystem und bieten eine gewissen Sicherheit durch notwendige Verifikation. Keine der bekannten Webseiten bieten dabei jedoch eine mobile Variante, die speziell für Reisende eine deutliche Erleichterung bieten kann. Existierende Apps wie der ADAC Campingplatz Finder setzen den Fokus lediglich auf öffentliche Campingplätze, die aufgrund des Preises, für die identifizierte Zielgruppe oftmals nicht in Frage kommen.\\
