%!TEX root = ../konzept.tex


\section{Alleinstellungsmerkmale und Chancen}
Mit speziellem Fokus auf das Grundstück sharing, ergeben sich aus Marktanalyse und anfänglicher Anbieteruntersuchung (Couchsurfing) verschiedene Alleinstellungsmerkmale des Find your Camp Systems.\\

Das Beispiel Couchsurfing setzt vorallem auf den sozialen Ansatz und ist darauf ausgelegt neue Bekanntschaften zu schließen. Reisende sind meistens mit wenigen Personen unterwegs, übernachten einige Tage bei ihrem Host und lassen sich von ihm die Stadt und Kultur zeigen.
Geeignet ist dieser Ansatz weniger bei größeren Reisegruppen oder Familien. Zusätzlich liegt vorallem auf der Vermieter sein kein finanzieller Gewinn und das Eindringen in seinen privaten Lebensraum kann viele potentielle Nutzer abschrecken.
Airbnb ermöglicht die private Vermietung, der Kostenpunkt ist jedoch weiterhin hoch und in beliebten Gegenden ist weiterhin eine Reservierung von nöten. 
Grundsätzlich lassen sich in den Beispielen positive Ansätze finden, die beibehalten und ausgebessert werden können. Vorallem aber die Negativpunkte lassen sich ausbessern.\\ Aus allen Betrachtungen ergeben sich damit folgende Ansatzpunkte für eine auf die Problemdomäne fokusierte Anwendung:\\

\subsection{Mieter und Vermieter}
\begin{itemize}
   \item
   \textbf{Einheitliches Kommunikationssystem}: Die Kommunikation muss nicht über eine Webpräsenz oder diverse unterschiedliche Wege stattfinden. (Nachrichtensystem der Webpräsenz, Email, Telefon). Auch die Bezahlung kann auf diesem Weg abgeschlossen werden.

   \item 
   \textbf{Weitestgehende Kontrolle der Privatsphäre und der eigenen Daten.}

   \item 
   \textbf{Zeitoptimierung}: Gängige Beispiele setzen auf Angebot und Nachfrage Anzeigen, diese können veralten oder nicht aktuell sein und zu spät gelesen werden. Der passende Vermieter erhält im neuen System direkte Anfragen die zeitliche und inhaltliche Relevanz haben und kann diese direkt beantworten. Der Mieter erhält in kürzerer Zeit eine Antwort.

   \item
   \textbf{Filtern relevanter Anfragen}: Anhand der Benutzerdaten, findet eine Kontaktaufnahme nur zwischen kompatiblen Benutzern statt. Dadurch verringert sich die Anzahl zielloser Anfragen und Kommunikationen.

   \item 
   \textbf{Zeitliche Unabhängigkeit}: Während der Vermietung besteht die Möglichkeit, dass alle Beteiligten ihre Aktivitäten unabhängig voneinander ausführen können. Der Mieter ist nicht unbedingt auf den Zugang durch den Vermieter angewiesen.

\end{itemize}


\subsection{Vermieter}
\begin{itemize}
   \item 
   \textbf{Finanzieller Anreiz}: Vermieter haben die Möglichkeit für ihre Vermietung Kosten zu erheben und finanziellen Gewinn zu schlagen.
   Dabei muss es sich nicht nur um die Verleihung der Wohnung handeln, sondern grundsätzlich vorhandene Grundstücke wie Gärten, Hof, Landstücke.

   \item 
   \textbf{Kontrolle der Privatsphäre}: Im Gegensatz zum Couchsurfing lässt sich der Bereich eingrenzen, indem Reisende in den eigenen Lebensraum eindringen können. Auch ein permanenter sozialer Kontakt ist nicht von nöten, sodass der ganze Prozess auf einer rein geschäftlichen Ebene ausgetragen werden kann.

   \item 
   \textbf{Sicherheit der Daten}: Sensiblen Informationen können nur bei Bedarf freigegeben werden und sind nur lokal gespeichert. Damit erhalten nur Kunden auch die benötigten Informationen und das freie einsehen über eine Webpräsenz ist nicht möglich.

\end{itemize}
   

\subsection{Mieter}
\begin{itemize}
   \item 
   \textbf{Mobilität}: Eine mobile Anwendung unterstützt die verbreiteste Technologie, die Reisende in der Regel mit sich führen. Zusätzlich dazu können die unique Features eines Smartphones Anwendung finden.
   Er ist flexibler und muss sich nicht um Zugung zu stationären Rechnern kümmern.

   \item 
   \textbf{Vergrößerte Reise- und Interessentengruppe}: Es ist möglich mit einer größeren Anzahl an Personen zu verreisen, die innerhalb einer Wohnung nicht untergebracht werden können. Dazu bestünde auch die Option Familien mit Kindern unterzubringen (falls man diese als Host nicht aufnehmen würde). 


   \item
   \textbf{Soziale Einstellung berücksichtigen}: Nicht jeder möchte viel Kontakt mit seinen Host haben und es besteht die Möglichkeit, dass sich Leute auf den geschäftlichen Prozess beschränken wollen. Eventuell besteht kein Interesse an den sozialen Aspekten des Sharings.

   \item 
   \textbf{Spontanität}: Reisende können auf ihrer Reise spontanere Suchanfragen starten und müssen sich nicht zwangsläufig an vorgegebene Routen halten. 

\end{itemize}
     

Zusätzlich hebt sich das System von vorhanden Angeboten ab, da der Fokus auf private Grundstücke liegt, die nicht innerhalb der eigenen Wohnung liegen und die aufgrund dessen noch keine verstärkte Wahrnehmung erhielten. Die Gestaltung als Applikation unterstützt die natürliche Flexibilität eines Reisenden und ermöglicht ihm neben dem Finden eines Angebotes auch die Abwicklung der Kommunikation sowie gegebenfalls das Orientieren durch die GPS Funktion. 
