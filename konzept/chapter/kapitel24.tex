%!TEX root = ../konzept.tex


\section{Alleinstellungsmerkmale und Chancen}
Mit speziellem Fokus auf das Grundstück Sharing, ergeben sich aus Marktanalyse und anfänglicher Anbieteruntersuchung (Couchsurfing) verschiedene Alleinstellungsmerkmale des Find your Camp Systems.\\

Das Beispiel Couchsurfing setzt vorallem auf den sozialen Aspekt und ist darauf ausgelegt neue Bekanntschaften zu schließen. Reisende sind meistens mit wenigen Personen unterwegs, übernachten einige Tage bei ihrem Host und lassen sich von ihm die Stadt und Kultur zeigen.
Geeignet ist dieser Ansatz weniger bei größeren Reisegruppen oder Familien. Zusätzlich liegt vorallem auf der Vermieterseite kein finanzieller Gewinn und das Eindringen in seinen privaten Lebensraum kann viele potentielle Nutzer abschrecken.
Airbnb ermöglicht die private Vermietung, der Kostenpunkt ist jedoch weiterhin hoch und in beliebten Gegenden ist weiterhin eine Reservierung von nöten. 

Grundsätzlich lassen sich in den Beispielen positive Ansätze finden, die beibehalten und ausgebessert werden können. Vorallem aber die Negativpunkte sollen ausgebessert werden.\\ Aus allen Betrachtungen ergeben sich damit folgende Ansatzpunkte für potentielle Optimierungen:

\subsection{Mieter und Vermieter}
\begin{itemize}
   \item
   \textbf{Einheitliches Kommunikationssystem}: Die Kommunikation muss nicht über eine Webpräsenz oder diverse unterschiedliche Wege stattfinden. (Nachrichtensystem der Webpräsenz, Email, Telefon), sondern wird von allen Anwendern über die gleiche Software geschehen. Auch die Bezahlung kann auf diesem Weg abgeschlossen werden.

   \item 
   \textbf{Zeitoptimierung}: Gängige Beispiele setzen auf Angebot und Nachfrage Anzeigen, diese können veralten oder nicht aktuell sein und zu spät gelesen werden. Der passende Vermieter erhält im neuen System direkte Anfragen die zeitliche und inhaltliche Relevanz haben und kann diese direkt beantworten. Der Mieter soll dadurch in kürzerer Zeit eine Antwort erhalten.

   \item
   \textbf{Filtern relevanter Anfragen (1)}: Anhand der Benutzerdaten, findet eine Kontaktaufnahme nur zwischen kompatiblen Benutzern statt. Dadurch verringert sich die Anzahl zielloser Anfragen und Kommunikationen.

   \item 
   \textbf{Zeitliche Unabhängigkeit}: Während der Vermietung besteht die Möglichkeit, dass alle Beteiligten ihre Aktivitäten unabhängig voneinander ausführen können. Der Mieter ist (im Vergleich zu Couchsurfing) nicht unbedingt auf den Zugang zur Wohnung durch den Vermieter angewiesen. 

\end{itemize}


\subsection{Vermieter}
\begin{itemize}
   \item 
   \textbf{Finanzieller Anreiz}: Vermieter haben die Möglichkeit für ihre Vermietung Kosten zu erheben und finanziellen Gewinn zu schlagen.
   Dabei muss es sich nicht nur um die Verleihung der Wohnung handeln, sondern grundsätzlich vorhandene Grundstücke wie Gärten, Hof, Landstücke.

   \item 
   \textbf{Kontrolle der Privatsphäre}: Im Gegensatz zum Couchsurfing lässt sich der Bereich eingrenzen, indem Reisende in den eigenen Lebensraum eindringen können. Auch ein permanenter sozialer Kontakt ist nicht von nöten, sodass der ganze Prozess auf einer rein geschäftlichen Ebene ausgetragen werden kann.

   \item 
   \textbf{Sicherheit der Daten (2)}: Sensiblen Informationen können nur bei Bedarf freigegeben werden und sind nur lokal gespeichert. Damit erhalten nur Kunden auch die benötigten Informationen und das freie Einsehen über eine Webpräsenz ist nicht möglich.

\end{itemize}
   

\subsection{Mieter}
\begin{itemize}
   \item 
   \textbf{Mobilität (3)}: Eine mobile Anwendung unterstützt die verbreiteste Technologie, die Reisende in der Regel mit sich führen. Zusätzlich dazu können die unique Features eines Smartphones anwendung finden.
   Er ist flexibler und muss sich nicht um Zugung zu stationären Rechnern kümmern.

   \item 
   \textbf{Vergrößerte Reise- und Interessentengruppe}: Es ist möglich mit einer größeren Anzahl an Personen zu verreisen, die innerhalb einer Wohnung nicht untergebracht werden können. Dazu bestünde auch die Option Familien mit Kindern unterzubringen (falls man diese als Host nicht aufnehmen würde). 

   \item
   \textbf{Soziale Einstellung berücksichtigen}: Nicht jeder möchte viel Kontakt mit seinen Host haben und es besteht die Möglichkeit, dass sich Leute auf den geschäftlichen Prozess beschränken wollen. Eventuell besteht kein Interesse an den sozialen Aspekten des Sharings.

   \item 
   \textbf{Spontanität}: Reisende können auf ihrer Reise spontanere Suchanfragen starten und müssen sich nicht zwangsläufig an vorgegebene Routen halten. Theoretisch ist dies auch unter bereits vorhanden umständen möglich (Reisender entscheidet in diesem Ort selbstständig eine Unterkunft zu suchen), aber die Applikation unterstützt hierbei speziell die Suche. Im Gegenzug dazu, verliert der Vermieter jedoch organisatorische Sicherheit und es kann die Gefahr auftreten, dass Anfragen nicht angenommen werden können, da sie zu kurzfristig erscheinen.

\end{itemize}

Viele dieser Punkte, ergeben sich dabei auch als Folge des Handlungskontextes. Gerade auf Mieterseite werden einige Chancen ermöglicht, jedoch lässt sich nicht mi Sicherheit auf diese setzen und haben für die Entwicklung des Projektes keine größere Priorität. 

Speziell die mit 1 - 3 markierten Aspekte, sollen in der weiteren Betrachtung fokusiert werden, da sich diese als Alleinstellungsmerkmale auszeichnen und für das System als besonders relevant erachtet werden.
 
