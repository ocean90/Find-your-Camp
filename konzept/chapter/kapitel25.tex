%!TEX root = ../konzept.tex

\section{Risiken}
Zusätzlich zu den Chancen, wurden im weiteren auch potentielle Risiken betrachtet, die sowohl mit der grundlegenden Thematik als auch mit der späteren Umsetzung während des Projektes auftreten können.
Dieser Schritt diente zudem der Abwägung, ob das Verhältnis zwischen Chancen und Risiken letztendlich zu einem gewinnbringenden und zielerfüllenden Ergebnis führen kann.\\
TODO genauer Ausformulieren und Ansatzpunkte zum Entgegenwirken ergänzen

\begin{itemize}
   \item \textbf{Sicherheitsaspkete}: Sowohl auf Daten bezogen, als auch beim Kontakt mit den Leuten. Daher sollte eine Verifikation der Anwender stattfinden und den Benutzern die Freiheit über ihre Daten gewährleistet werden.

   \item \textbf{Aktivität der Nutzer}: Das System wächst und fällt mit der Benutzerbeteiligung. Angesprochene Zielgruppe wird vom System nicht ausreichend angesprochen. Sowohl durch qualitative als auch funktionale Aspekte. Langfristige Bindung und das schaffen positiver Erfahrungen ist notwendig und sollte durch ein Reputationssystem gesteigert werden.

   \item \textbf{Geschäftsmodell}: Finanzielle Absicherung muss gewährleistet sein, da sich die Kosten des Projektes ansonsten nicht decken lassen 
   (Appkosten, Kaufabwicklung über App, Werbung)

   \item \textbf{Eingeschränkte Zielgruppe}:
   Aufgrund technischer Einschränken enthält nur ein geringer Anteil der potentiellen Interessenten Zugang zum System. Das kann sich auf die verfügbare Hardware beziehen, sodass ältere Interessenten beispielsweise keine Möglichkeit haben das Programm zu benutzen oder Einschränkungen in der Verbreitung. D.h zB das nur Androiduser ab bestimmten Betriebssystemversionen die Applikation einsetzen können.
     
   \item
   \textbf{Mieten ja, Vermieten nein}: Motivation als Vermieter ist nicht für die breite Masse gegeben. Jüngere, aufgeschlossene Leute die auf soziale Kontakte etc. aus sind würden teilnehmen, besitzen in der Regel aber kein eigenes Haus oder Land zum vermieten. Daher muss auch für Landbesitzer unterschiedlichen Alters oder Einstellungen eine deutliche Motivation und Sicherheit gewährleistet werden.

   \item
   \textbf{Software und Hardware spezifische Probleme}:
   Probleme in der Funktionalität können während der Entwicklung beachtet werden, Hardwarefehler nicht zwangsläufig. Wichtig ist das Schaffen von Kompromissen in Fehlersituationen. 
   Da die Mieter in unterschiedlichen Gegenden unterwegs sind und beispielsweise nicht immer Verbindung zum Internet haben durch Netzabdeckung, sollte man sich dahingehend Ansatzpunkte überlegen. Z.b. nach Annahme eines Angebotes werden die Koordinaten des Vermieters lokal Zwischengespeichert, sodass auch ohne Verbindung der standort gefunden werden kann und Kontaktmöglichkeiten wie Telefonnummer gesichert sind.


    
      
\end{itemize}
 