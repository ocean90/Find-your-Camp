%!TEX root = ../konzept.tex

\section{Risiken}
Zusätzlich zu den Chancen, wurden im Weiteren auch potentielle Risiken betrachtet, die sowohl mit der grundlegenden Thematik als auch mit der späteren Umsetzung während des Projektes auftreten können.
Dieser Schritt diente zudem der Abwägung, ob das Verhältnis zwischen Chancen und Risiken letztendlich zu einem gewinnbringenden und zielerfüllenden Ergebnis führen kann.

\begin{itemize}
   \item \textbf{Sicherheitsaspekte}:\\ Sowohl auf Daten bezogen, als auch beim Kontakt mit den Leuten. Wenn die Vermieter ihre persönlichen Daten angeben, gewähren sie Interessenten einen gewissen Einblick in ihr Eigentum. Kommt dazu noch die Lokation durch GPS, besteht potentielle Einbruchgefahr. Daher sollte eine Verifikation der Anwender stattfinden und den Benutzern die Freiheit über ihre Daten gewährleistet werden. Auch der Fremdzugriff bei verlorenen oder gestohlenen Geräte sollte bedacht werden (Codesperre, Identifikation). Da die gesamte Bezahlung über die Anwendung abgeschlossen werden soll, muss auch dahingehend eine gewisse Sicherheit gewährleistet werden. Um vorallem die permanenten Daten zu schützen, soll sich die Systemarchitektur diesem Problem annehmen\footnote{Kapitel 4.2 Konzeptseite 30}.

   \item \textbf{Aktivität der Nutzer}:\\ Das System wächst und fällt mit der Benutzerbeteiligung. Gefahr besteht, wenn angesprochene Zielgruppe nicht ausreichend vom System angesprochen. (Sowohl durch qualitative als auch funktionale Aspekte.) Zu einer langfristigen Bindung und das Schaffen positiver Erfahrungen, kann ein Reputations- und Reviewsystem eingebaut werden, sowie eine Freunde/ Bekanntenfunktion mit der Kontakte gehalten werden können. (Soll aber nicht in diesem Projekt betrachtet werden.)

   \item \textbf{Kosten}:\\ Um das Projekt langfristig betreiben zu können, muss die finanzielle Absicherung gewährleistet sein. Daher soll ein mögliches Geschäftsmodell erarbeitet werden, mit welchen das System die Kosten decken kann\footnote{Kapitel 5 Konzeptseite 36}.

   \item \textbf{Eingeschränkte Zielgruppe}:\\
   Aufgrund technischer Einschränken enthält nur ein geringer Anteil der potentiellen Interessenten Zugang zum System. Das kann sich auf die verfügbare Hardware beziehen, sodass ältere Interessenten beispielsweise keine Möglichkeit haben das Programm zu benutzen oder Einschränkungen in der Verbreitung. Wenn zum Beispiel nur bestimmte Betriebssystemversionen unterstützt werden. Das Projekt betrachtet nur die Entwicklung für Geräte mit Android Betriebssystem, in Zukunft wäre eine Portierung auf weitere Plattformen eine Möglichkeit zur weiteren Verbreitung.
     
   \item
   \textbf{Mieten ja, Vermieten nein}:\\ Motivation als Vermieter ist nicht für die breite Masse gegeben. Jüngere und grundsätzlich aufgeschlossene Leute, die auf soziale Kontakte etc. aus sind, würden teilnehmen, besitzen in der Regel aber kein eigenes Haus oder Land zum vermieten. Daher muss auch für Landbesitzer unterschiedlichen Alters oder Einstellungen eine deutliche Motivation und Sicherheit gewährleistet werden. Demnach muss das System auch unerfahrenen Anwendern einen leichten Einstieg und ausreichende Nutzungsmotivation bieten\footnote{Kapitel 3.3 Konzeptseite 20}.

   \item
   \textbf{Lokale Verbreitung}:\\ Angebotene Grundstücke sind nicht in allen Regionen vorhanden. In ländlichen Gegenden ist die Wahrscheinlichkeit größer, dass Grundstücksbesitzer vorhanden sind, während in Großstädten keine Zielgruppe vertreten ist. Speziell dazu muss auch betrachtet werden, wer eigentlich in diese ländlichen Gegenden reisen würde und ob diese Zielgruppe dann entsprechende Technologien besitzen.

   \item
   \textbf{Software und Hardware spezifische Probleme}:\\
   Probleme in der Funktionalität können während der Entwicklung beseitigt werden, Hardwarefehler nicht zwangsläufig. Wichtig ist das Schaffen von Kompromissen in Fehlersituationen. 
   Da die Mieter in unterschiedlichen Gegenden unterwegs sind und nicht immer Verbindung zum Internet durch Netzabdeckung haben, sollte man sich dahingehend Ansatzpunkte überlegen. Z.b. nach Annahme eines Angebotes werden die Koordinaten des Vermieters lokal zwischengespeichert, sodass auch ohne Verbindung der Standort gefunden werden kann und Kontaktmöglichkeiten wie Telefonnummer gesichert sind. Auch der Zeitpunkt der Nachrichtenübermittlung kann ausschlaggebend sein. Da die Anwendung nicht permanent aktiv ist, eignet sich vorallem ein Kommunikationsmodell, bei dem Nachrichten in der Cloud zwischengespeichert werden und bei Aktivität weitergeleitet werden. Es sollte nicht damit gerechnet werden, dass Reisende und Vermieter zur gleichen Zeit aktiv sind. 

\end{itemize}

Als wesentliches Risiko dieser Untersuchung, ging die Verfügbarkeit der Vermieter hervor. Das Prinzip lässt sich nur dann erfolgreich verwirklichen, wenn gesuchte Grundstücksbesitzer auf das System aufmerksam werden, benötigte Technologien besitzen, mit der Anwendung interagieren können und eine Motivation besitzen ihr Grundstück zu vermieten. 
