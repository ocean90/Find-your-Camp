%!TEX root = ../konzept.tex

\section{Weiteres MCI Vorgehen}
Für die kommende Projektentwicklung ist vorgesehen mit dem angefangenen requirement engeenering fortzufahren. 
Nach der Identifikation der Stakeholder, ihren Bedürfnisse und dem Nutzungskontext, soll eine Task Analysis folgen.\\
Zudem sollen Interkationen und Aufgabenmodelle dokumentiert werden  um strukturelle und funktionale Anforderungen an die Software zu ermitteln. Denkbare Ansätze sind abstraktere Aktivitäts-/Interaktions- und Informationsszenarien. Diese dienen der Konzipierung des preskriptiven Models, nachdem das deskriptive Modell innerhalb dieses Konzeptes bereits zum großen Teil abgedeckt wurde.

Auch die Entwicklung von Use Cases nach Allistar Cockburn oder essentiell use cases, sowie eine Analyse von Aktivitäten durch die HTA Methode stellen valide Optionen dar und werden im späteren Zusammenhang untersucht. 