%!TEX root = ../konzept.tex

\section{Weiteres MCI Vorgehen}
Für die kommende Projektentwicklung ist vorgesehen, angefangene Benutzermodellierung zu vertiefen und über User Profiles, User Roles und/oder Personae konkretere Ansätze zu verfolgen. 
Anschließend soll auf Grundlage der Benutzermodellierung eine Auseinandersetzung mit ihren speziellen Anforderungen erfolgen. Erste Gedanken und Ansätze fanden bereits in diesem Konzept statt, diese sollen  auf konkrete Interaktionen und Aufgaben angewendet werden.\\
Möglichkeiten zur Dokumentation sind Szenarien in Form von Aktivitäts-/Interaktions- oder Informationsszenarien. Diese dienen der Konzipierung des preskriptiven Models. Auch für die spätere Umsetzung der Funktionalitäten dient dieser Schritt, um die konkreten Interaktionen zwischen den Benutzern zu ermitteln.

Auch die Entwicklung von Use Cases, beispielsweise nach dem Template von Allistar Cockburn oder den abstrakteren essentiell use cases, sowie eine Analyse von Aktivitäten durch die HTA Methode stellen valide Modelle dar und werden im späteren Zusammenhang untersucht\footnote{Informationen zu Use Cases, Szenarienmethoden und Persona im Draft ab S. 309 sowie im Werk von David Benyon ab S.50 }.  

Nach dem Verständnis der Nutzungsanforderungen und Aufgabenstrukturen, sollen Ansätze zur Gestaltung erarbeitet werden und durch Evaluationstechniken (im Idealfall mit Einbezug realer Stakeholder) geprüft werden.