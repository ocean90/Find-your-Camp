%!TEX root = ../konzept.tex

\section{Weiteres MCI Vorgehen}

Welches sind die Aufgaben, welche Struktur weisen sie auf und in welchen Beziehungen stehen die Aufgaben zueinander?

\subsection{Vorgehensmodelle}

{\color{blue}Requirement Engeneering
Stakeholder + Bedürfnisse + Nutzungskontext + task analysis (+-)
-> Problemszenarien: Aktivitäts/Interaktions/Informationsszenarien
oder
-> context models, concrete use cases
 
Analyse deskr. -> Problemszenarien: Handeln in aktueller Situation
ENtwurf presk: -> Informationsszenarien: wann werden welche Informationen in der Handlung benötigt
        	       	-> Aktivitätsszenarien: Aktionen in Handlung
                   	-> Interaktionsszenarionen: Interaktionszyklen
 }


\subsection{Dokumentationstechniken der Aufgaben}
 Szenarien, Use Cases, HTA, GOMS

{\color{blue}
 - Häufigkeitsgrad, Variationsgrad, notwendiges Wissen, Grad der Einflussnahme
        	- Task Scenario: erzählerische, aktuelle Anwendungsbeschreibung
        	- Use Scenario: narrativ, angestrebte Verwendudng des Systems basierend auf Anforderungen und zeigt wie system zu verwenden sein wird
 
koginitive, motivationale Aspekte einbezihene, Vor und Nachbedingungen
Szenario:
        	- Geschichte, Aufgaben, Aktivitäten, Interaktion
        	- konkreter Kontext
        	- Instanz des Use Cases
Pros/ COns bei Problemszenario
 
Use case: Beziehung stakeholder <-> technisches System
        	- Wie wird das System genutzt? Was machen Leute? Was macht System?
        	
strukturiert, stringenter Ansatz, tendietiell linearisierte Darstellung der Interaktion
        	- Verhalten als Antwort auf Anfrage des Primary Users
 
        	-> verschiedene Sequenzen -> verschiedene Szenarioen
 
Methode: Allistar Cockburn Template
 
Concrete Use Case
user intention/action | systems responsibility
- Interface Metaphern
- detaillierte Beschreibung, nicht personalisiert, generisch
essential use case
        	- strukturiert, narrativ, domain und user sprache, abstrakt, technologiefrei, implementations unabhängig, neutrale Konnotation
        	- beschreibt Aufgabe mit hohem Abstraktionsgrad ohne Interaktionsparadigmen
        	- Fokus was will user machen/ System verantwortung
-> Ermittlung/Validierung der Anforderungen
- funktionale Anforderungen, Interessen Stakeholder, Systemverhalten, Detaillierte Informationen zu Fehlerkonditionen und Behandlungen
 
HTA:    + kogintive Perspektive (als einzige Methode)
        	+ Wissen graphisch dargestellt
        	+ geringe Wissensbarriere
        	+ Skalierbarkeit
- Analyseaspekt, Arbeit Modellhaft repräsentieren
- Relation Aktivität <-> kogn. Perspektive
- Aufgabe hinsichtlich Struktur analysieren (aus koginitiver Perspektive)
- Dekomposition: + goals/ Tasks/ operations
Arbeitsschritte:
1. Zielsetzung Analysieren
2. Konsens Stakeholder, goals
3. Methoden/ Quellauswahl
4. erste Dekomposition in Diagram/ Tabellenform


 
Welches ist das konzeptuelle Modell für das präskriptive Aufgabenmodell? Welche Bezüge existieren bzgl. der Entwicklungsziele?
präskriptives Modell: “Soll” Zustand
working engeneerings
descriptive -> claim analysis -> prescriptive
 	Welche alternativen konzeptuellen Modelle wurden entwickelt und wie wurde mit Design-Alternativen verfahren?

}
 
