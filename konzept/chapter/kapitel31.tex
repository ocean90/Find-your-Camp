%!TEX root = ../konzept.tex

\section{Benutzermodellierung}
Die erste Stufe der Benutzermodellierung dient der Identifikation potentieller Nutzergruppen des Systems.
Das Spektrum an Interessenten und Betroffenen innerhalb der Anwendungsdomäne, wird dazu in
verschiedene Benutzerklassen eingeteilt. Es werden alle Stakeholder gesucht, die ein Anrecht, Anspruch oder Interesse am System haben und in Zukunft entwickeln könnten. Zudem wurde ein erster Blick auf deren Charakteristiken gewagt und der Einstieg zu entsprechenden User Profiles geschaffen.\\
 
Als primary User des Systems, letztendlich die Enduser, die direkt mit der Anwendung interagieren, wurden folgende Benutzer ermittelt:
 
Zum einen wird die Anwendung von Interessenten in der Tätigkeit des \textbf{Mietenden} verwendet.
Diese können grundsätzlich aus allen Altersgruppen stammen, vorausgesetzt sie sind voll geschäftsfähig, da ein Mietgeschäft stattfindet, und sie sind im Besitz der benötigten Hardware.
Eingrenzen lässt sich das Benutzerfeld weiterhin auf Reisende oder Touristen, die eine Unterkunft benötigen und im Regelfall Campingausrüstung bei sich tragen. In diese Kategorie fallen demnach auch Personen wie Backpacker, Langzeitreisende, Wanderer.
Sie suchen entweder aktiv selbst oder sind diejenigen, die mit der Funktionalität im späteren Verlauf, beispielsweise bei einem QR Code Check\footnote{Ob diese Funktion im letztendlichen System so anwendung findet ist nicht festgelegt.}, direkt in Kontakt mit der Anwendung kommen.
 
Die zweite Interessentengruppe sind die \textbf{Vermieter}, die ihr Grundstück als Unterkunft zur Verfügung stellen. Auch hier wirkt vorherige Einstufung der Altersgruppe, wobei die Verfügbarkeit der Hardware in der Regel zu einer gewissen Altersgrenze führen wird, die Schätzungsweise bis 50 Jahre geht.
Vermieter in der Rolle des primary users sind Grundstücksbesitzer bzw. Garteneigentümer und Personen, welche während des Mietprozesses aktiv mit der Anwendung interagieren.\\
 
Weitere Betrachtung widmet sich den secondary usern, welche nicht regelmäßig selbst mit der Applikation interagieren. Sie liefern den primary usern entsprechenden Input, diese benutzen die Anwendung dann an ihrer Stelle als \textbf{Zwischennutzer} und geben den erhaltenen Output zurück. Für diese Benutzergruppe wurde auf Mieterseite die \textbf{„erweiterte“ Reisegruppe} definiert. Jeder der neben dem eigentlichen Benutzer die Unterkunft benutzt und im Vorfeld aktiv an der Suche beteiligt ist, indem beispielsweise Anhaltspunkte gegeben werden in welchem Umkreis gesucht werden soll.
Für den Vermieter ergibt sich hierbei der Personenkreis, der selbst ein Grundstück zur Verfügung stellen kann und möchte, aber entsprechende Technologien nicht besitzt und das Angebot jemand anderen Abwickeln lässt. Die Anzahl dieser Anwendertypen wird im Vergleich zu den vorherigen Benutzergruppen jedoch eher gering eingeschätzt.\\
 
Als weitere Stakeholder im Bereich der Entscheidungsträger für spätere Anschaffung und Benutzung (tertiary user) wurden \textbf{potentielle Werbepartner} identifiziert, die als Teil des Geschäftsmodells zum Beispiel bei Events auftreten können. Zusätzlichen Einfluss kann die Anwendung außerdem auf die \textbf{Stadt} in Form von Tourismus haben und dabei auch mit \textbf{bestehenden Unterkunftsanbietern} wie Hotels, Hostels oder öffentlichen Campingplätzen konkurrieren. Auch wenn diese nicht zwangsläufig als Nutzer der Anwendung auftreten, so haben sie Interesse am möglichen Erfolg und Misserfolg und können dementsprechend beeinflusst werden.
Dazu kommen \textbf{Personen aus organisatorischen Bereichen} wie System Administratoren oder Supportmitarbeiter.\\
 
 Zusammenfassend ergeben sich damit folgende Stakeholdergruppen:
 \begin{itemize}
   \item
   Mieter: Reisende als Endnutzer und gesamte Reisegruppe

   \item 
   Vermieter: Grundstücksbesitzer und angehörige Familienmitglieder

   \item
   potentielle Interessenten: Werbepartner, Unternehmen, Stadt

   \item 
   vorhandene Unterkunftsanbieter, speziell Campingplätze

   \item
   Systemadministratoren, Support\\
\end{itemize} 


Für die spätere Auseinandersetzung ist es notwendig die für das Projekt wichtigsten Stakeholdergruppen zu bestimmen und dahingehend die Anforderungsermittlung durchzuführen.
In Anbetracht des Projektziels fallen hierbei vorerst die primary und secondary user auf, die in einer detaillierten Auseinandersetzung innerhalb der User Profiles genauer bestimmt werden sollen.
Bei der Charakterisierung ihrer Merkmale sollte dabei auf sinnvolle Eigenschaften geachtet werden, die für das System von Relevanz sind. Eine Herausforderung dabei wird auch das Abwägung der Prioritäten und das Eingrenzen der Individuen in gemeinsame Eigenschaften. Die Darstellung soll dabei möglichst abstrakt sein und in einer neutralen Form aufbereitet werden.
 
Da die menschzentrierte Entwicklung ein iterativer Prozess ist, kann zu späteren momenten eine erneute Auseinandersetzung mit den gewonnen Stakeholdern stattfinden.
Ansätze hierbei wäre ein Perspektivwechsel nach Tätigkeitsperspektive, Rollenperspektive, Interessenperspektive oder kulturelle Perspektive sowie Ergebnisse aus weiterer Marktanalyse und Recherche. Eventuell auch unter Einbezug möglicher Interessenten. (Interviews mit echten Campern etc.) Für den Fall, dass im Projektverlauf Probleme bei der Interaktionsmodellierung auftreten, wäre eine erneute Auseinandersetzung denkbar. Wie erwähnt soll der Fokus  jedoch verstärkt auf der eigentlichen Benutzung, anstelle der Nutzergruppen liegen.\\
 


 


