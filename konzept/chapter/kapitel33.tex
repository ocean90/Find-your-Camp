%!TEX root = ../konzept.tex

\section{Nutzungskontext}

TODO ausformulieren und überarbeiten

Benutzer und Interessentengruppen aus zuvor Identifizierten Stakeholdern. Ausschlaggebende Merkmale die Einfluss haben können:
\begin{itemize}
   \item Altersklassen: Nicht exakt definierbar, grundsätzlich jeder der ein Smartphone besitzt und auf diese Art und Weise Reisen will. Kernzielgruppe wird zwischen 16 - 50 Jahre geschätzt. Prinzipiell könnten auch Personen darüber hinaus Interesse an der Applikation haben, aufgrund von eigenen Einstellungen spricht der sozialere Aspekt diese aber eher weniger an und sind wahrscheinlich nicht mit Campingausrüstung längere Zeit unterwegs.

   \item Einkommen:

   \item Was passiert mit ausländischen Touristen, die kein deutsche Bankkonto haben (Bezahlung) + Mehrsprachigkeit?

   \item Verfügbare Technologien: Smartphone mit Applikation, im Sonderfall (Registrierung) über Webpräsenz am Computer, sowie Internetanbindung über Anbieter im deutschen Netz. (Was ist mit ausländischen Touristen?).  
   
   \item Physikalisches Umfeld: Suche kann sowohl im Freien als auch in Gebäuden stattfinden. Da Wandernde zB auch fernab von Hauptstraßen und Orten reisen können ist mit Netzproblemen zu rechnen. Zusätzlich sollte Rücksicht auf Akkuleistung genommen werden, da die Stromversorgung selten vorhanden sein wird.

    \item Erfahrung der Benutzer: Mit Erfahrung der Campingdomäne kann gerechnet werden (nicht zwangsläufig) und mit einem Smartphone kann in der Regel umgegangen werden. Anwender wissen, dass sie sich auf sozialen Kontakt einlassen sollten.

    \item Einstellung der Benutzer: Sozial offenere Menschen, eventuell naturgebundene Leute die auf der Durchreise sind bzw. an kürzeren Aufenthalten interesse haben.  


\end{itemize}

 