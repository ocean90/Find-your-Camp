%!TEX root = ../konzept.tex

\section{Nutzungskontext}
Nach der Betrachtung der vorhandenen Benutzer, eine erste Beschäftigung mit ihre Zielen und den Aufgaben des Systems in der vorherigen Anforderderungsanalyse, sowie weiteren Informationen die in der bisherigen Ausarbeitung hin und wieder angesprochen wurden, folgt ein Grobriss des Nutzungskontext\footnote{Nach Inhalt der ISO 9241-210, Prof. Hartmann Draft zur MCI ab S. 544}.\\
Bereits verdeutlichte Bereiche sollen hierbei nicht genauer wiederholt werden, sondern es geht vorerst um die Ergänzung zusätzlicher Überlegungen.\\


Benutzer
\begin{itemize}
   \item 
   Altersklassen: Nicht exakt definierbar, grundsätzlich jeder der ein Smartphone besitzt und auf diese Art und Weise Reisen will. Kernzielgruppe wird zwischen 16 - 50 Jahre geschätzt. Prinzipiell könnten auch Personen darüber hinaus Interesse an der Applikation haben. Vorraussetzung ist lediglich die vorhandene Technologie. Vermutlich spricht der sozialere Aspekt diese aber eher weniger an und sind wahrscheinlich nicht mit Campingausrüstung längere Zeit unterwegs. Physische Merkmale befähigen sie als Mieter dazu, eine Reise aufzunehmen.

   \item 
   Einkommen: Grundsätzlich alle Einkommensklassen möglich, aufgrund des Kostenfaktors jedoch hauptsächlich Einkommensschwache oder eine besonders finanzbewusste Zielgruppe.

   \item 
   Erfahrung der Benutzer: Mit Erfahrung der Campingdomäne kann gerechnet werden, muss aber nicht zwangsläufig.
Erfahrene Leute in diesem Bereich, haben eventuell schon einige Apps ausprobiert und in Benutzung und können mit solchen System sicher umgehen.
   Grundsätzlich wird davon ausgegangen, dass die Stakeholder mit einem Smartphone umgehen können, aber eventuell zum ersten Mal auf diesem Weg Reservierungen und Bezahlungen durchführen. 

   \item
   Fähigkeiten: Sprachkenntnisse können variieren, da auch ausländische Touristen die Software benutzen könnten.

    \item 
    Einstellung der Benutzer: In der Regel sozial offenere Menschen, eventuell naturgebundene Leute die auf der Durchreise sind bzw. an kürzeren Aufenthalten interesse haben. 

   
\end{itemize}

\newpage


Aufgaben und Ziele
\begin{itemize}
   \item 
   Mieter: Finden eines geeigneten Schlafplatzes mit Hilfe des Smartphones als Unterkunft. Dabei die Suchanfrage von Benutzerseite ausgehend und die Anfrageverarbeitung auf Seite des Systems verarbeiten. 
   \item 
   Vermieter: Erfolgreiches Vermieten eines Schlafplatzes. Relevante Anfragen vom System empfangen und nach eigenem Ermessen beantworten. 
   \item
   Sichere Informationsübertagung zwischen Kontaktpersonen
   \item
   Sichere Abwicklung der Bezahlung\\  


\end{itemize}


Arbeitsmittel
\begin{itemize}
   \item 
   Smartphone mit Applikation und Android Betriebssystem. (Im Sonderfall wäre eine Registrierung über Webpräsenz am Computer eine denkbare Option) 
   \item  
   Internetanbindung in Zusammenarbeit mit Telefonanbieter\\
   

\end{itemize}


physikalische Umfeld
\begin{itemize}

   \item 
   Umgebung: In der Regel im Freien, kann aber auch im Gebäuden stattfinden. Nicht vorhersehbar, ob in Großstadt oder ländlicher Gegend.

   \item
   Anbindung:  Da Wandernde auch fernab von Hauptstraßen und Orten reisen können ist mit Netzproblemen zu rechnen. Zusätzlich sollte Rücksicht auf Akkuleistung genommen werden, da die Stromversorgung selten vorhanden sein wird. 

   \item
   Gepäck: Reisende führen relativ viel Gepäck mit sich und benötigen ihre Campingausrüstung, daher in Flexibilität etwas eingeschränkt.
\end{itemize}



