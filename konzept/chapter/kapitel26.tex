%!TEX root = ../konzept.tex

\section{Geschäftsmodell}
TODO Ausformulieren, Information zum Gewerbe anmelden

Mögliche Ansätze
\begin{itemize}
   \item Kostenlose Nutzung, dafür einmalige Zahlung für App. Keine Zutsatsgebühren für Angebote, Verleihprozesse etc.\\
   Vorteil: Nutzer werden nur einmalig mit Appkosten konfrontiert, geringe Folgekosten für Anwender.
   Nachteil: Schwer kritische Menge zu erreichen, direkte Kosten können Einstiegshürde darstellen, keine langfristigen Einnahmen für Serverkosten etc. Finanzierung müste mit anderen Mitteln wie Werbung erreicht werden. Da kostenpflichtige Applikation eher unpassend. In diesem Fall eine werbefinanzierte Variante gratis und kostenpflichtige werbefreie Option.

   \item In-App Käufe, Guthaben aufladen. Die Bezahlung der Mietvorgänge wird komplett über die Applikation geregelt. Aufladen des entsprechenden Guthaben gibt Anteil. Weitere Möglichkeit In-App Käufe mit zusätzlichen Funktionen (ausbaubar)

   \item Werbung + Eventkooperationen.
   Innerhalb der Anwendungen gibt es die Möglichkeit mit Partnern zu kooperieren. Vorstellbar wären zB Eventpartner, die bei bestimmten Veranstaltungungen dafür werben und entsprechende Angebote daauf auslegen. Valider Punkt, da zudem auch Events als Motivationsgrund der Anwender gelten kann. Zusätzlich dazu innerhalb einer (werbepflichten) Applikation Werbung von Partnern der entsprechenden Domäne.

   \item Bezahlung läuft über App ab. Dadurch haben die Benutzer eine Bestätigung und Sicherheit, da sie zum einen bargeldlos bezahlen können und eine Absicherung über den Mietvorgang haben. (Keiner kann sich nach angenommener Leistung aus der Bezahlung abwenden). Durch den Bezahlprozess wird eine Anteilmäßige Provision draufgerechnet. 


\end{itemize}
 