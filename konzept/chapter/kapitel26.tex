%!TEX root = ../konzept.tex

\chapter{Geschäftsmodell}

Um das Projekt langfristig umsetzen zu können, ist es notwendig, die finanziellen Kosten zu decken und Gewinn zu erzielen. 
Daher wurden erste Ansätze eines möglichen Geschäftsmodells überlegt.

\begin{itemize}
   \item \textbf{Kostenlose Nutzung, dafür einmalige Zahlung für App}. Keine Zutsatsgebühren für Inserate, Verleihprozesse etc.\\
   Vorteil: Nutzer werden nur einmalig mit Appkosten konfrontiert, haben danach keine weiteren Folgekosten.

   Nachteil: Schwer kritische Masse zu erreichen, da direkte Kosten Einstiegshürde darstellen können. Dadurch können keine laufende Einnahmen für Serverkosten, Personalkosten oder Ähnliches erzielt werden. Finanzierung müsste daher mit anderen Mitteln, wie Werbung, erreicht werden. In diesem Fall wäre eine kostenpflichtige Applikation eher unpassend. Es ergebe sich die Option einer werbefinanzierten kostenlosen Variante und dem kostenpflichtigen, dafür werbefreien Gegenstück.\\


   \item \textbf{In-App Käufe oder Guthaben aufladen} 
   Die Bezahlung der Mietvorgänge wird komplett über die Applikation geregelt. Aufladen des entsprechenden Guthaben gibt anteilmäßige Einnahmen. Eine weitere Möglichkeit wären In-App Käufe mit zusätzlichen Funktionen. Dabei sollten diese jedoch im Kontext Sinn ergeben und auch ohne diese sollte der Anwender seine Ziele erreichen können. 
   
   Dadurch haben Benutzer eine höhere Sicherheit, da sie zum einen bargeldlos bezahlen können und zum anderen eine Bestätigung über den Mietvorgang haben. (Keiner kann sich nach angenommener Leistung aus der Bezahlung abwenden). Durch den Bezahlprozess wird eine Anteilmäßige Provision an den Mietkosten draufgerechnet. 
\\

   \item \textbf{Werbung + Eventkooperationen}.
   Innerhalb der Anwendungen gibt es die Möglichkeit mit Partnern zu kooperieren. Vorstellbar wären z.B. Eventpartner, die bei bestimmten Veranstaltungungen dafür werben und entsprechende Angebote darauf auslegen. Dies ist ein greifender Punkt, da auch Events als Motivationsgrund der Anwender gelten können und zu solchen vermehrt eingesetzt werden. Zusätzlich dazu könnte innerhalb einer (werbepflichten) Applikation Werbung von Partnern der entsprechenden Domäne geschaltet werden.

\end{itemize}

\newpage
Anhand dieser Ansatzpunkte wurde anschließend ein grobes Geschäftsmodell entwickelt, dass Elemente aller 3 Ansätze kombiniert und für alle Stakeholder eine möglichst positive Erfahrung mit dem Produkt bieten soll.

Um möglichst viele Anwender zu erreichen und auch neue Interessenten zu gewinnen, soll die Applikation kostenlos zur Verfügung gestellt werden. Wie bereits im Vorfeld beschrieben, könnte ein Einstiegspreis für viele Leute bereits eine Hürde darstellen. Zudem muss man bedenken, dass sich der Nutzen für den Anwender nur durch die Benutzung ergibt. 