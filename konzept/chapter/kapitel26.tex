%!TEX root = ../konzept.tex

\chapter{Geschäftsmodell}

Um das Projekt langfristig umsetzen zu können, ist es notwendig, die finanziellen Kosten zu decken und Gewinn zu erzielen. 
Daher wurden erste Ansätze eines möglichen Geschäftsmodells überlegt.

\begin{itemize}
   \item \textbf{Kostenlose Nutzung, dafür einmalige Zahlung für App}. Keine Zutsatsgebühren für Inserate, Verleihprozesse etc.\\
   Vorteil: Nutzer werden nur einmalig mit Appkosten konfrontiert, haben danach keine weiteren Folgekosten.

   Nachteil: Schwer kritische Masse zu erreichen, da direkte Kosten Einstiegshürde darstellen können. Dadurch können keine laufende Einnahmen für Serverkosten, Personalkosten oder Ähnliches erzielt werden. Finanzierung müsste daher mit anderen Mitteln, wie Werbung, erreicht werden. In diesem Fall wäre eine kostenpflichtige Applikation eher unpassend. Es ergebe sich die Option einer werbefinanzierten kostenlosen Variante und dem kostenpflichtigen, dafür werbefreien Gegenstück.\\


   \item \textbf{In-App Käufe oder Guthaben aufladen} 
   Die Bezahlung der Mietvorgänge wird komplett über die Applikation geregelt. Aufladen des entsprechenden Guthaben gibt anteilmäßige Einnahmen. Eine weitere Möglichkeit wären In-App Käufe mit zusätzlichen Funktionen. Dabei sollten diese jedoch im Kontext Sinn ergeben und auch ohne diese sollte der Anwender seine Ziele erreichen können. 
   
   Dadurch haben Benutzer eine höhere Sicherheit, da sie zum einen bargeldlos bezahlen können und zum anderen eine Bestätigung über den Mietvorgang haben. (Keiner kann sich nach angenommener Leistung aus der Bezahlung abwenden). Durch den Bezahlprozess wird eine Anteilmäßige Provision an den Mietkosten draufgerechnet. 
\\

   \item \textbf{Werbung + Eventkooperationen}.
   Innerhalb der Anwendungen gibt es die Möglichkeit mit Partnern zu kooperieren. Vorstellbar wären z.B. Eventpartner, die bei bestimmten Veranstaltungungen dafür werben und entsprechende Angebote darauf auslegen. Dies ist ein greifender Punkt, da auch Events als Motivationsgrund der Anwender gelten können und zu solchen vermehrt eingesetzt werden. Zusätzlich dazu könnte innerhalb einer (werbepflichten) Applikation Werbung von Partnern der entsprechenden Domäne geschaltet werden.

\end{itemize}

\newpage
Anhand dieser Ansatzpunkte wurde anschließend ein grobes Geschäftsmodell entwickelt, dass Elemente aller 3 Ansätze kombiniert und für alle Stakeholder eine möglichst positive Erfahrung mit dem Produkt bieten soll.\\

Um möglichst viele Anwender zu erreichen und auch neue Interessenten zu gewinnen, soll die Applikation kostenlos zur Verfügung gestellt werden. Wie bereits im Vorfeld beschrieben, könnte ein Einstiegspreis für viele Leute bereits eine Hürde darstellen. Zudem muss man bedenken, dass sich der Nutzen für den Anwender nur durch Verwendung auf einer Reise oder bei einer Vermietung ergibt. Eine Bezahlung bietet dem Kunden daher noch keinen direkten Mehrwert und muss vermieden werden. 
Innerhalb der Applikation wird mit Werbeeinblendungen gearbeitet werden. Die Einbindung muss dabei möglichst störfrei sein und den Benutzer bei der Interaktion mit der Anwendung nicht aufhalten. Damit die Werbung kein behindernder Faktor ist, sollen verschiedene Einbindungsmethoden konzipiert und bei der Entwicklung von Gestaltungslösungen abgewogen werden. Die Inhalte der Werbung müssen thematisch zur Domäne passen und dem Benutzer dadurch relevante Angebote präsentieren. Sinnvoller Weise sollte die Kooperation mit einem Werbepartner eine gewisse Mindestlaufzeit aufweisen. Denkbar wäre zum Beispiel eine Laufzeit von 1-2 Monaten, da die Benutzeraktivität nicht unbedingt regelmäßig ist. Die Kosten belaufen sich dabei auf die Anzahl der Anwender. Bei einem Preis von 0,1 Cent pro Anwender und schätzungweise 2000 aktiven Beteiligten,ergeben sich daraus, bei einer Laufzeit von 1 Monate,  Einnahmen von 100 Euro. Je nach Benutzeraktivität kann dieser Wert variieren. 
Zusätzlich dazu, können Einnahmen durch getätigte Weiterleitung über die Werbung einkalkuliert werden und ein prozentmäßiger Anteil von 1-2\% an Verkäufen, die über die Weiterleitung initiiert werden.

Weiterhin soll eine Kooperation mit Eventveranstaltern stattfinden, da diese eine bedeutende Nutzungsmotivation darstellen können. Neben Werbeeinblendungen kann man hierbei z.B. relevante Anbieter besonders hervorheben und bei erfolgreicher Vermietung, eine Gebühr des Eventveranstalters von ca. 5-10\% des Mietpreises einfordern. Findet beispielsweise im Raum Köln eine OpenAir Veranstaltung statt und vermieten in dieser Nacht 10 Vermieter erfolgreich ihr Grundstück für 10 Euro die Nacht, so werden Einnahmen von 5-10 Euro (pro Nacht) getätigt. 

Das finanzielle Mindestziel des Projektes sollte die Rückgewinnung der Investition und die Deckung der laufenden Kosten sein. Um dieses zu erreichen, ist die Haupteinnahme die prozentuale Beteiligung an Vermietungen. Die Bezahlung wird dementsprechend über die Anwendung getätigt und vom Mietpreis geht ein Anteil von 5\% als Gewinn hervor. Da das Projekt nicht alleinig durch Werbekosten finanziert werden kann, ist dieser Schritt notwendig um auf Appkosten und Inseratskosten zu verzichten. Um auf Dauer möglichst viele Vermieter zu gewinnen, soll das Anbieten an sich kein Geld kosten. 
