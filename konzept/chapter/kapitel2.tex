%!TEX root = ../konzept.tex

\chapter{Projektbeschreibung}

Zu Beginn des Projektes stand die Problemfindung und die erste Auseinandersetzung mit wesentlichen Aspekten des zu entwerfenden Systems im Vordergrund. Das Sammeln von diversen Ideen einer potentiellen verteilten Anwendung, führte anschließend zum ersten Expose auf Grundlage des Shared Economy Prinzips\footnote{Genaue Erklärung folgt im Punkt 2.1}.\\
Inhalt dieser Idee war die Entwicklung eines Verleihsystems für Alltagsgegenstände in der Nachbarschaft. Ermittelte Marktkonkurrenz verfolgte dabei ähnliche Ansätze, erhoben für den Leihprozess jedoch teilweise hohe Kosten und ermöglichten dabei auch den deutschlandweiten Austausch, wodurch es im Versand zu zeitlichen Verzögerungen kommt.
Ansatzpunkt hierbei war zum einen die Kostensenkung und zum anderen die Spezialisierung auf das direkte Umfeld. Besonders der zeitliche Faktor wurde als Problem identifiziert und sollte im neuen System verbessert werden. Schwierigkeiten erwiesen sich jedoch bei der Nutzungsmotivation, sodass nach einer erneuten Auseiandersetzung der Beschluss gefasst wurde, die Thematik auf einen bestimmten Kontext anzuwenden.\\   
In einer weiterführenden Beschäftigung wurde die neue Idee ausgebaut und die “Find your Camp”- Applikation als Thematik des Entwicklungsprojektes beschlossen. 

% Problemstellung
%!TEX root = ../konzept.tex


\section{Problemstellung}
TODO

\subsection{Relevanz}

\newpage

% Marktanalyse
%!TEX root = ../konzept.tex

\section{Marktanalyse}
    Die Idee private Grundstücke als Campingplatz zu verleihen ist nicht neu. Schaut man sich den vorhandenen Markt an, findet man bereits Unternehmen die sich mit genau diesem Thema beschäftigen. Die Anzahl an vorhandenen Anbietern ist dabei aber noch verhältnismäßig gering und das Prinzip eine Entwicklung aus kürzerer Zeit. Das der Verleih des eigenen Platzes als Unterkunft für Reisende, eine akzeptierte Alternative gegenüber herkömmlichen Unternachtsmöglichkeiten darstellt, wurde bereits im Vorfeld herausgestellt.\\
    Speziellen Fokus auf Campingmöglichkeiten legt dabei das im Jahr 2009 gegründete Unternehmen Freagle\footnote{www.fragle.org}. Regional gibt Freagle keine Einschränken vor und ermöglicht es Benutzers weltweit teilzunehmen. Mietung und Vermietung ist Benutzer prinzipiell kostenfrei. Bei der Anmeldung ist es vorgesehen einen eigenen Platz anzubieten. Sollte man dazu keine Gelegenkeit haben, besteht unter anderem die Option für eine Jahresgebühr von 12,50 Euro beizutreten. Vorhandene Angebote werden auf einer Weltkarte angezeigt und können ausgesucht und angefragt werden. Zur Sicherheit wird eine Verifizierung des Ortes vorgesehen und Mitglieder benötigen eine Freagle Card, die als ID der Leute dient.

    Freagle ist rein webbasiert und bietet für mobile Anwender keine  Applikation oder angepasste Website. Der Login über ein Smartphone ist nicht möglich\footnote{Option verfügbar, aber nicht funktionabel}.
    Das Verhältnis von angebotene Plätzen zu Mitglieder beträgt etwa 1:5 und zeigt, dass grundsätzlich nicht alle Leute einen Platz anbieten können oder wollen.

    Speziell an Freagle ergeben sich damit mehrere Ansatzpunkte, die verbessert werden können. Da gerade Reisende auf Flexibilität angewiesen sind und oftmals keinen Rechner mit sich führen, dient das Smartphone für viele als wesentliches Hilfsmittel. Gerade in diesem Kontext ist eine dafür angesetzte Weboberfläche oder Applikation von großem Nutzen. Zudem wird eine frühzeitige Planung und Kontaktaufnahme vorrausgesetzt und die Suche wird komplett in die Hände des Benutzers gelegt. In diesem Fall könnte man sich den Möglichkeiten eines Smartphones bedienen und unrelevante Ergebnisse direkt rausfiltern. 

    Um das Interesse der Vermieter zu erhöhen, besteht die Option neben sozialen Kontakten einen weiteren Anreiz zu bieten. Geplant ist hierbei ein kostenbasiertes Vermietsystem, dass als Einnahmequelle dienen kann. Dabei stellt sich grundsätzlich die Frage, ob Benutzer dazu bereit wären für einen solchen Dienst zu zahlen, aber auch am Beispiel Freagle zeigt sich, dass die Interessenten eine jährliche Gebühr zahlen, um solch ein Angebot wahrzunehmen.\\
    
    Ein weiteres Beispiel, dass aber auch lediglich über eine Webpräsenz verfügt, ist \\  Campinmygarden\footnote{http://campinmygarden.com}. Hierbei liegt der lokale Schwerpunkt und Marktanteil deutliche auf Großbritanien. Weitere Länder werden prinzipiell unterstützt, aber die Teilnehmerzahl ist im Vergleich zum ersten Beispiel deutlich geringer.
    Eine Besonderheit die Campinmygarden auszeichnet, ist die Einbeziehung anstehender Events indem aufgezeigt wird, wann welche in der Nähe dieses Ortes stattfinden. Ansonsten werden ähnliche Funktionalitäten angeboten zu Freagle, wobei in Vermieter in diesem Fall eine Vermietgebühr angeben können und somit finanziellen Gewinn machen. Ein großer Kritikpunkt an Campinmygarden ist die Transparenz der Userdaten. Bereits als unregistrierter User hat man die Möglichkeit alle Angebote einzusehen und anhand einer Karte sogar den angegeben Standort mit Ausstattung angezeigt zu bekommen.\\
    Als Verbesserungsansatz aus diesem Beispiel lässt sich speziell die Sicherheit und Transparenz der sensiblen Benutzerdaten mitnehmen. 
    In einem System, indem es darum geht das Vertrauen der Benutzer zu gewinnen und sie dazu zu ermutigen fremde Menschen in ihren Privatraum eindringen zu lassen, muss den Benutzern selbst die Kontrolle über ihre Information gegeben werden.\\

    Existierende Applikationen, wie der ADAC Camping- und Schnellplatzführer 2013, unterstützen das Suchen und Finden von Zeltplätzen, beziehen dabei aber lediglich die öffentlichen Anbieter ein. Aufgrund des Preisfaktors, sowie den üblichen Reservierungs- oder Buchungsvoraussetzungen kommen sie deshalb (in der Regel\footnote{Für den Fall das in der Gegend kein Grundstücksanbieter angemeldet ist, muss der User sich zwangsläufig an solche Angebote wenden. }) nicht in Frage und unterstützen lediglich die Mieter, aber nicht die privaten Vermieter.\\

    Für die ausgewählte Thematik, schließen diese Beispiele einen großen Teil des bisher vorhandenen Anbieter an (Webpräsenz) oder Repräsentieren die Kernfunktionalität vieler ähnlicher Anwendungen. 
    Mit der Analyse des Marktes, folgt die Chancenermittlung und das Herausstellen der Alleinstellungsmerkmale.



\newpage

% Alleinstellungsmerkmal
%!TEX root = ../konzept.tex


\section{Alleinstellungsmerkmale und Chancen}
Mit speziellem Fokus auf das Grundstück Sharing, ergeben sich aus Marktanalyse und anfänglicher Anbieteruntersuchung (Couchsurfing) verschiedene Alleinstellungsmerkmale des Find your Camp Systems.\\

Das Beispiel Couchsurfing setzt vorallem auf den sozialen Aspekt und ist darauf ausgelegt neue Bekanntschaften zu schließen. Reisende sind meistens mit wenigen Personen unterwegs, übernachten einige Tage bei ihrem Host und lassen sich von ihm die Stadt und Kultur zeigen.
Geeignet ist dieser Ansatz weniger bei größeren Reisegruppen oder Familien. Zusätzlich liegt vorallem auf der Vermieterseite kein finanzieller Gewinn und das Eindringen in seinen privaten Lebensraum kann viele potentielle Nutzer abschrecken.
Airbnb ermöglicht die private Vermietung, der Kostenpunkt ist jedoch weiterhin hoch und in beliebten Gegenden ist weiterhin eine Reservierung von nöten. 

Grundsätzlich lassen sich in den Beispielen positive Ansätze finden, die beibehalten und ausgebessert werden können. Vorallem aber die Negativpunkte sollen ausgebessert werden.\\ Aus allen Betrachtungen ergeben sich damit folgende Ansatzpunkte für potentielle Optimierungen:

\subsection{Mieter und Vermieter}
\begin{itemize}
   \item
   \textbf{Einheitliches Kommunikationssystem}: Die Kommunikation muss nicht über eine Webpräsenz oder diverse unterschiedliche Wege stattfinden. (Nachrichtensystem der Webpräsenz, Email, Telefon), sondern wird von allen Anwendern über die gleiche Software geschehen. Auch die Bezahlung kann auf diesem Weg abgeschlossen werden.

   \item 
   \textbf{Zeitoptimierung}: Gängige Beispiele setzen auf Angebot und Nachfrage Anzeigen, diese können veralten oder nicht aktuell sein und zu spät gelesen werden. Der passende Vermieter erhält im neuen System direkte Anfragen die zeitliche und inhaltliche Relevanz haben und kann diese direkt beantworten. Der Mieter soll dadurch in kürzerer Zeit eine Antwort erhalten.

   \item
   \textbf{Filtern relevanter Anfragen (1)}: Anhand der Benutzerdaten, findet eine Kontaktaufnahme nur zwischen kompatiblen Benutzern statt. Dadurch verringert sich die Anzahl zielloser Anfragen und Kommunikationen.

   \item 
   \textbf{Zeitliche Unabhängigkeit}: Während der Vermietung besteht die Möglichkeit, dass alle Beteiligten ihre Aktivitäten unabhängig voneinander ausführen können. Der Mieter ist (im Vergleich zu Couchsurfing) nicht unbedingt auf den Zugang zur Wohnung durch den Vermieter angewiesen. 

\end{itemize}


\subsection{Vermieter}
\begin{itemize}
   \item 
   \textbf{Finanzieller Anreiz}: Vermieter haben die Möglichkeit für ihre Vermietung Kosten zu erheben und finanziellen Gewinn zu schlagen.
   Dabei muss es sich nicht nur um die Verleihung der Wohnung handeln, sondern grundsätzlich vorhandene Grundstücke wie Gärten, Hof, Landstücke.

   \item 
   \textbf{Kontrolle der Privatsphäre}: Im Gegensatz zum Couchsurfing lässt sich der Bereich eingrenzen, indem Reisende in den eigenen Lebensraum eindringen können. Auch ein permanenter sozialer Kontakt ist nicht von nöten, sodass der ganze Prozess auf einer rein geschäftlichen Ebene ausgetragen werden kann.

   \item 
   \textbf{Sicherheit der Daten (2)}: Sensiblen Informationen können nur bei Bedarf freigegeben werden und sind nur lokal gespeichert. Damit erhalten nur Kunden auch die benötigten Informationen und das freie Einsehen über eine Webpräsenz ist nicht möglich.

\end{itemize}
   

\subsection{Mieter}
\begin{itemize}
   \item 
   \textbf{Mobilität (3)}: Eine mobile Anwendung unterstützt die verbreiteste Technologie, die Reisende in der Regel mit sich führen. Zusätzlich dazu können die unique Features eines Smartphones anwendung finden.
   Er ist flexibler und muss sich nicht um Zugung zu stationären Rechnern kümmern.

   \item 
   \textbf{Vergrößerte Reise- und Interessentengruppe}: Es ist möglich mit einer größeren Anzahl an Personen zu verreisen, die innerhalb einer Wohnung nicht untergebracht werden können. Dazu bestünde auch die Option Familien mit Kindern unterzubringen (falls man diese als Host nicht aufnehmen würde). 

   \item
   \textbf{Soziale Einstellung berücksichtigen}: Nicht jeder möchte viel Kontakt mit seinen Host haben und es besteht die Möglichkeit, dass sich Leute auf den geschäftlichen Prozess beschränken wollen. Eventuell besteht kein Interesse an den sozialen Aspekten des Sharings.

   \item 
   \textbf{Spontanität}: Reisende können auf ihrer Reise spontanere Suchanfragen starten und müssen sich nicht zwangsläufig an vorgegebene Routen halten. Theoretisch ist dies auch unter bereits vorhanden umständen möglich (Reisender entscheidet in diesem Ort selbstständig eine Unterkunft zu suchen), aber die Applikation unterstützt hierbei speziell die Suche. Im Gegenzug dazu, verliert der Vermieter jedoch organisatorische Sicherheit und es kann die Gefahr auftreten, dass Anfragen nicht angenommen werden können, da sie zu kurzfristig erscheinen.

\end{itemize}

Viele dieser Punkte, ergeben sich dabei auch als Folge des Handlungskontextes. Gerade auf Mieterseite werden einige Chancen ermöglicht, jedoch lässt sich nicht mi Sicherheit auf diese setzen und haben für die Entwicklung des Projektes keine größere Priorität. 

Speziell die mit 1 - 3 markierten Aspekte, sollen in der weiteren Betrachtung fokusiert werden, da sich diese als Alleinstellungsmerkmale auszeichnen und für das System als besonders relevant erachtet werden.
 


%\newpage

% Risiken
%!TEX root = ../konzept.tex


\section{Alleinstellungsmerkmale und Chancen}
Motivation/Alleinstellungsmerkmal
      Vermieter
      Geld
      Soziale Kontakte
      Sicherheit durch Zertifizierung
      Mieter
      Kostengünstiger ggü. Hotel bzw. Campigplätzen/Zeltlager
      Soziale Kontakte
      Spontanität wird gegeben
      Zertifizierte Vermieter
      Mobilität des Nutzers
      Beide
      Teil einer Community
      Kontaktaufnahme
       
 
Aspekte die das neue System zu vorhandenen Optionen abheben, wäre zum einen der verstärkte Bezug auf private Grundstücke, die auf dem deutschen Markt bisher stark vernachlässigt werden.\\
Die Gestaltung als Applikation unterstützt die natürliche Flexibilität eines Reisenden und ermöglicht ihm neben dem Finden eines Angebotes auch die Abwicklung der Kommunikation sowie gegebenfalls das Orientieren durch die GPS Funktion. Der Einsatz eines QR Code oder NFC Systems zur Identifikation kann zudem die Sicherheit deutlich steigern.

\newpage


% Geschäftsmodell
%!TEX root = ../konzept.tex

\chapter{Geschäftsmodell}

Um das Projekt langfristig umsetzen zu können, ist es notwendig, die finanziellen Kosten zu decken und Gewinn zu erzielen. 
Daher wurden erste Ansätze eines möglichen Geschäftsmodells überlegt.

\begin{itemize}
   \item \textbf{Kostenlose Nutzung, dafür einmalige Zahlung für App}. Keine Zutsatsgebühren für Inserate, Verleihprozesse etc.\\
   Vorteil: Nutzer werden nur einmalig mit Appkosten konfrontiert, haben danach keine weiteren Folgekosten.

   Nachteil: Schwer kritische Masse zu erreichen, da direkte Kosten Einstiegshürde darstellen können. Dadurch können keine laufende Einnahmen für Serverkosten, Personalkosten oder Ähnliches erzielt werden. Finanzierung müsste daher mit anderen Mitteln, wie Werbung, erreicht werden. In diesem Fall wäre eine kostenpflichtige Applikation eher unpassend. Es ergebe sich die Option einer werbefinanzierten kostenlosen Variante und dem kostenpflichtigen, dafür werbefreien Gegenstück.\\


   \item \textbf{In-App Käufe oder Guthaben aufladen} 
   Die Bezahlung der Mietvorgänge wird komplett über die Applikation geregelt. Aufladen des entsprechenden Guthaben gibt anteilmäßige Einnahmen. Eine weitere Möglichkeit wären In-App Käufe mit zusätzlichen Funktionen. Dabei sollten diese jedoch im Kontext Sinn ergeben und auch ohne diese sollte der Anwender seine Ziele erreichen können. 
   
   Dadurch haben Benutzer eine höhere Sicherheit, da sie zum einen bargeldlos bezahlen können und zum anderen eine Bestätigung über den Mietvorgang haben. (Keiner kann sich nach angenommener Leistung aus der Bezahlung abwenden). Durch den Bezahlprozess wird eine Anteilmäßige Provision an den Mietkosten draufgerechnet. 
\\

   \item \textbf{Werbung + Eventkooperationen}.
   Innerhalb der Anwendungen gibt es die Möglichkeit mit Partnern zu kooperieren. Vorstellbar wären z.B. Eventpartner, die bei bestimmten Veranstaltungungen dafür werben und entsprechende Angebote darauf auslegen. Dies ist ein greifender Punkt, da auch Events als Motivationsgrund der Anwender gelten können und zu solchen vermehrt eingesetzt werden. Zusätzlich dazu könnte innerhalb einer (werbepflichten) Applikation Werbung von Partnern der entsprechenden Domäne geschaltet werden.

\end{itemize}

\newpage
Anhand dieser Ansatzpunkte wurde anschließend ein grobes Geschäftsmodell entwickelt, dass Elemente aller 3 Ansätze kombiniert und für alle Stakeholder eine möglichst positive Erfahrung mit dem Produkt bieten soll.

Um möglichst viele Anwender zu erreichen und auch neue Interessenten zu gewinnen, soll die Applikation kostenlos zur Verfügung gestellt werden. Wie bereits im Vorfeld beschrieben, könnte ein Einstiegspreis für viele Leute bereits eine Hürde darstellen. Zudem muss man bedenken, dass sich der Nutzen für den Anwender nur durch die Benutzung ergibt. 

\newpage


% Zielsetzung
%!TEX root = ../konzept.tex

\section{Zielhierachie}

TODO mehr Bezug zum Projekt
Das Projekt unterliegt einer gewissen Zielhierachie, die im Rahmen der Zielsetzung aufgestellt wurde. Dabei wurden zum einen high-level Ziele betrachtet, die auch über den Projektrahmen hinausgehen und eine mögliche Entwicklung im Kontext beschreiben, sowie konkretere Ziele, die sich direkt auf das Projekt anwenden lassen.

\subsubsection{Strategische Ziele}
Konkretisiert man den Betrachtungsraum auf den Projektrahmen, so ist das strategische Ziel die prototypische Entwicklung eines Verleihsystems für private Grundstücke als Campingplatz für Reisende. Schwerpunkt des Projektes liegt dabei auf der Konzipierung und Dokumentierung des Entwicklungsprozess und die Realisierung der Minimalziele.

\subsubsection{Taktisches Ziele}
Für das Projekt speziell bedeutet dies, das Aufstellen eines Interaktionsplans zwischen Vermieter und Mieter während eines Mietprozesses und das Herausstellen der einzelnen Aktivitätsschritte, die zur Aufgabenbewältigung notwendig sind.
Daraus einhergehend die funktionale Unterstützung dieser. Grob umfasst dies, die Angebotssuche und Mietanfrage des Mieters (Reisenden), den Kommunikationsaufbau zwischen Mieter und Vermieter und den Datenaustausch der Beteiligten, gefolgt vom Kommunikationsabschluss. (Zum Beispiel durch Bewertung und Bezahlung)

\subsubsection{Operative Ziele}
Operativ bedeutet dies, dass die einzelnen Aktivitätsschritte genau untersucht werden und darauf hin eine mögliche Realisierung geplant wird. Die Gestaltung einer möglichen Lösung der taktischen Schritte anhand verschiedener Technologien, die in diesem Zusammenhang ermittelt, abgewogen und umgesetzt werden sollten.
Genau bedeutet dies, mit Hilfe der Verteiltheit des Systems, einzelne Kommunikationsbereiche genauer zu betrachten.

\subsubsection{Minimal Ziele}
Schwerpunkt des Projektes liegt auf der Kommunikation der Mieter und Vermieter. Das Minimal Ziel ist das Finden eines eingetragenen Vermieters, das Stellen einer Anfrage und die daraus resultierende Kontaktaufnahme. Zusätzlich der damit verbundene Austausch der Benutzerdaten. Die Anfrage soll anhand von Matchingkriterien nur an relevante Anbieter weitergeleitet werden.

\subsubsection{Optionale Ziele}
Optionales Projektziel ist die Vertiefung des Matchingalgorithmen. Die grundlegende Funktionalität sollte das Minimalziel sein, die Feinheit und Komplexität der Kriterien, kann jedoch (voraussichtlich) nur in einem einfacheren Rahmen betrachtet werden.






