%!TEX root = ../konzept.tex

\chapter{Projektbeschreibung}

Zu Beginn des Projektes stand die Problemfindung und die erste Auseinandersetzung mit wesentlichen Aspekten des zu entwerfenden Systems im Vordergrund. Das Sammeln von diversen Ideen einer potentiellen verteilten Anwendung, führte anschließend zum ersten Expose auf Grundlage des Shared Economy Prinzips\footnote{Genaue Erklärung folgt im Punkt 2.1}.\\
Inhalt dieser Idee war die Entwicklung eines Verleihsystems für Alltagsgegenstände in der Nachbarschaft. Ermittelte Marktkonkurrenz verfolgte dabei ähnliche Ansätze, erhoben für den Leihprozess jedoch teilweise hohe Kosten und ermöglichten dabei auch den deutschlandweiten Austausch, wodurch es im Versand zu zeitlichen Verzögerungen kommt.
Ansatzpunkte hierbei war zum einen die Kostensenkung und zum anderen die Spezialisierung auf das direkte Umfeld. Besonders der zeitliche Faktor wurde als Problem identifiziert und sollte in geplantem Ansatz verbessert werden. Schwierigkeiten erwiesen sich jedoch bei der Nutzungsmotivation, sodass nach einer erneuten Auseiandersetzung der Beschluss gefasst wurde, die Thematik auf einen bestimmten Kontext anzuwenden.\\   
In einer weiterführenden Beschäftigung wurde die neue Idee ausgebaut und die “Find your Camp”- Applikation als Thematik des Entwicklungsprojektes beschlossen. 

% Problemstellung
%!TEX root = ../konzept.tex


\section{Problemstellung}
TODO

\subsection{Relevanz}

\newpage

% Zielsetzung
%!TEX root = ../konzept.tex

\section{Zielhierachie}

\subsubsection{Strategische Ziele}
Langfristig gesehen ist das oberste Ziel das System, eine Etablierung des Shared Economy Prinzips auf die Domäne des Grundstückverleihs und der Aufbau eines Netzwerks aus zahlreichen Interessenten.\\
Mit der fortschreitenden Entwicklung diverser Sharing Angebote wäre zudem der Ausbau einer globalen Alternative solcher Anbieter möglich. Wie derzeit gängige Unternehmen Hotels, Campingplätze und Hostels gewerblich anbieten und über Suchmaschinen zu finden sind, könnte solch ein Netzwerk eine weitere Zielgruppe bedienen. Priorität hätte dabei der Aufbau sozialer Kontakte und das Nutzen neuer Reisemöglichkeiten.


\subsubsection{Taktisches Ziele}
Mittelfristig ist es das Ziel einen aktiven Anwenderkreis aufzubauen und diesen durch erfolgreiche, sichere und zufriedenstellende Erfahrungen beim Reisen mit dem System langfristig zu gewinnen.
Finanziell würde eine Entwicklung hinsichtliche Kooperationen mit Werbepartnern und Eventveranstaltern angestrebt werden, um das System auf Dauer gewinnbringend zu betreiben und weiter zu entwickeln. \\
Funktional ließen sich mit dem technischen Fortschritt neue Ansätze einbauen (z.B. bei der Bezahlung, Navigation, Augmented Reality) und die Vorteile der Smartphones weiter ausnutzen. \\
In kürzerer (mittelfirstiger) Sicht das Unterstützen mehrer Betriebssysteme und Hersteller oder der Ausbau des Angebots auf Webpräsenz und weitere Geräte.


\subsubsection{Operative Ziele}
Ziel der kurzfristigen Entwicklungsphase ist die Umsetzung eines Vermietsystems für private Grundstücke als Aufenthaltsplatz für Reisende, auf Grundlage des Share Economy Konzeptes.\\
Das verteilte System soll vorhandene Angebote aufzeigen, die Kommunikation zwischen Mieter und Vermieter ermöglichen und den Mietprozess möglichst sicher sowie erfolgreich organisieren und abschließen können.\\
Die Motivation der Mieter liegt darin, kostengünstige Alternativen gegenüber herkömmlichen Aufenthaltsmöglichkeiten zu finden, dabei möglichst mobil zu sein, sowie kurzfristige Suchen zu ermöglichen. Der Vermieter kann zum einen sozialen Nutzen daraus ziehen, dabei aber auch einen finanziellen Gewinn erzielen. 


\subsubsection{Optionale Ziele}


\subsubsection{Minimal Ziele}



\newpage

% Marktanalyse
%!TEX root = ../konzept.tex

\section{Marktanalyse}
    Die Idee private Grundstücke als Campingplatz zu verleihen ist nicht neu. Schaut man sich den vorhandenen Markt an, findet man bereits Unternehmen die sich mit genau diesem Thema beschäftigen. Die Anzahl an vorhandenen Anbietern ist dabei aber noch verhältnismäßig gering und das Prinzip eine Entwicklung aus kürzerer Zeit. Das der Verleih des eigenen Platzes als Unterkunft für Reisende, eine akzeptierte Alternative gegenüber herkömmlichen Unterkunftsanbietern darstellt, wurde bereits im Vorfeld herausgestellt.\\
    Speziellen Fokus auf Campingmöglichkeiten legt dabei das im Jahr 2009 gegründete Unternehmen \textit{Freagle}\footnote{www.fragle.org}. Regional gibt Freagle keine Einschränken vor und ermöglicht es Benutzers weltweit teilzunehmen. Mietung und Vermietung ist grundsätzlich kostenfrei. Bei der Anmeldung ist es vorgesehen einen eigenen Platz anzubieten. Sollte man dazu keine Gelegenheit haben, besteht unter anderem die Option für eine Jahresgebühr von 12,50 Euro beizutreten. Vorhandene Angebote werden auf einer Weltkarte angezeigt und können ausgesucht und angefragt werden. Zur Sicherheit wird eine Verifizierung des Ortes vorgesehen und Mitglieder benötigen eine Freagle Card, die als ID der Reisende dient.

    Freagle ist rein webbasiert und bietet für mobile Anwender keine  Applikation oder angepasste Website. Der Login über ein Smartphone ist nicht möglich\footnote{Option verfügbar, aber nicht funktionabel. Getestet am 24.-27.10.13 mit Iphone}.
    Das Verhältnis von angebotene Plätzen zu Mitglieder beträgt etwa 1:5 und zeigt, dass grundsätzlich nicht alle Leute einen Platz anbieten können oder wollen.

    Speziell an Freagle ergeben sich damit mehrere Ansatzpunkte, die verbessert werden können. Da gerade Reisende auf Flexibilität angewiesen sind und oftmals keinen Rechner mit sich führen, dient das Smartphone für viele als wesentliches Hilfsmittel. Gerade in diesem Kontext ist eine dafür angesetzte Weboberfläche oder Applikation von großem Nutzen. Zudem wird eine frühzeitige Planung und Kontaktaufnahme vorrausgesetzt und die Suche wird komplett in die Hände des Benutzers gelegt. In diesem Fall könnte man sich den Möglichkeiten eines Smartphones bedienen und unrelevante Suchergebnisse direkt rausfiltern. Die Anwendung wird daher als mobile Applikation realisiert.

    Um das Interesse der Vermieter zu erhöhen, besteht die Option neben sozialen Kontakten einen weiteren Anreiz zu bieten. Geplant ist hierbei ein Vermietsystem, dass als Einnahmequelle dienen kann. Dabei stellt sich grundsätzlich die Frage, ob Benutzer dazu bereit wären für einen solchen Dienst zu zahlen, aber auch am Beispiel Freagle zeigt sich, dass die Interessenten eine jährliche Gebühr zahlen, um solch ein Angebot wahrzunehmen.\\
    
    Ein weiteres Angebot, dass aber auch lediglich über eine Webpräsenz verfügt, ist \textit{Campinmygarden}\footnote{http://campinmygarden.com}. Hierbei liegt der lokale Schwerpunkt und Marktanteil deutliche auf Großbritannien. Weitere Länder werden prinzipiell unterstützt, aber die Teilnehmerzahl ist im Vergleich zum ersten Beispiel deutlich geringer.
    Eine Besonderheit die Campinmygarden auszeichnet, ist die Einbeziehung anstehender Events indem aufgezeigt wird, wann welche Veranstaltungen in der Nähe dieses Ortes stattfinden. Ansonsten werden ähnliche Funktionalitäten angeboten, wobei Vermieter in diesem Fall eine Gebühr angeben können und somit finanziellen Gewinn machen. Ein großer Kritikpunkt an Campinmygarden ist die Transparenz der Userdaten. Bereits als unregistrierter User hat man die Möglichkeit alle Angebote einzusehen und anhand einer Karte sogar den angegeben Standort mit Ausstattung angezeigt zu bekommen.\\
    Als Verbesserungsansatz aus diesem Beispiel lässt sich speziell die Sicherheit und Transparenz der sensiblen Benutzerdaten mitnehmen. 
    In einem System, indem es darum geht das Vertrauen der Benutzer zu gewinnen und sie zu ermutigen fremde Menschen in ihren Privatraum eindringen zu lassen, muss den Benutzern selbst die Kontrolle über ihre Information gegeben werden. Eine Verbesserung dahingehend soll durch die Verteiltheit des Systems erreicht werden, unter den Aspekten der Datenspeicherung. \\

    Existierende Applikationen, wie der \textit{ADAC Camping- und Schnellplatzführer 2013}, unterstützen das Suchen und Finden von Zeltplätzen, beziehen dabei aber lediglich die öffentlichen Anbieter ein. Aufgrund des Preisfaktors, sowie den Reservierungs- oder Buchungsvoraussetzungen kommen sie deshalb (in der Regel\footnote{Für den Fall das in der Gegend kein Grundstücksanbieter angemeldet ist, muss der User sich zwangsläufig an solche Angebote wenden. }) nicht in Frage und unterstützen lediglich die Mieter, aber nicht die privaten Vermieter.\\

    Für die ausgewählte Problemdomäne sind die beiden Webpräsenzen die bisher einzigen Anbieter, die in der Recherche ausgemacht werden konnten. Neben der beschriebenen Applikation, gibt es noch zahlreiche Andere. Das vorgestellte Beispiel repräsentiert jedoch die Kernfunktionalität vieler ähnlicher Anwendungen und passt aufgrund der genannten Kriterien nicht auf die gewünschte Zielegruppe und Problemfeld. 
    Mit der Analyse des Marktes, folgte die Chancenermittlung und das Herausstellen der Alleinstellungsmerkmale.



\newpage

% Alleinstellungsmerkmal
%!TEX root = ../konzept.tex


\section{Alleinstellungsmerkmale und Chancen}
Mit speziellem Fokus auf das Grundstück Sharing, ergeben sich aus Marktanalyse und anfänglicher Anbieteruntersuchung (Couchsurfing) verschiedene Alleinstellungsmerkmale des Find your Camp Systems.\\

Das Beispiel Couchsurfing setzt vorallem auf den sozialen Aspekt und ist darauf ausgelegt neue Bekanntschaften zu schließen. Reisende sind meistens mit wenigen Personen unterwegs, übernachten einige Tage bei ihrem Host und lassen sich von ihm die Stadt und Kultur zeigen.
Geeignet ist dieser Ansatz weniger bei größeren Reisegruppen oder Familien. Zusätzlich liegt vorallem auf der Vermieterseite kein finanzieller Gewinn und das Eindringen in seinen privaten Lebensraum kann viele potentielle Nutzer abschrecken.
Airbnb ermöglicht die private Vermietung, der Kostenpunkt ist jedoch weiterhin hoch und in beliebten Gegenden ist weiterhin eine Reservierung von nöten. 

Grundsätzlich lassen sich in den Beispielen positive Ansätze finden, die beibehalten und ausgebessert werden können. Vorallem aber die Negativpunkte sollen ausgebessert werden.\\ Aus allen Betrachtungen ergeben sich damit folgende Ansatzpunkte für potentielle Optimierungen:

\subsection{Mieter und Vermieter}
\begin{itemize}
   \item
   \textbf{Einheitliches Kommunikationssystem}: Die Kommunikation muss nicht über eine Webpräsenz oder diverse unterschiedliche Wege stattfinden. (Nachrichtensystem der Webpräsenz, Email, Telefon), sondern wird von allen Anwendern über die gleiche Software geschehen. Auch die Bezahlung kann auf diesem Weg abgeschlossen werden.

   \item 
   \textbf{Zeitoptimierung}: Gängige Beispiele setzen auf Angebot und Nachfrage Anzeigen, diese können veralten oder nicht aktuell sein und zu spät gelesen werden. Der passende Vermieter erhält im neuen System direkte Anfragen die zeitliche und inhaltliche Relevanz haben und kann diese direkt beantworten. Der Mieter soll dadurch in kürzerer Zeit eine Antwort erhalten.

   \item
   \textbf{Filtern relevanter Anfragen (1)}: Anhand der Benutzerdaten, findet eine Kontaktaufnahme nur zwischen kompatiblen Benutzern statt. Dadurch verringert sich die Anzahl zielloser Anfragen und Kommunikationen.

   \item 
   \textbf{Zeitliche Unabhängigkeit}: Während der Vermietung besteht die Möglichkeit, dass alle Beteiligten ihre Aktivitäten unabhängig voneinander ausführen können. Der Mieter ist (im Vergleich zu Couchsurfing) nicht unbedingt auf den Zugang zur Wohnung durch den Vermieter angewiesen. 

\end{itemize}


\subsection{Vermieter}
\begin{itemize}
   \item 
   \textbf{Finanzieller Anreiz}: Vermieter haben die Möglichkeit für ihre Vermietung Kosten zu erheben und finanziellen Gewinn zu schlagen.
   Dabei muss es sich nicht nur um die Verleihung der Wohnung handeln, sondern grundsätzlich vorhandene Grundstücke wie Gärten, Hof, Landstücke.

   \item 
   \textbf{Kontrolle der Privatsphäre}: Im Gegensatz zum Couchsurfing lässt sich der Bereich eingrenzen, indem Reisende in den eigenen Lebensraum eindringen können. Auch ein permanenter sozialer Kontakt ist nicht von nöten, sodass der ganze Prozess auf einer rein geschäftlichen Ebene ausgetragen werden kann.

   \item 
   \textbf{Sicherheit der Daten (2)}: Sensiblen Informationen können nur bei Bedarf freigegeben werden und sind nur lokal gespeichert. Damit erhalten nur Kunden auch die benötigten Informationen und das freie Einsehen über eine Webpräsenz ist nicht möglich.

\end{itemize}
   

\subsection{Mieter}
\begin{itemize}
   \item 
   \textbf{Mobilität (3)}: Eine mobile Anwendung unterstützt die verbreiteste Technologie, die Reisende in der Regel mit sich führen. Zusätzlich dazu können die unique Features eines Smartphones anwendung finden.
   Er ist flexibler und muss sich nicht um Zugung zu stationären Rechnern kümmern.

   \item 
   \textbf{Vergrößerte Reise- und Interessentengruppe}: Es ist möglich mit einer größeren Anzahl an Personen zu verreisen, die innerhalb einer Wohnung nicht untergebracht werden können. Dazu bestünde auch die Option Familien mit Kindern unterzubringen (falls man diese als Host nicht aufnehmen würde). 

   \item
   \textbf{Soziale Einstellung berücksichtigen}: Nicht jeder möchte viel Kontakt mit seinen Host haben und es besteht die Möglichkeit, dass sich Leute auf den geschäftlichen Prozess beschränken wollen. Eventuell besteht kein Interesse an den sozialen Aspekten des Sharings.

   \item 
   \textbf{Spontanität}: Reisende können auf ihrer Reise spontanere Suchanfragen starten und müssen sich nicht zwangsläufig an vorgegebene Routen halten. Theoretisch ist dies auch unter bereits vorhanden umständen möglich (Reisender entscheidet in diesem Ort selbstständig eine Unterkunft zu suchen), aber die Applikation unterstützt hierbei speziell die Suche. Im Gegenzug dazu, verliert der Vermieter jedoch organisatorische Sicherheit und es kann die Gefahr auftreten, dass Anfragen nicht angenommen werden können, da sie zu kurzfristig erscheinen.

\end{itemize}

Viele dieser Punkte, ergeben sich dabei auch als Folge des Handlungskontextes. Gerade auf Mieterseite werden einige Chancen ermöglicht, jedoch lässt sich nicht mi Sicherheit auf diese setzen und haben für die Entwicklung des Projektes keine größere Priorität. 

Speziell die mit 1 - 3 markierten Aspekte, sollen in der weiteren Betrachtung fokusiert werden, da sich diese als Alleinstellungsmerkmale auszeichnen und für das System als besonders relevant erachtet werden.
 


\newpage

% Risiken
%!TEX root = ../konzept.tex

\section{Risiken}
Zusätzlich zu den Chancen, wurden im Weiteren auch potentielle Risiken betrachtet, die sowohl mit der grundlegenden Thematik als auch mit der späteren Umsetzung während des Projektes auftreten können.
Dieser Schritt diente zudem der Abwägung, ob das Verhältnis zwischen Chancen und Risiken letztendlich zu einem gewinnbringenden und zielerfüllenden Ergebnis führen kann.

\begin{itemize}
   \item \textbf{Sicherheitsaspekte}:\\ Sowohl auf Daten bezogen, als auch beim Kontakt mit den Leuten. Wenn die Vermieter ihre persönlichen Daten angeben, gewähren sie Interessenten einen gewissen Einblick in ihr Eigentum. Kommt dazu noch die Lokation durch GPS, besteht potentielle Einbruchgefahr. Daher sollte eine Verifikation der Anwender stattfinden und den Benutzern die Freiheit über ihre Daten gewährleistet werden. Auch der Fremdzugriff bei verlorenen oder gestohlenen Geräte sollte bedacht werden (Codesperre, Identifikation). Da die gesamte Bezahlung über die Anwendung abgeschlossen werden soll, muss auch dahingehend eine gewisse Sicherheit gewährleistet werden. Um vorallem die permanenten Daten zu schützen, soll sich die Systemarchitektur diesem Problem annehmen\footnote{Kapitel 4.2 Konzeptseite 30}.

   \item \textbf{Aktivität der Nutzer}:\\ Das System wächst und fällt mit der Benutzerbeteiligung. Gefahr besteht, wenn angesprochene Zielgruppe nicht ausreichend vom System angesprochen. (Sowohl durch qualitative als auch funktionale Aspekte.) Zu einer langfristigen Bindung und das Schaffen positiver Erfahrungen, kann ein Reputations- und Reviewsystem eingebaut werden, sowie eine Freunde/ Bekanntenfunktion mit der Kontakte gehalten werden können. (Soll aber nicht in diesem Projekt betrachtet werden.)

   \item \textbf{Kosten}:\\ Um das Projekt langfristig betreiben zu können, muss die finanzielle Absicherung gewährleistet sein. Daher soll ein mögliches Geschäftsmodell erarbeitet werden, mit welchen das System die Kosten decken kann\footnote{Kapitel 5 Konzeptseite 36}.

   \item \textbf{Eingeschränkte Zielgruppe}:\\
   Aufgrund technischer Einschränken enthält nur ein geringer Anteil der potentiellen Interessenten Zugang zum System. Das kann sich auf die verfügbare Hardware beziehen, sodass ältere Interessenten beispielsweise keine Möglichkeit haben das Programm zu benutzen oder Einschränkungen in der Verbreitung. Wenn zum Beispiel nur bestimmte Betriebssystemversionen unterstützt werden. Das Projekt betrachtet nur die Entwicklung für Geräte mit Android Betriebssystem, in Zukunft wäre eine Portierung auf weitere Plattformen eine Möglichkeit zur weiteren Verbreitung.
     
   \item
   \textbf{Mieten ja, Vermieten nein}:\\ Motivation als Vermieter ist nicht für die breite Masse gegeben. Jüngere und grundsätzlich aufgeschlossene Leute, die auf soziale Kontakte etc. aus sind, würden teilnehmen, besitzen in der Regel aber kein eigenes Haus oder Land zum vermieten. Daher muss auch für Landbesitzer unterschiedlichen Alters oder Einstellungen eine deutliche Motivation und Sicherheit gewährleistet werden. Demnach muss das System auch unerfahrenen Anwendern einen leichten Einstieg und ausreichende Nutzungsmotivation bieten\footnote{Kapitel 3.3 Konzeptseite 20}.

   \item
   \textbf{Lokale Verbreitung}:\\ Angebotene Grundstücke sind nicht in allen Regionen vorhanden. In ländlichen Gegenden ist die Wahrscheinlichkeit größer, dass Grundstücksbesitzer vorhanden sind, während in Großstädten keine Zielgruppe vertreten ist. Speziell dazu muss auch betrachtet werden, wer eigentlich in diese ländlichen Gegenden reisen würde und ob diese Zielgruppe dann entsprechende Technologien besitzen.

   \item
   \textbf{Software und Hardware spezifische Probleme}:\\
   Probleme in der Funktionalität können während der Entwicklung beseitigt werden, Hardwarefehler nicht zwangsläufig. Wichtig ist das Schaffen von Kompromissen in Fehlersituationen. 
   Da die Mieter in unterschiedlichen Gegenden unterwegs sind und nicht immer Verbindung zum Internet durch Netzabdeckung haben, sollte man sich dahingehend Ansatzpunkte überlegen. Z.b. nach Annahme eines Angebotes werden die Koordinaten des Vermieters lokal zwischengespeichert, sodass auch ohne Verbindung der Standort gefunden werden kann und Kontaktmöglichkeiten wie Telefonnummer gesichert sind. Auch der Zeitpunkt der Nachrichtenübermittlung kann ausschlaggebend sein. Da die Anwendung nicht permanent aktiv ist, eignet sich vorallem ein Kommunikationsmodell, bei dem Nachrichten in der Cloud zwischengespeichert werden und bei Aktivität weitergeleitet werden. Es sollte nicht damit gerechnet werden, dass Reisende und Vermieter zur gleichen Zeit aktiv sind. 

\end{itemize}

Als wesentliches Risiko dieser Untersuchung, ging die Verfügbarkeit der Vermieter hervor. Das Prinzip lässt sich nur dann erfolgreich verwirklichen, wenn gesuchte Grundstücksbesitzer auf das System aufmerksam werden, benötigte Technologien besitzen, mit der Anwendung interagieren können und eine Motivation besitzen ihr Grundstück zu vermieten. 


% Geschäftsmodell
%!TEX root = ../konzept.tex

\section{Geschäftsmodell}
TODO Ausformulieren, Information zum Gewerbe anmelden

Mögliche Ansätze
\begin{itemize}
   \item Kostenlose Nutzung, dafür einmalige Zahlung für App. Keine Zutsatsgebühren für Angebote, Verleihprozesse etc.\\
   Vorteil: Nutzer werden nur einmalig mit Appkosten konfrontiert, geringe Folgekosten für Anwender.
   Nachteil: Schwer kritische Menge zu erreichen, direkte Kosten können Einstiegshürde darstellen, keine langfristigen Einnahmen für Serverkosten etc. Finanzierung müste mit anderen Mitteln wie Werbung erreicht werden. Da kostenpflichtige Applikation eher unpassend. In diesem Fall eine werbefinanzierte Variante gratis und kostenpflichtige werbefreie Option.

   \item In-App Käufe, Guthaben aufladen. Die Bezahlung der Mietvorgänge wird komplett über die Applikation geregelt. Aufladen des entsprechenden Guthaben gibt Anteil. Weitere Möglichkeit In-App Käufe mit zusätzlichen Funktionen (ausbaubar)

   \item Werbung + Eventkooperationen.
   Innerhalb der Anwendungen gibt es die Möglichkeit mit Partnern zu kooperieren. Vorstellbar wären zB Eventpartner, die bei bestimmten Veranstaltungungen dafür werben und entsprechende Angebote daauf auslegen. Valider Punkt, da zudem auch Events als Motivationsgrund der Anwender gelten kann. Zusätzlich dazu innerhalb einer (werbepflichten) Applikation Werbung von Partnern der entsprechenden Domäne.

   \item Bezahlung läuft über App ab. Dadurch haben die Benutzer eine Bestätigung und Sicherheit, da sie zum einen bargeldlos bezahlen können und eine Absicherung über den Mietvorgang haben. (Keiner kann sich nach angenommener Leistung aus der Bezahlung abwenden). Durch den Bezahlprozess wird eine Anteilmäßige Provision draufgerechnet. 


\end{itemize}
 
