%!TEX root = ../konzept.tex

\section{Anforderungsanalyse}

\subsection{Problemanalyse}
Um einen ersten Ansatz für potentielle Ansatzpunkte zu finden, wurde ein Problemszenario definiert, dass einen derzeitigen Handlungsablauf widerspiegelt. Dazu wurde mit fiktiven Persona der primary User Gruppe gearbeitet, jedoch nur mit wenigen spezifischen Charakteristiken der einzelnen Menschen gearbeitet.\\

1 SZENARIO: Vater-Sohn
Der 40-jährige Vater will eine Fahrrad-Tour von München nach Berlin mit seinem 15-jährigen Sohn ab nächster Woche in der Zeit vom 1. Juli bis 10. Juli machen. Die Fahrt hin soll komplett mit dem Fahrrad zurückgelegt werden. Dabei soll ein Stück durch Tschechien gefahren werden. Die Strecke Berlin-München zurück soll größenteils mit der Bahn absolviert werden. Für die Übernachtungen nimmt der Vater ein Zelt mit, denn um Kosten zu sparen sollen Hotels vermieden werden.
Der Vater hat eine Fahrradstrecke ausgesucht wo er damit rechnet. dass pro Tag 100km zurückgelegt werden können. Auf Basis der Strecke begibt er sich auf die Suche nach passenden Zeltplätzen. Als Bedingung für einen Zeltplatz ist zum einen der benötigte Platz - das Zelt ist 3x3m groß - sowie Zugriff auf Wasser. Eine Dusche mit warmen Wasser ist nicht zwingend notwendig.
Die Fahrradstrecke ist so gewählt, dass größere Städte und Umwege vermieden werden, sowie die strategisch kürzeste Strecke.
Der Vater beginnt mit einer allgemeinen Suche über Google über den Suchbegriff “private Zeltplätze Langenbach”. Dabei stößt er auf die Seite http://www.schwarzzeltvolk.de/gardensurfing-private-garten-als-zeltplatze/
Dort wird ein Portal names “Freagle.org” und “CampinmyGarden.com” vorgestellt. Er registiert sich zunächst bei “Freagle.org”. Nach der Anmeldung stellt er allerdings fest, dass das Portal auf Gemeinschaft und Gastfreundschaft setzt, heißt er muss selbst einen Garten anbieten. Der Vater hat jedoch keinen eigenen Garten.
Das Portal “CampinmyGarden.com” ist nur auf englischer Sprache verfügbar. Nach einer kurzen Auffrischung seiner Englischkenntnisse findet er sich auf der Seite zurecht. Eine erste Suche zeigt schnell, dass das Angebot in Deutschland recht eingeschränkt ist.
Er findet allerdings einen Garten in Stammham: http://campinmygarden.com/campsites/993
Weitere Gärten findet er im Raum Nürnberg, Illmenau, Leipzig und Dresden. Der Vater fragt die Vermieter über das Portal an. Dafür muss er immer wieder die selben Daten für die Anfrage über ein Formular abschicken. Eine direkte Kontaktmöglichkeit vorab scheint nicht möglich zu sein.
Der Plan, auch einen Stopp in Tschechien zu machen wird erstmal gestrichen, da hier die Sprachbarriere sowie die Unsicherheit nicht von dem Portal gelöst werden konnte.
Da die Gärten nicht auf der schon vorher angedachten Strecke liegen, muss diese dementsprechend geändert werden.

Es ist der 1. Juli und die Reise beginnt. Einige Stunden später ist der Vater mit seinem Sohn in Stammham angekommen. Er holt sein Smartphone raus und merkt, dass er nun alle Daten zusammen suchen muss, um sie in seiner Navigation eingeben zu können um das engültige Ziel zu finden. Er hofft, dass der Vermieter zu Hause ist, da der vorab angekündigte Termin um 1 Stunde schon überzogen ist. Am Garten angekommen ist keiner mehr da. Der Vater versucht den Vermieter zu erreichen, zunächst per Mail, dann per Telefonanruf. Kurze Zeit später meldet sich der Vermieter und empfängt den Vater und den Sohn.

Der nächste Tag bricht an und die Reisenden wollen sich wieder auf den Weg machen. Die Bezahlung wurde nicht vorab getätigt und muss nun nachgeholt werden. Der Vater hat jedoch aus Sicherheitsaspekten nicht so viel Geld dabei und muss erst zur nächsten Bank, um die benötigte Summe abzuheben.\\

Im Problemszenarien wurden die Benutzer mit folgenden Problemen konfrontiert: 
TODO\\

Aus diesem Beispiel und vertiefender Betrachtung, konnten für das geplante System daher folgende funktionale und nonfunktionale Nutzeranforderungen ermittelt werden.
        	

\subsection{funktionale Anforderungen}        	

TODO ausformulieren


- Funktionen für effektive und effiziente Erledigung
- Informationsrepräsentation am User Interface (was wird benötigt
- Allokation finden: besonders potentielle Stellen für Aufgabenunterstützung finden


Suchen im Radius
Profil eines Camps
Fotos
Anzahl
Belegung
Bewertung
Möglichkeiten (Duschen, etc)
Zelt schon vorhanden
Reservierung
Kontaktaufnahme (Google Cloud Messaging)
Check-In/Check-Out
Bewertung, wenn Ort wieder verlassen wird (GPS)
Statistiken
Badges/Bewertung
Push-Benachrichtigungen
Mieter
für im Umkreis befindlicher Camps nach Aktivierung
Vermieter
für im Umkreis befindliche Sucher



\subsection{qualitative Anforderungen}   
TODO 
\subsubsection{Zuverlässigkeit}
\subsubsection{Usabillity}
\subsubsection{Effiziens}


