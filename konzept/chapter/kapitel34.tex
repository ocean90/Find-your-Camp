%!TEX root = ../konzept.tex

\section{Anforderungsanalyse}
Ziel des Projektes ist die Entwicklung einer Software, welche im Vergleich zu vorhandenen Systemen der Domäne, gebrauchstauglicher ist. Bei der Untersuchung der Benutzerinteraktionen, sollen zudem Ansätze gefunden werden, bei denen sich, unter anderem durch Einbezug von unique features der Geräte, Möglichkeiten zur Produktivitätssteigerung ergeben. 
Um einen ersten Ansatz für potentielle Verbesserungsstellen zu finden, wurde ein Problemszenario definiert, dass einen derzeitigen Handlungsablauf widerspiegelt. Dazu wurde mit fiktiven Persona der primary User Gruppe gearbeitet, jedoch vorerst nur mit wenigen spezifischen Charakteristiken der einzelnen Menschen gearbeitet.\\

1 Szenario: Vater-Sohn
Ein 40-jähriger Vater will eine Fahrrad-Tour von München nach Berlin mit seinem 15-jährigen Sohn ab nächster Woche in der Zeit vom 1. Juli bis 10. Juli machen. Die Strecke hin soll komplett mit dem Fahrrad zurückgelegt werden. Dabei soll auch ein Stück durch Tschechien zurückgelegt werden. Der Rückweg von Berlin nach München, soll größenteils mit der Bahn absolviert werden. Für die Übernachtungen nimmt der Vater ein Zelt mit, denn um Kosten zu sparen sollen Hotels vermieden werden.
Der Vater hat eine Fahrradstrecke ausgesucht und rechnet damit, dass pro Tag 100km zurückgelegt werden können. Auf Basis der Strecke begibt er sich auf die Suche nach passenden Zeltplätzen. Eine Bedingung zum potentiellen Zeltplatz, ist zum einen der benötigte Platz - das Zelt ist 3x3m groß - sowie Zugriff auf Wasser. Eine Dusche mit warmen Wasser ist nicht zwingend notwendig.
Die Fahrradstrecke ist so gewählt, dass größere Städte und Umwege vermieden werden und die strategisch kürzeste Strecke zugrunde liegt.\\
Der Vater beginnt mit einer allgemeinen Suche über Google über den Suchbegriff “private Zeltplätze Langenbach”. Dabei stößt er auf die Seite\\ http://www.schwarzzeltvolk.de/gardensurfing-private-garten-als-zeltplatze/
Dort wird ein Portal names “Freagle.org” und “CampinmyGarden.com” vorgestellt. Er registiert sich zunächst bei “Freagle.org”. Nach der Anmeldung stellt er allerdings fest, dass das Portal auf Gemeinschaft und Gastfreundschaft setzt, heißt er muss selbst einen Garten anbieten. Der Vater hat jedoch keinen eigenen Garten und keine Möglichkeit sich auf anderen Wegen zu registrieren (3 Freunde werben). Eine Mitgliedsgebühr möchte er auch nicht zahlen, da er noch gar nicht sicher ist, wirklich ein passendes Angebot zu finden.\\
Das Portal “CampinmyGarden.com” ist nur auf englischer Sprache verfügbar. Nach einer kurzen Auffrischung seiner Englischkenntnisse findet er sich auf der Seite zurecht. Eine erste Suche zeigt schnell, dass das Angebot in Deutschland recht eingeschränkt ist.\\
Er findet allerdings einen Garten in Stammham: http://campinmygarden.com/campsites/993
Weitere Gärten findet er im Raum Nürnberg, Illmenau, Leipzig und Dresden. Der Vater fragt die Vermieter über das Portal an. Dafür muss er immer wieder die selben Daten für die Anfrage über ein Formular abschicken. Eine direkte Kontaktmöglichkeit vorab scheint nicht möglich zu sein.
Der Plan, auch einen Halt in Tschechien zu machen, wird erstmal gestrichen, da hier die Sprachbarriere sowie die Unsicherheit nicht von dem Portal gelöst werden konnte.
Da die Gärten nicht auf der schon vorher angedachten Strecke liegen, muss diese dementsprechend geändert werden.\\

Es ist der 1. Juli und die Reise beginnt. Einige Stunden später ist der Vater mit seinem Sohn in Stammham angekommen. Er holt sein Smartphone raus und merkt, dass er nun alle Daten zusammen suchen muss, um sie in seiner Navigation eingeben zu können um das engültige Ziel zu finden. Er hofft, dass der Vermieter zu Hause ist, da der vorab angekündigte Termin zeitlich nicht ganz eingehalten werden konnte. Der Vater versucht den Vermieter zu erreichen, zunächst per Mail, dann per Telefon, er erreicht ihn jedoch nicht direkt. Kurze Zeit später meldet sich der Vermieter und empfängt den Vater und den Sohn.

Der nächste Tag bricht an und die Reisenden wollen sich wieder auf den Weg machen. Die Bezahlung wurde nicht vorab getätigt und muss nun nachgeholt werden. Der Vater hat jedoch aus Sicherheitsaspekten nicht so viel Geld dabei und muss erst zur nächsten Bank, um die benötigte Summe abzuheben.\\


Anhand dieses Problemszenarios, lassen sich folgende Aktivitäten ausmachen:
\begin{itemize}
   \item 
   Planung der Route, dabei Kalkulieren mit ungenauen Werten (Reisegeschwindigkeit)
   \item 
   Suchen von preislich angemessenen Anbietern, Besichtigung mehrerer Portale
   \item 
   Kontaktaufnahme mit mehreren Anbietern (auf mehreren Portalen), dabei wiederholte Eingabe der Daten
   \item
   Navigation zum Zielort, Verwalten der Zielinformationen
   \item
   Kontaktaufnahme mit Vermieter
   \item
   Bezahlung

\end{itemize}

Das Szenario orientierte sich an einem möglichen Ablauf unter jetzigen Voraussetzungen. Da nicht jede Situation unter optimalen Bedingungen abläuft, wurde der Reisende auch hier vor diverse Herausforderungen gestellt, die realistisch erscheinen. \\
Ein Problem das während der Planung auftritt, ist die Zeitkalkulation. Indem er vorher die Route mit festen Zeiten plant und dementsprechend unterschiedliche Anbieter anschreiben muss, ist er während seiner Reise auf diese Strecke und auf Abmachungen angewiesen. Abweichungen im Fahrplan oder zeitliche Rückschläge stellen ihn vor neuen Problemen. 

Bei der Benutzung der Portale wird er mit weiteren Schwierigkeiten konfrontiert. Bei Freagle findet keine Registrierung statt, da er selbst weder einen Garten besitzt noch extra Gebühren zahlen möchte. Campinmygarden stellt ihn anfangs vor eine geringe Sprachhürde. Dennoch klappt die Kontaktaufnahme. Für mehrere Anfragen muss er jedoch wiederholte Tätigkeiten ausführen und wird in seinem Suchverhalten nicht unterstützt.

Bei der Suche des Zielortes muss er die Informationen erstmal auf sein Smartphone speichern und anschließend Kontakt mit dem Vermieter aufnehmen. Da er nur die Emailadresse kennt und eine Telefonnummer, auf der er nicht zu erreichen ist, muss er warten bis der Vermieter zurück ist. 
Problem an dieser Kontaktaufnahme kann sein, dass der Vermieter unterwegs nicht immer seine Emails kontrolliert und eventuell die Hausnummer angegeben hat. Auch die Bezahlung läuft nicht ohne Verzögerung ab und hat Einfluss auf den Zeitplan.\\

Aus diesem Beispiel und vertiefender Betrachtung, konnten erste funktionale Anforderungen ermittelt werden.
        	

\subsection{Funktionale Anforderungen}        	
Im Vergleich zu obigem Beispiel sollte dabei beachtet werden, dass die Software auf einem Smartphone läuft und von dem Kontext ausgegangen wird, dass ein Benutzer auf seiner Strecke die entsprechenden Suchfunktionen nutzt. Zusätzlich dazu wurden auch Grundfunktionalitäten betrachtet, die nicht mit dem direkten Suchen zusammenhängen.

\begin{itemize}
   \item 
   Registrierung der Anwender und Speichern von relevanten Informationen.
   \item
   Erstellung und Bearbeitung eines Benutzerprofils/ Camp Profils.
   \item
   Erstellen eines Reiseprofils: geplante Reisezeit, gewünschte Ausstattung und Gruppengröße, anhand dessen das Matching stattfindet.
   \item
   Lokalisierung über GPS.
   \item 
   Suchen nach vorhanden Grundstücksanbietern und Rückmeldung über Suchtreffer.
   \item 
   Anzeige der Anfrage beim potentiellen Vermieter.
   \item
   Kontaktaufnahme und Kommunikation des Mieters und Vermieters (Google Cloud Messaging) über Nachrichten.
   \item
   Übersenden von Kontaktinformationen: Reiseinformationen des Mieters, sowie Grundstücksinformationen des Vermieters. 
   \item 
   Matchingfunktion anhand derer relevante Anfragen gefiltert werden.
   \item 
   (Eventuell) Lokale Speicherung der Kontaktinformationen zur GPS Benutzung oder bei Verbindungsabbruch.
   \item
   Bezahlfunktion über Applikation.
   \item
   Bewertung des Benutzers.
   \item
   Einblenden von Werbeaktionen

\end{itemize}

\newpage 

\subsection{Qualitative Anforderungen} 
Neben den funktionalen Anforderungen, ergaben sich auch erste nonfunktionale Anforderungen an das System.


Qualitätsattribute der gewünschten Funktionen,
Anforderungen an das implementierte System als 
Ausführungsverhalten (Verarbeitung unter Echtzeitbedingungen, Auslastung von Ressourcen, Genauigkeit, Antwortzeiten, Durchsatz, Speicherbedarf)

\begin{itemize}
   \item 
   \textbf{Zuverlässigkeit}: Funktionen müssen zuverlässig arbeiten und die Aktivitäten zielgerichtet unterstützen. Dazu gehört vorallem die genaue GPS Ortung, ein passendes Matchingsystem und die effektive Nachrichtenweiterleitung. Auch der Bezahlvorgang sollte problemlos funktionieren, da hierbei nachhaltiger Schaden bei Fehlern entstehen kann. 

   \item
   \textbf{Ausfallsicherheit}: Da die Anwendung verstärkt im freien Verwendung findet, muss die Anwendung auf Verbindungsausfälle reagieren. Dazu kann bei leerem Akku eine Unterbrechung während einer Aktivität stattfinden. In folge dessen, mussen die Information weiterhin unversehr bleiben.

   \item
   \textbf{Robustheit}: Die Anwendung muss, speziell für das Matching, sehr robust bei fehlerhaften Eingaben des Anwenders sein. Bei fehlerhafter Benutzung (z.B. durch Unerfahrenheit oder durch Auswahl einer falschen Option aufgrund von Sonneneinstrahlung/Sichteinschränkung), sollen für den Anwender keine Folgeschäden auftreten.

   \item
   \textbf{Usability}: Geforderte Funktionen müssen ausführbar sein und möglichst effektiv zum Ziel führen. Dabei sollte auch das physikalische Umfeld betrachtet werden, da der Anwender sich verstärkt im freien Aufhalten wird und unter zeitlichen Faktoren stehen kann.
   
   \item 
   \textbf{Effiziens}: Funktionen müssen die Suche gezielt unterstützen und eine sinnvolle Allokation der Arbeitsschritte muss vorhanden sein, um dem Benutzer möglichst viel Arbeit abzunehmen. (Mit möglichst geringem Aufwand an gewünschtes Ziel führen.)
   Die einzelnen Aktivitäten müssen möglichst schnell durchgeführt und Daten schnell weitergeleitet werden. Da mit Verbindungsabbrüchen zu rechnen ist, sollen einzelne Schritte während der Kommunikation zwischen Mieter und Vermieter zwischengespeichert werden.
   
   \item
   \textbf{Sicherheit}: Permanent gespeicherte Daten müssen sicher gespeichert und übertragen werden. 

   \item
   \textbf{Barrierefrei und Zugänglichkeit}: Applikation sollte so gestaltet sein, dass eine möglichst große Interessentengruppe angesprochen wird. Sie sollte daher einfach und effizient zu benutzen sein, leicht erlernbar und den User unterstützen. Von der  Gestaltung sollten auch Anwender mit visuellen Beeinträchtigungen unterstützt werden.
\end{itemize}



\subsection{Organisatorische Anforderungen}          
\begin{itemize}
   \item
   \textbf{TODO 1}

   \item 
   \textbf{TODO 2}

   \item
   \textbf{TODO 3}

\end{itemize}
