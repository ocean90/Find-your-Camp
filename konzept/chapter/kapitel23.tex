%!TEX root = ../konzept.tex

\section{Marktanalyse}
    Die Idee private Grundstücke als Campingplatz zu verleihen ist nicht neu. Schaut man sich den vorhandenen Markt an, findet man bereits Unternehmen die sich mit genau diesem Thema beschäftigen. Die Anzahl an vorhandenen Anbietern ist dabei aber noch verhältnismäßig gering und das Prinzip eine Entwicklung aus kürzerer Zeit. Das der Verleih des eigenen Platzes als Unterkunft für Reisende, eine akzeptierte Alternative gegenüber herkömmlichen Unternachtsmöglichkeiten darstellt, wurde bereits im Vorfeld herausgestellt.\\
    Speziellen Fokus auf Campingmöglichkeiten legt dabei das im Jahr 2009 gegründete Unternehmen Freagle\footnote{www.fragle.org}. Regional gibt Freagle keine Einschränken vor und ermöglicht es Benutzers weltweit teilzunehmen. Mietung und Vermietung ist Benutzer prinzipiell kostenfrei. Bei der Anmeldung ist es vorgesehen einen eigenen Platz anzubieten. Sollte man dazu keine Gelegenkeit haben, besteht unter anderem die Option für eine Jahresgebühr von 12,50 Euro beizutreten. Vorhandene Angebote werden auf einer Weltkarte angezeigt und können ausgesucht und angefragt werden. Zur Sicherheit wird eine Verifizierung des Ortes vorgesehen und Mitglieder benötigen eine Freagle Card, die als ID der Leute dient.

    Freagle ist rein webbasiert und bietet für mobile Anwender keine  Applikation oder angepasste Website. Der Login über ein Smartphone ist nicht möglich\footnote{Option verfügbar, aber nicht funktionabel}.
    Das Verhältnis von angebotene Plätzen zu Mitglieder beträgt etwa 1:5 und zeigt, dass grundsätzlich nicht alle Leute einen Platz anbieten können oder wollen.

    Speziell an Freagle ergeben sich damit mehrere Ansatzpunkte, die verbessert werden können. Da gerade Reisende auf Flexibilität angewiesen sind und oftmals keinen Rechner mit sich führen, dient das Smartphone für viele als wesentliches Hilfsmittel. Gerade in diesem Kontext ist eine dafür angesetzte Weboberfläche oder Applikation von großem Nutzen. Zudem wird eine frühzeitige Planung und Kontaktaufnahme vorrausgesetzt und die Suche wird komplett in die Hände des Benutzers gelegt. In diesem Fall könnte man sich den Möglichkeiten eines Smartphones bedienen und unrelevante Ergebnisse direkt rausfiltern. 

    Um das Interesse der Vermieter zu erhöhen, besteht die Option neben sozialen Kontakten einen weiteren Anreiz zu bieten. Geplant ist hierbei ein kostenbasiertes Vermietsystem, dass als Einnahmequelle dienen kann. Dabei stellt sich grundsätzlich die Frage, ob Benutzer dazu bereit wären für einen solchen Dienst zu zahlen, aber auch am Beispiel Freagle zeigt sich, dass die Interessenten eine jährliche Gebühr zahlen, um solch ein Angebot wahrzunehmen.\\
    
    Ein weiteres Beispiel, dass aber auch lediglich über eine Webpräsenz verfügt, ist \\  http://campinmygarden.com. Hierbei liegt der lokale Schwerpunkt und Marktanteil deutliche auf Großbritanien. Weitere Länder werden prinzipiell unterstützt, aber die Teilnehmerzahl ist im Vergleich zum ersten Beispiel deutlich geringer.
    Eine Besonderheit die Campinmygarden auszeichnet, ist die Einbeziehung anstehender Events. Ansonsten werden ähnliche Funktionalitäten angeboten, wobei TODO Preis.\\

    Existierende Applikationen, wie der ADAC Camping- und Schnellplatzführer 2013, unterstützen das Suchen und Finden von Zeltplätzen, beziehen dabei aber lediglich die öffentlichen Anbieter ein. Aufgrund des Preisfaktors, sowie den üblichen Reservierungs-/Buchungsvoraussetzungen kommen sie deshalb (in der Regel\footnote{Für den Fall das in der Gegend kein Grundstücksanbieter angemeldet ist, muss der User sich zwangsläufig an solche Angebote wenden. }) nicht in Frage und unterstützen lediglich die Mieter, aber nicht die privaten Vermieter.\\

    Für die ausgewählte Thematik, schließen diese Beispiele einen großen Teil des bisher vorhandenen Anbieter an (Webpräsenz) oder Repräsentieren die Kernfunktionalität vieler ähnlicher Anwendungen. 
    Mit der Analyse des Marktes, folgt die Chancenermittlung und das Herausstellen der Alleinstellungsmerkmale.

