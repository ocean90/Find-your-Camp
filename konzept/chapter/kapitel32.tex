%!TEX root = ../konzept.tex

\section{Nutzungsmotivation}
In Bezug auf den Nutzungskontext und den besprochenen Verbesserungsansätzen zu vorhandenen Systemen im Rahmen der Projektbeschreibung\footnote{Kapitel 2.2 und 2.3}, soll folgende Motivation zur Benutzung der Find your Camp Applikation vorhanden sein.\\

Mieter und Vermieter
\begin{itemize}
   \item 
   \textbf{Anwendung als Applikation bisher nicht vorhanden}: Handy auf Reisen flexibler und wahrscheinlicher als Laptop. Damit auch schneller zugänglicher. Einheitliches Kommunikations- und Bezahlsystem.

   \item 
   \textbf{Zeitoptimierung}: Aktuelle Anfragen werden direkt weitergeleitet und bearbeitet. Der Vermieter erkennt direkte Anfragen und muss nicht warten bis jemand auf sein Inserat antwortet. (Daher auch veraltete Anzeigen vermieden.) Der Mieter erhält ein schnelleres Feedback und kann mit einer einzigen Anfrage alle potentiellen Anbieter erreichen.

   \item 
   \textbf{Anfrageunterstützung gewährleistet}: Die Suche und Anfrage wird zu einem großen Teil von der Anwendung übernommen, der Mieter muss keine lange Zeit beim lesen der Angebote/Anfragen verbringen, die letztendlich nicht in Frage kommen. Die Tätigkeit wird im Vergleich zu bisherigen Angeboten einfacher umzusetzen und effektiver gestaltet.

   \item 
   \textbf{Soziale Erfahrung}: Ein persönlicher Kontakt mit den Personen wird angestrebt und kann zum Gewinn von neuen Erfahrungen und Einsichten von Menschen und Kulturen führen.
    Besteht kein Interesse an dieser Auseinandersetzung ist die Vermietung auf reiner Geschäftsebene ebenfalls möglich.

   \item
   \textbf{Kontrolle der Benutzerdaten}: Informationen und Daten werden nicht so leicht zugänglich gemacht und nur an relevante Personen vergeben.

\end{itemize}

Mieter
\begin{itemize}
   \item 
   \textbf{Flexiblere Reisen}: Es muss keine frühzeitige Buchung erfolgen und Abweichungen in der Reiseroute führen nicht zwangsläufig zu (finanziellen) Folgen.

   \item 
   \textbf{Kostenmotivation}: Gegenüber öffentlichen Unterkunftsanbieter, fallen deutlich geringere Kosten zur Mietung an. 

   \item 
   \textbf{Shared Economy für größere Gruppe}: Auch für größere Reisegruppen oder Familien, wird eine Möglichkeit geboten, um von den öffentlichen Angeboten abzuweichen. Speziell beim Couchsurfing ist die Gruppengröße oftmals reduziert, sodass das Reise auf diese Weise für Interessenten zusätzlich erschwert war.

\end{itemize}

\newpage
Vermieter
\begin{itemize}
   \item 
   \textbf{Einnahme}: Eine Möglichkeit ungenutztes Grundstück zusätzlich zu vermieten und dadurch Einnahmen zu erzielen.

   \item 
   \textbf{Kontrolle der Privatsphäre}: Eindringen in Lebensraum kann im Vergleich zum Couchsurfing stärker beeinflusst werden.
   
\end{itemize}


\subsection{Gewerbe}
Auf eine Motivation der Vermieter sollte hierbei jedoch deutlich geachtet werden. Ist ein Anbieter darauf aus, mit seinem Grundstück hauptsächlich Einnahmen zu erzielen, muss es ihm bewusst sein, dass er in diesem Fall eine Tätigkeit mit Gewinnerzielung betreibt. Vom System her ist es nicht zwangsläufig notwendig, für die Vermietung seines Grundstückes auch eine Gegenleistung einzufordern. Macht er dies auf Dauer dennoch, so betreibt er ein touristische Gewerbe und muss dieses grundsätzlich anmelden. Die Beantragung eines Gewerbescheins ist einmalig mit Kosten verbunden, die bei der langfristigen Erzielung von Einnahmen aber relativ gering ist\footnote{Je nach Ort zwischen 15 und 65 Euro http://www.gewerbe-anmelden.info/gewerbeschein/kosten.html}. Auch die Tatsache, das sich in der Regel nicht damit rechnen lässt, dass permanent Anfragen am Grundstück bestehen, stellt den Vermieter nicht vor weitere Kosten. Die Einnahmen müssen zu jeweiligen Abgaben beim Finanzamt, wie normales Einkommen, angegeben werden, die Höhe dürfte jedoch im Normalfall zu gering sein um relevante Folgen zu haben.\\

Der Punkt, an dem die Vermietung wirklich als \textit{nachhaltig} angesehen werden (und als Gewerbe angemeldet werden muss), ist dabei auch nicht genau definiert, Mindesteinnahmegrenzen schwanken bei der Recherche  von Quelle zu Quelle. Dieser Umstand sollte dem Vermieter bei einer finanziellen Motivation jedoch deutlich sein und er muss dementsprechend seinen Nutzen abwägen und auf erforderliche Vorraussetzungen achten.\footnote{Informationen http://www.gewerbe-anmelden.info/, http://www.steuerberaten.de/tag/nebengewerbe/ und http://portal.wko.at/wk/format\_detail.wk?angid=1\&stid=420794\&dstid=0}\\

Dieser Punkt ist aber nicht nur in diesem Konzept ein Risiko, sondern tritt auch bei entsprechenden Konkurrenzprodukten auf. Couchsurfing und Freagle, die das ganze auf sozialer Ebene austragen und vom System aus keine Kosten dafür verlangen, können nicht kontrollieren, dass dennoch eine Bezahlung stattfindet. Beim Couchsurfing ist es beispielsweise ein Teil der Kultur geworden, dem Gast ein Geschenk mitzubringen oder sich auf andere Weise bei ihm zu Bedanken. Airbnb und Campinmygarden, werden grundsätzlich vor dieses Problem gestellt und die Nachfrage zeigt, dass auch die Vermieter dazu bereits sind auf diese Kompromisse einzugehen. 




