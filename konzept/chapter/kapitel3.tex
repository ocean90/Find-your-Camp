%!TEX root = ../konzept.tex

\chapter{MCI Anforderungsermittlung}
Wesentlicher Schwerpunkt des Projektes, ist die Auseinandersetzung mit Methoden der Mensch Computer Interaktion. Dieser Bereich  setzt sich aus vielen interdisziplinären Fachrichtungen zusammen und bietet eine Vielzahl unterschiedlicher Vorgehensmodelle. 
Relevanz für das Projekt, haben Modelle, die sich verstärkt mit der Entwicklung menschzentrierter Software auseinandersetzen.
Nach der ISO 9241 im Teil 210 wird folgendes Modell des User Centered Design spezifiziert\footnote{http://blog.procontext.com/images/posts2010/prozessergebnisse-des-usability-engineering-1200x900.png}. 

\begin{figure}[H]
\includegraphics[width=.9\textwidth]{./images/prozessergebnisse.png}
\caption{Prozessmodell menschzentrierter Gestaltung nach ISO 9241-210 }
\label{prozessmodell}
\end{figure}

Wie bereits der Bezeichnung zu entnehmen ist, konzentriert sich dieses Vorgehensmodell auf die Zentrierung des Users. Für identifzierte Problemdomäne wäre eine dahingehende Untersuchung sinnvoll, um eine  Auseinandersetzung mit den Anwendern zu initiieren und ein System zu beschreiben, dass den Fokus auf die Unterstützung ihrer speziellen Eigenschaften legt. Auch der iterative Prozesscharakter ist eine der Stärken und soll für dieses Entwicklungsprojekt betrachtet werden.\\
Eine weiterer Ansatz der MCI, geprägt durch Larry Constantine und Lucy Lockwood ist das Usage centered Design. Dieses Vorgehensmodell legt dabei den Fokus auf die Interaktion zwischen den Anwendern und der funktionalen Benutzung. \\
Nach Betrachtung der geplanten Anwendung, wurde daher entschlossen mit Methoden des Usage centered Design zu arbeiten, da der Problemfokus nicht direkt auf den Benutzern liegt, sondern vorallem auf den Aktivitäten die diese durchführen um ihr Ziel zu erreichen. Anfänglich soll demnach eine Kombination beider Vorgehensmodelle verwendet werden, sodass eine grundliche Auseinandersetzung mit den möglichen Anwendern Anhaltspunkte für die Interaktionen liefern kann. Stakeholdermodellierung und die Erstellung von User Profiles identifizierter Gruppen wäre hierbei ein erster Schritt. Das User centered Design würde in einem vertiefenden Schritt eine Konkretisierung hinsichtlich Personae vorsehen und für diese Scenarien entwickeln. Als Projektrelevant werden jedoch aus genanntem Interaktionsfokus abstraktere Modelle gesehen.

TODO


\newpage

%!TEX root = ../konzept.tex

\section{Benutzermodellierung}
Die erste Stufe der Benutzermodellierung dient der Identifikation potentieller Nutzergruppen des Systems.
Das Spektrum an Interessenten und Betroffenen innerhalb der Anwendungsdomäne, wird dazu in
verschiedene Benutzerklassen eingeteilt. Es werden alle Stakeholder gesucht, die ein Anrecht, Anspruch oder Interesse am System haben und in Zukunft entwickeln könnten. Zudem wurde ein erster Blick auf deren Charakteristiken gewagt und der Einstieg zu entsprechenden User Profiles geschaffen.\footnote{Nach ISO 9241-210, in Prof. Hartmann Draftzur MCI ab S. 546}\\
 
Als primary User des Systems, letztendlich die Enduser, die direkt mit der Anwendung interagieren, wurden folgende Benutzer ermittelt:
 
Zum einen wird die Anwendung von Interessenten in der Tätigkeit des \textbf{Mietenden} verwendet.
Diese können grundsätzlich aus allen Altersgruppen stammen, vorausgesetzt sie sind voll geschäftsfähig, da ein Mietgeschäft stattfindet, und sie sind im Besitz der benötigten Hardware.
Eingrenzen lässt sich das Benutzerfeld weiterhin auf Reisende oder Touristen, die eine Unterkunft benötigen und im Regelfall Campingausrüstung bei sich tragen. In diese Kategorie fallen demnach auch Personen wie Backpacker, Langzeitreisende, Wanderer.
Sie suchen entweder aktiv selbst oder sind diejenigen, die mit der Funktionalität im späteren Verlauf, beispielsweise bei einem QR Code Check\footnote{Ob diese Funktion im letztendlichen System so anwendung findet ist nicht festgelegt.}, direkt in Kontakt mit der Anwendung kommen.
 Die zweite Interessentengruppe sind die \textbf{Vermieter}, die ihr Grundstück als Unterkunft zur Verfügung stellen. Auch hier wirkt vorherige Einstufung der Altersgruppe, wobei die Verfügbarkeit der Hardware in der Regel zu einer gewissen Altersgrenze führen wird, die Schätzungsweise bis 50 Jahre geht.
Vermieter in der Rolle des primary users sind Grundstücksbesitzer bzw. Garteneigentümer und Personen, welche während des Mietprozesses aktiv mit der Anwendung interagieren.\\
 
Weitere Betrachtung widmet sich den secondary usern, welche nicht regelmäßig selbst mit der Applikation interagieren. Sie liefern den primary usern entsprechenden Input, diese benutzen die Anwendung dann an ihrer Stelle als \textbf{Zwischennutzer} und geben den erhaltenen Output zurück. Für diese Benutzergruppe wurde auf Mieterseite die \textbf{„erweiterte“ Reisegruppe} definiert. Jeder der neben dem eigentlichen Benutzer die Unterkunft benutzt und im Vorfeld aktiv an der Suche beteiligt ist, indem beispielsweise Anhaltspunkte gegeben werden in welchem Umkreis gesucht werden soll.
Für den Vermieter ergibt sich hierbei der Personenkreis, der selbst ein Grundstück zur Verfügung stellen kann und möchte, aber entsprechende Technologien nicht besitzt und das Angebot jemand anderen Abwickeln lässt. Die Anzahl dieser Anwendertypen wird im Vergleich zu den vorherigen Benutzergruppen jedoch eher gering eingeschätzt.\\
 
Als weitere Stakeholder im Bereich der Entscheidungsträger für spätere Anschaffung und Benutzung (tertiary user) wurden \textbf{potentielle Werbepartner} identifiziert, die als Teil des Geschäftsmodells zum Beispiel bei Events auftreten können. Zusätzlichen Einfluss kann die Anwendung außerdem auf die \textbf{Stadt} in Form von Tourismus haben und dabei auch mit \textbf{bestehenden Unterkunftsanbietern} wie Hotels, Hostels oder öffentlichen Campingplätzen konkurrieren. Auch wenn diese nicht zwangsläufig als Nutzer der Anwendung auftreten, so haben sie Interesse am möglichen Erfolg und Misserfolg und können dementsprechend beeinflusst werden.
Dazu kommen \textbf{Personen aus organisatorischen Bereichen} wie System Administratoren oder Supportmitarbeiter.\\
 
 Zusammenfassend ergeben sich damit folgende Stakeholdergruppen:

 \begin{itemize}
   \item
   Mieter: Reisende als Endnutzer und gesamte Reisegruppe

   \item 
   Vermieter: Grundstücksbesitzer und angehörige Familienmitglieder

   \item
   potentielle Interessenten: Werbepartner, Unternehmen, Stadt

   \item
   Systemadministratoren, Support\\
\end{itemize} 

Diese Stakeholder stehen dabei in einem bestimmten Verhältnis zum System. In welchem Objektbereich dies geschieht, wird in folgender Tabelle (Tab.\ref{tab:stakeholderidentifikation}) dargestellt.


\begin{table}[H]
\caption{Identifizierte Stakeholder und ihr Objektbereich}

\centering
\begin{tabular}{l l l}
\\ [-0.5ex]

\hline\hline
\\ [-0.5ex]
Personengruppe & Bezugsbereich & Objektbereich
\\ [1ex]
\hline
\\ [-0.5ex]
externe Stakeholder  & &\\[1ex]
\textbf{Mieter} & Anrecht & Bei Bezahlung der Applikation oder\\[0.5ex]
               & & möglichen Funktionen wie werbefreiheit,\\[0.5ex]
               & & hat der Mieter ein recht darauf,\\[0.5ex]
               & & die bezahlte Leistung zu erhalten. \\[0.5ex]
               & & + Rechtlich durch Datenschutzgesetz. \\[1ex]



      & Interesse & Verfolgt Projekt als möglicher Nutzer mit, \\[0.5ex]
                & & hat Interesse an der Applikation \\[0.5ex]
                & & für späterer Benutzung\\[1ex]

      & Anspruch  & Stellt funktionale Ideen vorschlagen, \\[0.5ex]
                & & hat keinen rechtlichen Anspruch\\[1ex]


\textbf{Vermieter} & Anrecht & wie beim Mieter \\[0.5ex]
          & Interesse & wie beim Mieter \\[0.5ex]
          & Anspruch & wie beim Mieter \\[1ex]

\textbf{Kooperationspartner} & Interesse & Verfolgt Entwicklung des Projektes,\\[0.5ex]
                               & & auch hinsichtlich der Anwenderzahl\\[0.5ex]
                               & & für potentielle Kooperationen\\[1ex]

                     & Anrecht & Als vertraglicher Werbepartner durch \\[0.5ex]
                              & & Einschaltung von Werbung \\[0.5ex]
                              & & oder bei Eventkooperationen\\[1ex]
                      & Anspruch  & kann funktionale Ansprüche vorschlagen, \\[1ex]
                              & & jedoch rechtlich nicht bindend\\[1ex]


\textbf{Unterkunftsanbieter} & Interesse & Verfolgt Projekt aufgrund Konkurrenz \\[1ex]

\textbf{Stadt} & Interesse & Verfolgt Projekt durch Tourismusfaktor \\[1ex]
 & Anrecht & Als Finanzierungspartner, Unterstützer \\[1ex]

interne Stakeholder & &\\[1ex]
\textbf{Mitarbeiter (Support)} & Anrecht & Arbeitsvertrag \\[1ex]

\textbf{Entwickler} & Anrecht & Arbeitsvertrag \\[1ex]
\textbf{Eigentümer} & Anteil & Besitzer des Systems \\[1ex]

\hline
\end{tabular}
\label{tab:stakeholderidentifikation}
\end{table}


Für die spätere Auseinandersetzung ist es notwendig die für das Projekt wichtigsten Stakeholdergruppen zu bestimmen und dahingehend die Anforderungsermittlung durchzuführen.
In Anbetracht des Projektziels fallen hierbei vorerst die primary und secondary user auf, die in einer detaillierten Auseinandersetzung innerhalb der User Profiles genauer bestimmt werden sollen.
Bei der Charakterisierung ihrer Merkmale sollte dabei auf sinnvolle Eigenschaften geachtet werden, die für das System von Relevanz sind. Eine Herausforderung dabei wird auch das Abwägung der Prioritäten und das Eingrenzen der Individuen in gemeinsame Eigenschaften. Die Darstellung soll dabei möglichst abstrakt sein und in einer neutralen Form aufbereitet werden.
 
Da die menschzentrierte Entwicklung ein iterativer Prozess ist, kann zu späteren momenten eine erneute Auseinandersetzung mit den gewonnen Stakeholdern stattfinden.
Ansätze hierbei wäre ein Perspektivwechsel nach Tätigkeitsperspektive, Rollenperspektive, Interessenperspektive oder kulturelle Perspektive sowie Ergebnisse aus weiterer Marktanalyse und Recherche. Eventuell auch unter Einbezug möglicher Interessenten. (Interviews mit echten Campern etc.) Für den Fall, dass im Projektverlauf Probleme bei der Interaktionsmodellierung auftreten, wäre eine erneute Auseinandersetzung denkbar. Wie erwähnt soll der Fokus  jedoch verstärkt auf der eigentlichen Benutzung, anstelle der Nutzergruppen liegen.\\
 





\newpage

%!TEX root = ../dokumentation.tex

\section{Vertiefende Anforderungsanalyse}


\newpage

%!TEX root = ../konzept.tex

\section{Nutzungskontext}
Nach der Betrachtung der vorhandenen Benutzer, eine erste Beschäftigung mit ihre Zielen und den Aufgaben des Systems in der vorherigen Anforderderungsanalyse, sowie weiteren Informationen die in der bisherigen Ausarbeitung hin und wieder angesprochen wurden, folgt ein Grobriss des Nutzungskontext\footnote{Nach Inhalt der ISO 9241-210, Prof. Hartmann Draft zur MCI ab S. 544}.\\
Bereits verdeutlichte Bereiche sollen hierbei nicht genauer wiederholt werden, sondern es geht vorerst um die Ergänzung zusätzlicher Überlegungen.\\


Benutzer
\begin{itemize}
   \item 
   Altersklassen: Nicht exakt definierbar, grundsätzlich jeder der ein Smartphone besitzt und auf diese Art und Weise Reisen will. Kernzielgruppe wird zwischen 16 - 50 Jahre geschätzt. Prinzipiell könnten auch Personen darüber hinaus Interesse an der Applikation haben. Vorraussetzung ist lediglich die vorhandene Technologie. Vermutlich spricht der sozialere Aspekt diese aber eher weniger an und sind wahrscheinlich nicht mit Campingausrüstung längere Zeit unterwegs. Physische Merkmale befähigen sie als Mieter dazu, eine Reise aufzunehmen.

   \item 
   Einkommen: Grundsätzlich alle Einkommensklassen möglich, aufgrund des Kostenfaktors jedoch hauptsächlich Einkommensschwache oder eine besonders finanzbewusste Zielgruppe.

   \item 
   Erfahrung der Benutzer: Mit Erfahrung der Campingdomäne kann gerechnet werden, muss aber nicht zwangsläufig.
Erfahrene Leute in diesem Bereich, haben eventuell schon einige Apps ausprobiert und in Benutzung und können mit solchen System sicher umgehen.
   Grundsätzlich wird davon ausgegangen, dass die Stakeholder mit einem Smartphone umgehen können, aber eventuell zum ersten Mal auf diesem Weg Reservierungen und Bezahlungen durchführen. 

   \item
   Fähigkeiten: Sprachkenntnisse können variieren, da auch ausländische Touristen die Software benutzen könnten.

    \item 
    Einstellung der Benutzer: In der Regel sozial offenere Menschen, eventuell naturgebundene Leute die auf der Durchreise sind bzw. an kürzeren Aufenthalten interesse haben. 

   
\end{itemize}

\newpage


Aufgaben und Ziele
\begin{itemize}
   \item 
   Mieter: Finden eines geeigneten Schlafplatzes mit Hilfe des Smartphones als Unterkunft. Dabei die Suchanfrage von Benutzerseite ausgehend und die Anfrageverarbeitung auf Seite des Systems verarbeiten. 
   \item 
   Vermieter: Erfolgreiches Vermieten eines Schlafplatzes. Relevante Anfragen vom System empfangen und nach eigenem Ermessen beantworten. 
   \item
   Sichere Informationsübertagung zwischen Kontaktpersonen
   \item
   Sichere Abwicklung der Bezahlung\\  


\end{itemize}


Arbeitsmittel
\begin{itemize}
   \item 
   Smartphone mit Applikation und Android Betriebssystem. (Im Sonderfall wäre eine Registrierung über Webpräsenz am Computer eine denkbare Option) 
   \item  
   Internetanbindung in Zusammenarbeit mit Telefonanbieter\\
   

\end{itemize}


physikalische Umfeld
\begin{itemize}

   \item 
   Umgebung: In der Regel im Freien, kann aber auch im Gebäuden stattfinden. Nicht vorhersehbar, ob in Großstadt oder ländlicher Gegend.

   \item
   Anbindung:  Da Wandernde auch fernab von Hauptstraßen und Orten reisen können ist mit Netzproblemen zu rechnen. Zusätzlich sollte Rücksicht auf Akkuleistung genommen werden, da die Stromversorgung selten vorhanden sein wird. 

   \item
   Gepäck: Reisende führen relativ viel Gepäck mit sich und benötigen ihre Campingausrüstung, daher in Flexibilität etwas eingeschränkt.
\end{itemize}





\newpage

%!TEX root = ../dokumentation.tex

\section{Prototyping}


\newpage

%!TEX root = ../konzept.tex

\section{Weiteres MCI Vorgehen}

Welches sind die Aufgaben, welche Struktur weisen sie auf und in welchen Beziehungen stehen die Aufgaben zueinander?

\subsection{Vorgehensmodelle}

{\color{blue}Requirement Engeneering
Stakeholder + Bedürfnisse + Nutzungskontext + task analysis (+-)
-> Problemszenarien: Aktivitäts/Interaktions/Informationsszenarien
oder
-> context models, concrete use cases
 
Analyse deskr. -> Problemszenarien: Handeln in aktueller Situation
ENtwurf presk: -> Informationsszenarien: wann werden welche Informationen in der Handlung benötigt
        	       	-> Aktivitätsszenarien: Aktionen in Handlung
                   	-> Interaktionsszenarionen: Interaktionszyklen
 }


\subsection{Dokumentationstechniken der Aufgaben}
 Szenarien, Use Cases, HTA, GOMS

{\color{blue}
 - Häufigkeitsgrad, Variationsgrad, notwendiges Wissen, Grad der Einflussnahme
        	- Task Scenario: erzählerische, aktuelle Anwendungsbeschreibung
        	- Use Scenario: narrativ, angestrebte Verwendudng des Systems basierend auf Anforderungen und zeigt wie system zu verwenden sein wird
 
koginitive, motivationale Aspekte einbezihene, Vor und Nachbedingungen
Szenario:
        	- Geschichte, Aufgaben, Aktivitäten, Interaktion
        	- konkreter Kontext
        	- Instanz des Use Cases
Pros/ COns bei Problemszenario
 
Use case: Beziehung stakeholder <-> technisches System
        	- Wie wird das System genutzt? Was machen Leute? Was macht System?
        	
strukturiert, stringenter Ansatz, tendietiell linearisierte Darstellung der Interaktion
        	- Verhalten als Antwort auf Anfrage des Primary Users
 
        	-> verschiedene Sequenzen -> verschiedene Szenarioen
 
Methode: Allistar Cockburn Template
 
Concrete Use Case
user intention/action | systems responsibility
- Interface Metaphern
- detaillierte Beschreibung, nicht personalisiert, generisch
essential use case
        	- strukturiert, narrativ, domain und user sprache, abstrakt, technologiefrei, implementations unabhängig, neutrale Konnotation
        	- beschreibt Aufgabe mit hohem Abstraktionsgrad ohne Interaktionsparadigmen
        	- Fokus was will user machen/ System verantwortung
-> Ermittlung/Validierung der Anforderungen
- funktionale Anforderungen, Interessen Stakeholder, Systemverhalten, Detaillierte Informationen zu Fehlerkonditionen und Behandlungen
 
HTA:    + kogintive Perspektive (als einzige Methode)
        	+ Wissen graphisch dargestellt
        	+ geringe Wissensbarriere
        	+ Skalierbarkeit
- Analyseaspekt, Arbeit Modellhaft repräsentieren
- Relation Aktivität <-> kogn. Perspektive
- Aufgabe hinsichtlich Struktur analysieren (aus koginitiver Perspektive)
- Dekomposition: + goals/ Tasks/ operations
Arbeitsschritte:
1. Zielsetzung Analysieren
2. Konsens Stakeholder, goals
3. Methoden/ Quellauswahl
4. erste Dekomposition in Diagram/ Tabellenform


 
Welches ist das konzeptuelle Modell für das präskriptive Aufgabenmodell? Welche Bezüge existieren bzgl. der Entwicklungsziele?
präskriptives Modell: “Soll” Zustand
working engeneerings
descriptive -> claim analysis -> prescriptive
 	Welche alternativen konzeptuellen Modelle wurden entwickelt und wie wurde mit Design-Alternativen verfahren?

}
 


\newpage
