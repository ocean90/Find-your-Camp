%!TEX root = ../konzept.tex

\section{Kommunikationsabläufe und Interaktionen}
Zur Festlegung der Systemarchitektur werden zunächst die Anforderungen anhand eines möglichen Kommunikationsablaufs eines erfolgreichen Interaktionsszenarios in einer klassischen Server-Client-Architektur betrachtet (Abb. \ref{fig:kommunikationsablauf}).\\

Ein potentieller Mieter nutzt die Applikation zunächst, um seinen eigenen Standort zu ermitteln. Daraufhin sendet er seine Anfrage an das Hauptsystem, welches das erste \textit{Matching} durchführt.

In einer Datenbank befinden sich Zuordnungen zwischen Vermieter und dem Ort eines Mietobjektes. Auf Basis dieser Datenmenge wird ein Abgleich zwischen angefragtem Ort und vorhandenen Orten mit eingetragenen Grundstücken durchgeführt. Bei allen möglichen Treffern wird der zugehörige Nutzer herausgefiltert.

Zu den relevanten Daten eines Nutzers gehört unter anderem eine Identifikationsnummer, unter welcher das Smartphone des Nutzers ansprechbar ist.
Dadurch werden die nun gefilterten Vermieter angesprochen und bekommen ein Ereignis zugesendet.

Das Ereignis beinhaltet die relevanten Daten der Anfrage durch den Mieter. Sobald das Ereignis das Ziel erreicht hat, wird auf dem Endgerät des Vermieters das zweite \textit{Matching} durchgeführt, denn wie in den Alleinstellungsmerkmalen bereits hervorgehoben wurde, werden direkte Personen- und Objektdaten auf den Endgeräten der Nutzer gespeichert.

Das zweite \textit{Matching} vergleicht das gesendete Profil mit dem auf dem Gerät gespeicherten Profil. Ein Algorithmus berechnet dabei einen Quotienten, welcher den Grad der Übereinstimmung widerspiegelt. Übersteigt dieser Quotient einen Wert X, wird der Vermieter über die Anfrage visuell benachrichtigt. Anschließend sollte er darauf reagieren und die Anfrage ablehnen oder annehmen. Bei einer Bestätigung wird der Mieter über die Annahme informiert und kann die Daten des potentiellen Mietobjektes einsehen, sowie weiteren Kontakt zu dem Vermieter aufnehmen.

\begin{figure}[H]
\includegraphics[width=.9\textwidth]{./images/kommunikationsablauf.png}
\caption{Kommunikationsablauf eines erfolgreichen Interaktionsszenario }
\label{fig:kommunikationsablauf}
\end{figure}

Dieses Beispiel zeigt die die Interaktion zwischen zwei Benutzern während einer Mietanfrage. Zusätzliche Funktionalität die an diesem Schritt anschließt, ist die abschließende Bezahlung am Ende eines Mietvorgangs.

Auf Basis dieses Kommunikationsablaufs kann nun auf die einzelnen Komponenten der Systemarchitektur eingegangen werden.

