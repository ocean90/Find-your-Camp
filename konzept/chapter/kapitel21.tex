%!TEX root = ../konzept.tex

\section{Problemstellung}
Reisende, die während ihres Aufenthalts an mehreren Orten Halt machen, werden oftmals vor dem Problem gestellt, eine geeignete Unterkunft für die Nacht zu finden. Vorhandene Systeme ermöglichen bereits eine zufriedenstellende Suche von Alternativen und gewährleisten eine Übersicht eingetragener Hostels, Campingplätze oder Hotels. Im Gegenzug dafür, fordern sie jedoch vom Reisenden viele Gegenleistungen, die ihn in seiner Flexibilität und Reisefreiheit einschränken.\\
Wesentliche Punkte die bei dieser „herkömmlichen Art des Reisens“ beachtet werden müssen sind folgende:
\begin{itemize}
   \item Frühzeitige Planung: Gerade wenn man einen Urlaub während der Hochsaison plant, ist es notwendig bereits im Vorfeld für entsprechende Reservierungen zu sorgen. Alternativ wird man vor noch höhren Kosten gestellt oder findet keine gewünschte Unterkunft. 
   \item Geringe Flexibilität: Eingegangene Reservierungen sind weitestgehend verbindlich. Die Reiseplanung wird damit festgesetzt und ermöglicht keine kurzfristigere Änderung. Spontanität, Variation der Reiseroute und Zeitplan sind damit nicht möglich.
   \item Kosten: Gängige Unterkünfte stellen eine sichere Option dar, sind für viele Reisende jedoch auch ein entsprechendes Risiko und ein hoher Kostenfaktor. Dieser kann sowohl Einfluss auf die Reisedauer als auch den Reiseort haben. Oftmals wird dabei auf ein günstigeren Kompromiss eingegangen. 

\end{itemize}

TODO


\subsection{Shared Economy}
Die Entwicklung der vergangen Jahre zeigte beispielsweise am Couchsurfing (eventuell kurze Erklärung + Statistiken), dass es eine Nachfrage an alternativen Unterkünften gibt. Diese liegt grundsätzlich auf sozialen Aspekten sowie dem Kostenfaktor begründet und erhalten unter den Teilnehmenden großen Zuspruch. Als weiteres relevantes Beispiel wäre Airbnb zu nennen, die einen ähnlichen Ansatz verfolgen, indem sie private Unterkünfte zur Verfügung stellen, dabei jedoch eine entsprechende finanzielle Gegenleistung einfordern, die im Preissegment öffentlicher Angebote spielen.\\
(TODO zu Airbnb).\\

Übergang zu Shared Economy\\

{\color{blue}
Teilen ist das neue Besitzen. Du bist Mitglied einer starken Community, die sich gegen übermäßigen Ressourcenverbrauch und die vorherrschende Wegwerfmentalität einsetzt.\\

Garden Sharing
http://en.wikipedia.org/wiki/Garden\_sharing
http://www.garten-sharing.de/
Garten sharen
Gegen Miete
Ungenutze Grundstücke bereitstellen
Grundstückeigentürmer kann Kosten decken
Car Sharing
https://www.carzapp.net/
Park Sharing
http://www.parku.ch/en\_GB/
http://www.morgenpost.de/berlin-aktuell/startups/article120660878/Wie-Verbraucher-mit-Apps-nebenbei-Geld-verdienen-koennen.html
http://www.lets-share.de/thomas-doennebrink-von-ouishare-net/
Und wo funktioniert es noch gar nicht? Sharing setzt Kommunikation, Interaktion und Vertrauen voraus. Die Collaborative Economy funktioniert dort weniger, wo Menschen nicht gewohnt sind oder es verlernt haben, direkt miteinander zu kommunizieren – sowie sich gegenseitig misstrauen.  Umgekehrt gilt die Faustformel: Jüngere, umweltbewusstere und mobilere Menschen neigen eher zum Teilen. Die Treiber sind das Internet, die Medien, Smartphones und Vernetzung. Menschen, die hierzu keinen Zugang haben, werden langsamer erreicht. Förderlich sind auch Bevölkerungsdichte, Knappheit von Ressourcen oder der Umstand, wie ungenutzt Güter sind.

}



\newpage
\subsection{Weiterführende Statistik}
\textcolor{red}{Hier nochmal überlegen, ob die Inhalte dieses Unterpunktes wirklich relevant sind und zeitlich Akzeptabel.}\\


Mit der einsetzenden Finanzkrise 2007, zeigte sich im Tourismus eine Entwicklung, die sich nicht nur auf Deutschland, sondern den gesamten Europäischen Raum abzeichnen lässt. 
In folge der Krise handelten viele Urlauber, indem sie neue Möglichkeiten suchten ihre Übernachtungskosten zu reduzieren und wichen dabei auf öffentliche Campingplätze aus. 
Zeitgleich konkurrieren vorallem asiatische Länder um die Gunst der Touristen, indem sie ein Billigsegment schaffen, dass viele Interessierte in ihre Gegenden zieht. Aufgrund der Steuerbelastung sowie Personalkosten etc. fällt es vielen europäischen Ländern schwer mit diesem Trend mitzuhalten.\\
Als alternative gegenüber Hotelübernachtungen zeigen Statistiken, die von der European Commision erhoben und veröffentlicht wurden, eine Entwicklung der Campingplatzbesucher der letzten Jahre.  Während grundsätzlich in Skandinavien, sowie den Küsten und Bergregionen Europas großer Zuspruch herrscht, so lassen sich auch in Deutschland Gebiete identifizieren, die einen jährlichen Zuwachs von 10\% an Campingplatzbesucher gewinnen.\\

Quelle: http://epp.eurostat.ec.europa.eu/statistics\_explained/index.php/Tourism\_statistics\_at\_regional\_level/de 
Sichtdatum 20.10.2013




