%!TEX root = ../konzept.tex

\section{Proof-of-Concepts}
Die nächste Projektphase, die den Übergang von der bisher theoretischen Auseinandersetzung zur praktischen Umsetzung markiert, befasst sich mit der prototypischen Realisierung des geplanten Vorhabens. Dabei wird auf die als wesentlich erachteten  Funktionalitäten eingegangen und diese testweise umgesetzt.\\
Die gewünschten Ergebnisse dienen anschließend dazu, die Möglichkeit der prinzipiellen Umsetzung zu validieren und sind für den weiteren Projektverlauf von essentieller Bedeutung. Es sollen frühzeitig aufkommende Risiken und Schwierigkeiten identifiziert  und im Idealfall bewältigt werden.

\vspace{0.5cm}

Entsprechend der im Vorfeld geplanten technischen Realisierung, ergeben sich für das Projekt damit folgende Schwerpunkte die im Proof-of-Concepts beachtet werden sollten:\\
\begin{itemize}
   \item \textbf{PoC 1: „Hello World“ Android-App} \\
   Für den ersten Kontakt mit Android und der zugehörigen Entwicklungsumbegung.
   \item \textbf{PoC 2: GCM HTTP Connection Server} \\
   Realisierung einer Anbindung zu GCM über die HTTP Variante.
   \item \textbf{PoC 3: GCM Cloud Connection Server} \\
   Realisierung einer Anbindung zu GCM über die CCS Variante.
   \item \textbf{PoC 4: Lokale Datenbankanbindung} \\
   Realisierung einer Anwendung zur Speicherung von Nutzerdaten auf dem lokalen Endgerät.
   \item \textbf{PoC 5: Datenfreigabe über GCM} \\
   Aufbauend auf PoC 2/3: Übertragung von Vermieterdaten an Mieter.
\end{itemize}
