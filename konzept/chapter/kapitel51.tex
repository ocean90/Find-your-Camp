%!TEX root = ../konzept.tex

\section{Proof-of-Concepts}

TODO Fallbacks einbauen?
Die nächste Projektphase, die den Übergang von der bisher theoretischen Auseinandersetzung zur praktischen Umsetzung markiert, befasst sich mit der prototypischen Realisierung des geplanten Vorhabens. Dabei wird auf die als wesentlich erachteten  Funktionalitäten eingegangen und diese testweise umgesetzt.\\
Die gewünschten Ergebnisse dienen anschließend dazu, die Möglichkeit der prinzipiellen Umsetzung zu validieren und sind für den weiteren Projektverlauf von essentieller Bedeutung. Es sollen frühzeitig aufkommende Risiken und Schwierigkeiten identifiziert  und im Idealfall bewältigt werden.

\vspace{0.5cm}

Entsprechend der im Vorfeld geplanten technischen Realisierung, ergeben sich für das Projekt damit folgende Schwerpunkte die im Proof-of-Concepts beachtet werden sollten:\\
\begin{itemize}
   \item \textbf{PoC 1: „Hello World“ Android-App} \\
   Der erste PoC dient der ersten Auseiandersetzung mit Android und der entsprechenden Entwicklungsumgebung. Sinnvollerweise sollte dieser Schritt direkt am Anfang durchgeführt werden, um für das Projekt frühzeitig mögliche Probleme und Eigenheiten der Entwicklung zu erkennen.

   \item \textbf{PoC 2: GCM HTTP Connection Server} \\
   Realisierung einer Anbindung zu GCM über die HTTP Variante. Das Architekturmodell sieht eine Kommunikation mit GCM vor. Eine der möglichen Varianten läuft dabei über HTTP, sieht aber keinen Upstream zur Cloud vor. Prinzipiell wird die CSS Variante (PoC 3) vorgezogen. Da diese jedoch ohne Freischaltung nicht zugänglich ist, soll ein erster Test über diese Verbindungsart einen möglichen Fallback verringern.

   \item \textbf{PoC 3: GCM Cloud Connection Server} \\
   Realisierung einer Anbindung zu GCM über die CCS Variante. Bei dieser Variante handelt es sich um ein relativ neues Prinzip von Google und ist daher noch nciht für jeden zugänglich. Die Up- und Downstream Eigenschaft verspricht im Bezug auf die entworfene Systemarchitektur, den größeren Nutzen gegenüber PoC 2. 
   Dieser PoC hängt dabei jedoch deutlich von der Freischaltung seitens Google ab. Im Idealfall kann dieser PoC den 2. komplett ersetzen, sollte frühzeitig die Möglichkeit gegeben werden damit zu arbeiten.
   Beide Versuche befassen sich hierbei mit dem Kommunikationsaufbau zwischen Mieter und Vermieter und stellen daher eine grundlegenden Minimalfunktionalität des Projektes dar.

   \item \textbf{PoC 4: Lokale Datenbankanbindung} \\
   Realisierung einer Anwendung zur Speicherung von Nutzerdaten auf dem lokalen Endgerät.
   Neben der Kommunikation über GCM spielt weiterhin die lokale Datenspeicherung eine wichtige Rolle. Speziell für das Projekt soll eine lokale Datenbank eine erhöhte Sicherheit der sensiblen Benutzerdaten gewährleisten. Somit steht auch hierbei das Alleinstellungsmerkmal sowie die Kernfunktionalität im Vordergrund und soll getestet werden.

   \item \textbf{PoC 5: Datenfreigabe über GCM} \\
   Aufbauend auf PoC 2/3 beschäftigt sich der letzte Versuch mit der Übertragung der Benutzerdaten vom Vermieter zum Mieter. Dabei wird auf die Daten der lokalen Datenanbindung zugegriffen und über eine GCM Verbindung relevante Userdaten übertragen. Innerhalb des Interaktionsablauf stellt dieser Schritt eine Antwort zur Mietanfrage dar.
   
\end{itemize}
