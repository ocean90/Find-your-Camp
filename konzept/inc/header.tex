%!TEX root = ../konzept.tex

%Da Latex für englischsprachige Texte ausgerichtet ist,
%wird als Dokumentenklasse das "`scrbook"' von Markus Kohm verwendet.
%Dieses ist für deutschsprachige Texte ausgelegt.
%BCOR12mm: 12mm Bindekorrektur (Verbreiterung des linken Randes)
%DIV11: entspricht in etwas der geforderten Textgröße und Seitenränder
%titlepage: eine Titelseite wird verwendet
%a4paper: DIN A4
%oneside: für eine spätere einseitige Bedruckung
%Original: \documentclass[BCOR12mm,DIV11,titlepage,a4paper,oneside]{scrbook}
\documentclass[DIV11,titlepage,a4paper,oneside]{scrbook}

\textheight = 23cm

%Paket für deutsche Silbentrennung etc.
\usepackage{ngerman}

%Paket für Zeichenkodierung, entspricht UTF-8
\usepackage[utf8x]{inputenc}

%Paket das die Ausgabefonts definiert
\usepackage[T1]{fontenc}

%Paket für das Einbinden von Grafiken über die figure-Umgebung
\usepackage{graphicx}

%Paket zum Ändern der Listenabstände
\usepackage{enumitem}
\setlist[itemize]{itemsep=0mm}
\setlist[enumerate]{itemsep=0mm}

%Paket zum Ändern der Kopf- und Fußzeile
\usepackage{fancyhdr}

%fancypagestyle{fancy}
\renewcommand{\headrulewidth}{0pt}
\fancyhf{}
%\fancyfoot[L]{FANCY}
\fancyfoot[R]{\thepage}

\fancypagestyle{plain}{
	\renewcommand{\headrulewidth}{0pt}
	\fancyhf{}
%	\fancyfoot[L]{PLAIN}
	\fancyfoot[R]{\thepage}
}

\pagestyle{plain}

%Abbildungsnummerierung ändern (abhängig von chapter, z.B. Abbildung 1.1)
\renewcommand*{\thefigure}{\thechapter.\arabic{figure}}
%Tabellennummerierung ändern (abhängig von chapter, z.B. Tabelle 1.1)
\renewcommand*{\thetable}{\thechapter.\arabic{table}}

%Paket, um ein Glossar/Abkürzungsverzeichnis anzulegen
\usepackage{nomencl}
\let\abbrev\nomenclature
%Der Name wird in Glossar geändert
\renewcommand{\nomname}{Glossar (optional)}
%Definiert die Aufteilung im Glossar zwischen Begriffen und Erläuterung
\setlength{\nomlabelwidth}{.25\hsize}
%Definiert die Punktelinien im Glossar
\renewcommand{\nomlabel}[1]{#1 \dotfill}
\setlength{\nomitemsep}{-\parsep}
%Veranlasst die Erstellung des Glossars
\makenomenclature

%Einrückungen nach Absätzen und Grafiken verhindern
\setlength{\parindent}{0pt}

%Verhindern, dass eine neue Seite für ein einzelnes Wort/Zeile verwendet wird
\clubpenalty = 10000 % schliesst Schusterjungen aus
\widowpenalty = 10000 % schliesst Hurenkinder aus (keine Beleidigung, sondern wirklich ein Fachbegriff)

%Paket für ein deutsches Literaturverzeichnis
\usepackage{bibgerm}

%Paket für die Verwendung von URLs durch den Befehl \url{}
\usepackage{url}

%Paket für Zeilenabstand (onehalfspace, singlespace)
\usepackage{setspace}

%Paket zur Erzeugung von Anführungszeichen durch \enquote{Text}
\usepackage[ngerman]{babel}
\usepackage[babel, german=quotes]{csquotes}

%Paket für farbigen Text
%black,white,green,red,blue,yellow,cyan,magenta
\usepackage{color}

%Paket für farbigen Hintergrund für Verbatim-Umgebung (Quelltext-Umgebung)
\usepackage{fancyvrb}
\usepackage{verbatim,moreverb}

%Grauton für Quelltext-Umgebung definieren 80% Grau
\definecolor{source}{gray}{0.95}

%Paket für Quelltext-Umgebung
\usepackage{listings}

%Paket für Positionierung der Objekte ohne Float (Verwendungsbsp.: \begin{figure}[H])
\usepackage{float}

%Paket für Tabellen über die gesamte Breite
\usepackage{tabularx}

%Paket für Definitionen, etc.
\usepackage{amsthm}

%Paket zur Erzeugung von Hyperrefs und PDF Informationen
\usepackage[pdftex,plainpages=false,pdfpagelabels,
            pdftitle={Projektdokumentation},
            pdfauthor={Dennis Meyer, Dominik Schilling}
            ]{hyperref}


% ###################
% Eigene Änderungen

\usepackage{subfig}
\usepackage{wrapfig}

% Codestyling
\definecolor{darkgreen}{rgb}{0,0.5,0}
\lstset{
    breaklines		= true,
    numbers			= left,
    stepnumber		= 1,
    basicstyle		= \ttfamily,
    numberstyle		= \footnotesize\ttfamily,
    backgroundcolor	= \color{source},
    language		= java,
    commentstyle    = \color{darkgreen}\ttfamily,
    keywordstyle    = \color{blue}\ttfamily,
    stringstyle		= \color{red},
    showspaces      = false,
    showstringspaces= false,
    captionpos		= b,
    frame			= single,
    aboveskip		= 1cm,
    belowskip		= 1cm,
}
\renewcommand\lstlistingname{Code}
\renewcommand\lstlistlistingname{Codeverzeichnis}
