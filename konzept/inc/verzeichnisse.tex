%!TEX root = ../konzept.tex

  %Erzeugt ein Abbildungsverzeichnis
	\listoffigures
	%Fügt die Zeile "`Abbildungsverzeichnis"' als Chapter ins Inhaltsverzeichnis ein
	\addcontentsline{toc}{chapter}{Abbildungsverzeichnis}

	%Erzeugt ein Tabellenverzeichnis
	\listoftables
	%Fügt die Zeile "`Tabellenverzeichnis"' als Chapter ins Inhaltsverzeichnis ein
	\addcontentsline{toc}{chapter}{Tabellenverzeichnis}

	%Erzeugt ein Codeverzeichnis
	%\lstlistoflistings
	%Fügt die Zeile "`Codeverzeichnis"' als Chapter ins Inhaltsverzeichnis ein
	%\addcontentsline{toc}{chapter}{Codeverzeichnis}

	%Erzeugt ein Glossar
%	\printnomenclature
	%Fügt die Zeile "`Glossar"' als Chapter ins Inhaltsverzeichnis ein
%	\addcontentsline{toc}{chapter}{Glossar}

	%Ändert den Stil des Literaturverzeichnisses
	\bibliographystyle{geralpha}
	%Erzeugt das Literaturverzeichnis anhand der Datei "`literatur.bib"'
%	\bibliography{doc/literatur}
	%Fügt die Zeile "`Literaturverzeichnis"' als Chapter ins Inhaltsverzeichnis ein
%	\addcontentsline{toc}{chapter}{Literaturverzeichnis}
