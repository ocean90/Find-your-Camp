 %================================================================================
% Erstellt am: 		10.07.2008
% Überarbeitet am:	08.07.2009
% Autor:			Holger Fischer
%
% Kann frei für Bachelor-/Diplom-/Masterarbeiten verwendet werden.
% Viel Erfolg!!!
%================================================================================

%Einbinden der Datei header.tex; diese enthält alle verwendeten Pakete,
%sowie Änderungen am Layout
\include{inc/header}

\begin{document}

%=== Einleitung ======================================================
%Seitennummerierung Abstract bis einschließlich Inhaltsverzeichnis
\frontmatter

%Einbinden der Titelseite
%!TEX root = ../konzept.tex

\begin{titlepage}

\begin{center}

%Logo der Fachhochschule Köln
\begin{figure}[!ht]
	\centering
		\includegraphics[natwidth=920pt, natheight=95pt, width=1.0\textwidth]{inc/logoheader.pdf}
\end{figure}

\vspace{4.0cm}

\begin{Huge}
	\textbf{Entwicklungsprojekt }\\
	\vspace{0.1cm}
	\textbf{interaktive Systeme}\\
	\vspace{0.1cm}
	\textbf{Konzeptentwurf}\\

\end{Huge}

\vspace{0.8cm}

\begin{LARGE}
	Konzept über ein menschzentriertes, verteiltes System\\
	\vspace{0.1cm}
	zum Mieten und Verleihen von \\
	\vspace{0.1cm}
	privaten Grundstücken als Campingplatz\\
\end{LARGE}

\vspace{2cm}

\begin{tabular}{rl}
        Dozent:  &  Prof. Dr. Kristian Fischer\\
       		 	 &  Prof. Dr. Gerhard Hartmann\\
       			 &  \small Fachhochschule Köln \\[1.0em]
      Betreuer:  &  David Bellingroth\\
				 &  Julian Rahe\\
       			 &  \small Fachhochschule Köln\\
\end{tabular}

\vspace{1.6cm}

\begin{large}
	ausgearbeitet von\\
	\vspace{0.2cm}
\end{large}

\begin{Large}
	Dennis Meyer, Matrikelnr. 11084479\\
	Dominik Schilling, Matrikelnr. 11081691\\
	\vspace{1cm}
	Wintersemester 2013
\end{Large}

\end{center}

\end{titlepage}


%Zeilenabstand für das Inhaltsverzeichnis 1 fach
\singlespacing
%Einbinden des Inhaltsverzeichnis
%!TEX root = ../konzept.tex

\cleardoublepage
\pagenumbering{gobble}
\tableofcontents
\cleardoublepage
\pagenumbering{arabic}


%Zeilenabstand für den Hauptteil ist 1,5 fach
\onehalfspacing

%=== Hauptteil =======================================================
%Seitennummerierung des Hauptteils
\mainmatter
	%Die Zähler für Tabellen und Abbildungen werden zurückgesetzt, damit
	%in jedem Kapitel die Nummerierung neu beginnt
	\setcounter{table}{1}
	\setcounter{figure}{1}
	%Einbinden des ersten Kapitels

	%Einführung
	%!TEX root = ../dokumentation.tex

\chapter{Einführung}
"Im Rahmen des Moduls Entwicklungsprojekt interaktive Systeme, geht es um die Konzipierung und Umsetzung einer verteilten, multimedialen Anwendung unter Verwendung zuvor erlernter Grundlagen. Innerhalb des Projektes sollen Methoden und Techniken der Veranstaltungen der Mensch Computer Interaktion und Web-basierte Anwendungen 2: Verteilte Systeme selbstständig geplant und durchgeführt werden."\footnote{Auszug der Konzepteinleitung, Seite 1}\\

Nach Ausarbeitung des Konzeptes innerhalb der ersten Projektphase, befasst sich diese Dokumentation mit der weiterführenden Projektarbeit bis hin zur finalen Abgabe der entstandenen Projektergebnisse.\\
Die bereits begonnene Auseinandersetzung mit wesentlichen Aspekten der Mensch Computer Interaktion und Webbasierte Anwendungen 2, wird weiter vertieft und gewonnene Ergebnisse dargelegt.\\

Die vorliegende Projektdokumentation ist strukturell in mehrere Teile aufgebaut.\\
Nach der allgemeinen Einführung folgt als erster Hauptbestandteil die Prozessdokumentation, die den zeitlichen Ablauf des Projektes fokusiert. 
Darin wird die Entwicklung nach Konzeptabgabe, einzelne Arbeitsschritte und angewendete MCI Methoden mit entstandenen Ergebnissen beschrieben.\\ Es folgt die Systemdokumentation, welche das entwickelte System genauer erklärt und die Alternativen und Abwägungen während der Entwicklungsphase herausstellt.\\ 
Der letzte Abschnitt ist die Installationsdokumentation, bei der die Anforderungen an ein System und dazugehörige Installationsschritte erläutert werden.\\

Die überarbeitete Version des Konzept befindet sich im Anhang dieses Dokuments und wird an gegebenen Stellen referenziert.



	%Die Zähler für Tabellen und Abbildungen werden zurückgesetzt, damit
	%in jedem Kapitel die Nummerierung neu beginnt
	\setcounter{table}{1}
	\setcounter{figure}{1}
 	%Einbinden des ersten Kapitels
	
	%Überarbeitungen des Konzeptes jetzt in Kapitel 3.1
	%%!TEX root = ../dokumentation.tex

\chapter{Überarbeitung des Konzeptes}





	%Die Zähler für Tabellen und Abbildungen werden zurückgesetzt, damit
	%in jedem Kapitel die Nummerierung neu beginnt
	\setcounter{table}{1}
	\setcounter{figure}{1}
	%Einbinden des zweiten Kapitels
	
	%Prozessdokumentation
	%!TEX root = ../dokumentation.tex

\chapter{Prozessdokumentation}

%!TEX root = ../dokumentation.tex

\section{Anfänglicher Projektstand}
Das \textbf{Find your Camp} Projekt, befasst sich mit der Entwicklung einer Smartphoneanwendung für Androidgeräte, als Verleihsystem von privaten Grundstücken als Unterkunft. Die erste Projektphase befasste sich dabei mit der Entwicklung eines Exposes und anschließender Konzepterarbeitung zur Ausgangsidee. Im Rahmen dieser Phase wurden vorhandene Alternativen am Markt untersucht, Alleinstellungsmerkmale und Risiken in Bezug auf die Anwendungsdomäne abgewogen und es fand eine thematische Einarbeitung in die Problemdomäne statt.\\
Der Beschäftigung mit MCI Aspekten folgte der Entschluss, das Vorgehen auf Basis des \textbf{Usage Centered Design} durchzuführen, da der Schwerpunkt des Projektes auf den Interaktionen der Anwender liegt und die erwartete Benutzergruppe nicht durch bestimmte Merkmale konkretisiert werden kann.\footnote{Konzeptseite 12}
Weiterhin fanden erste Auseinandersetzungen mit dem Nutzungskontext\footnote{ab Konzeptseite 18}, der Nutzungsmotivation und ersten Anforderungsanalysen statt.\\
Für die Umsetzung wurde eine Systemarchitektur unter Einbezug der Google Cloud angedacht und die erste Testphase über die Proof-of-Concepts\footnote{Konzeptseite 41} geplant.\\

Zur weiteren Projektephase wurden zudem 2 Meilensteine definiert. Die Bearbeitung des ersten Meilensteins, sah die Umsetzung der weiteren MCI Methoden und die Durchführung der Proof-of-Concepts vor, wie dem Auszug des Projektplans (Abb. \ref{fig:projektplan}) bei Konzeptabgabe zu entnehmen ist. Angedacht waren hierfür 3 Wochen Arbeitszeit, deren einzelnen Schritte nach und nach verfeinert werden.
\begin{figure}[H]
\includegraphics[width=1\textwidth]{./images/ausgangsplan.png}
\caption{Ausgangsplan bei Konzeptabgabe}
\label{fig:projektplan}
\end{figure}

Innerhalb welcher Zeitrahmen die einzelnen Arbeitsschritte letztendlich umgesetzt wurden, wird an gegeben Stellen deutlich gemacht.\\

\subsection{Konzeptüberarbeitungen}
Bevor mit der Dokumentation neuer Ergebnisse begonnen wurde, ging es um die Umsetzung des Konzeptfeedbacks und die Überarbeitung vorhandener Ausarbeitungen.\\
Damit die einzelnen Abschnitte inhaltlich etwas deutlicher aufeinander aufbauen, wurde die grundlegende Struktur des Dokumentes überarbeitet und das Geschäftsmodell sowie Zielhierachie als eigener Schwerpunkt herausgenommen. Die Abwägung des MCI Vorgehens wurde erweitert und an geeigneten Stellen präziser formuliert.\\

Erste Ansätze zum Geschäftsmodell wurden konkretisiert und mit Kennzahlen versehen. Die Vorgehensmöglichkeiten wurden dabei im Gespräch abgewogen und zu einem einheitlichen Modell zusammengefasst, anhand dessen eine Finanzierung als möglich erachtet wird. Die Zielhierachie wurde genauer auf das Projekt bezogen, da der vorherige Fokus zu wirtschaftlich und von langfristigen Zielen geprägt war.

\newpage
Ansätze zur Risikominimierung wurden ausführlicher und mit deutlicherem Bezug zum Konzept dokumentiert. Innherhalb der Anforderungsanalyse wurden nach eigenem Ermessen die funktionalen und qualitativen Anforderungen ausgebaut und zusätzlich erste organisatorische Anforderungen betrachtet.\\
Im WBA2 Teil, wurde das Kapitel zum Datenmodell erarbeitet und angedachte Proof-of-Concepts mit Bedeutung für das System genauer in Zusammenhang gebracht.\\

Für die Überarbeitung genannter Punkte, wurden im Vorfeld 1-2 weitere Arbeitstag eingeplant. Die letztendliche Arbeitszeit erstreckte sich jedoch auf eine gesamte Woche, was zur Folge hatte, dass der angesetzte Zeitrahmen der MCI Bearbeitungen verschoben werden musste. Der Grund der Überarbeitungsdauer liegt vor allem an der Bearbeitungstiefe, da gesamtes Konzept geprüft und wesentliche Aspekte weiter ausgearbeitet wurden. Kritisch für den Zeitplan wurde dies jedoch nicht angesehen, da die Überarbeitung auch Ergänzungen beinhaltet, die an anderer Stelle während der Dokumentation aufgetreten wären, jetzt jedoch im Konzept Einzug fanden.\\ Mit fortschreitender Projektzeit wurde deutlich, dass der erste gesetzte Meilenstein, bezogen auf die MCI Auseinandersetzung, mehr einer ersten Iterationsphase gleich kommt, als einer ausreichenden Betrachtung. Aufgrund des iterativen Charakters eines menschbezogenen Entwicklungsprojekts mit Evaluationen und Überarbeitungen der Ergebnisse, wurde die MCI Betrachtung über den gesamten Projektverlauf ausgelegt.



\newpage

%!TEX root = ../dokumentation.tex

\section{Weiterführende Benutzermodellierung}
Innerhalb der Konzeptphase wurde bereits der Grundstein zur Benutzermodellierung gelegt. Bezogen auf die Zieldomäne fand eine Identifikation vorhandener Stakeholdergruppen samt Bezugsart statt. Für die weitere Auseinandersetzung war vorgesehen, die einzelnen Benutzergruppen genauer zu analysieren und entsprechende User Profiles zu entwickeln, um die Anwender und ihre gemeinsamen Merkmale genauer bestimmen zu können.\\

Daraufhin aufbauend sollen Beispielpersona erstellt werden, da diese innerhalb des Designprozesse geeigneter sind um Interaktionsprozesse nachvollziehbarer zu verstehen. Der Vorteil von Persona gegenüber User Profiles liegt darin, dass kognitive Vorgänge durch die “Modellierung” einer Person mit Hintergrundinformationen genauer zu verstehen sind.\\
Fokus des Usage Centered Design auf der eigentlichen Aufgabenausführung liegt, sollte diese Modellierungstiefe erreicht werden, da innerhalb des Verleihprozesses unterschiedliche Anwendertypen auftreten können mit eigenen (moralischen) Ansichten zum Shared Economy Konzept. Speziell bezogen auf eine Zielgruppe, die ihr Grundstück nur unter bestimmten Sicherheitsaspekten verleihen würde, kann die Auseinandersetzung neue Erkenntnisse zu geforderten Funktionalitäten und Restriktionen liefern.\\

Die Entwicklung von Projekten mit MCI Methodik, sieht grundsätzlich die Involvierung realer Anwender vor\footnote{Nach Inhalt der ISO 9241-210, Prof. Hartmann Draft zur MCI ab S.538}, um das System gebrauchstauglicher zu entwerfen. Speziell für die Evaluation sind sie von großer Bedeutung, um vorhandene Schwächen zu identifizieren. In Hinblick auf die Problemdomäne, wurde eine Zusammenarbeit mit realen Stakeholdern als relativ schwer eingeschätzt, da (Langzeit-) Reisende mit Campingbezug, "Couchsurfer"\footnote{Bezeichnung für aktive Teilnehmer am Couchsurfing (ref. Martkanalyse)} oder sogar Leute mit Erfahrung 
aus der problemrelevanten Domäne nicht innerhalb kürzerer Zeit auffindbar sind. Auch dafür wurde die Entwicklung von Personae durchgeführt, um an ihnen im Notfall Evaluationen durchzuführen. (Prinzipiell sollte die Zusammenarbeit mit realen Benutzern während der Entwicklung eines Projektes vorgezogen werden.)\\

Anzumerken ist hierbei, dass durch den hohen Detailierungsgrad der Hintergrundinformationen, wichtige Aspekte der eigentlichen Rolle für das System verdeckt werden können. Das Usage Centered Design konzentriert sich daher in der Entwicklung ihrer Modelle, auf eine abstraktere Betrachtungsebene.\footnote{Constantine \& Lockwood: Software for Use S.27} 

\newpage
Auch im weiteren Projektverlauf sollte verstärkt der Einsatz von essential models in Frage kommen, damit der Fallback bei Schwierigkeiten in der Umsetzung, sei es durch die Proof-of-Concepts oder bei der Entwicklung, nicht allzu groß ausfällt. Concrete Models mit Einbezug der Technologie und Details zur Umsetzung, wären in solch einem Fall überflüssig. Abstraktere Modelle behalten hingegen ihre Relevanz.\\

Die Entscheidung an diesem Punkt mit Personae zu beginnen und dann die Rollen zu identifizieren, basiert auf dem Ziel, möglichst früh mit konkreten Szenarien experimentieren zu können und die Domäne genauer zu verstehen. Das Role Modeling ist als abschließender Benutzermodellierungsschritt vorgesehen.


\subsection{User Profiles}
Entsprechend der Stakeholderanalyse\footnote{Konzeptkapitel 3.1 Seite 14} und der genaueren Benutzerbetrachtung im Rahmen des Nutzungskontexts\footnote{Konzeptkapitel 3.2 Seite 18}, ging es im nächsten Detaillierungsgrad um die Entwicklung von User Profiles. Dabei wurden charakterisierende Merkmale der Stakeholdergruppen genauer analysiert.\footnote{Nach Prof. Hartmann Draft zur MCI S.376}\\
Vorerst stellte sich die Frage, für welche Stakeholdergruppen die weitere Modellierung den größten Nutzen hat. Da innerhalb des Entwicklungsprozesses speziell auf die Bedürfnisse der Anwender eingegangen wird, erscheint es am sinnvollsten die Usergruppen zu betrachten, die direkt mit dem System agieren. Für das Role Modeling werden hierbei speziell die Focal Roles betrachtet. Dazu zählen vorallem die primary und secondary user. Für die tertiären user, die in ihrer Funktion hauptsächlich als Interessenten oder Richtungsgeber tätig sind, soll dieser Schritt nicht im Detail durchgeführt werden. Als potentielle Stakeholder dieser Klasse wurden beispielsweise andere Unterkunftsanbieter oder die Stadt, als Interessent durch den Tourismus, identifiziert. Da diese Gruppen grundsätzlich eher konzeptioneller Natur sind, spielen sie für die Entwicklung keine tragende Rolle.\\

Zuerst werden die primary user untersucht. Dazu gehören die Reisenden als Mieter und die Grundstücksbesitzer als Vermieter, die direkt mit dem System interagieren.

\newpage
\subsubsection{User Profile 1: Mieter (primary user)}
\begin{itemize}
   \item 
   \textbf{Demografisch:} Das Mindestalter beträgt 18 Jahre, es gibt kein obere Altersgrenze. Kernzielgruppe wird auf 18 - 40 Jahre geschätzt. Können prinzipiell aus jedem Land stammen, Wohnort ist für Reisende nicht relevant, müssen nur Anwedung und Internetzugang besitzen. Grundsätzlich also auch Reise aus verschiedenen Ländern möglich.

   \item 
  \textbf{Sozialökonomisch:} Mieter stammen aus unterschiedlichen Einkommensklassen, die Kernzielgruppe wird aus einkommensschwächeren bzw. finanzbewussten Leuten geschätzt (größte Motivation). 
   Aufgrund ihres Alters (und geschätzten Einkommens) wird nur ein ein geringer Teil auch Eigentumsbesitzer sein. Grundstücksbesitz eher bei Verwandten oder in Form von Gärten.

   \item 
   \textbf{Kultureller Hintergrund:} Können unterschiedlichen kulturellen Hintergrund haben, Kernzielgruppe wird jedoch in Deutschland oder dem europäischen Umfeld aufgewachsen sein. Sprachliche Kenntnisse im wesentlichen Deutsch und Englisch.

   \item
  \textbf{Physiologisch:} Aufgrund der Reisetätigkeit als Backpacker/Wanderer/Camper ist das Mindestmaß, dass die Anwendung benutzt werden kann; Grundsätzlich ist jeder potentieller Nutzer, daher sollte mit Beeinträchtigungen gerechnet werden. Auditive und visuelle Beeinträchtigungen sind möglich. Der Großteil besitzt aber keine schwerwiegenden Beeinträchtigungen, die speziell fokusiert werden müssen.

   \item 
   \textbf{Einstellung:} Vermutlich sozial offenere Menschen (vertrauensvolle), naturgebunden und reisefreudig. Haben keine Angst im Umgang mit der Technologie und der Thematikbezogenen Interaktion mit fremden Mietern als Vermieter. Auf ihrer Reise können sie zeitlich stark gebunden sein.

   \item 
  \textbf{Domänenkenntnisse:} Mit Erfahrung der Campingdomäne kann gerechnet werden, muss aber nicht zwangsläufig vorhanden sein.
   Erfahrene Leute in diesem Bereich, haben eventuell schon Kontakt mit Internetportalen gehabt, aber nicht zwangsläufig mit einer Smartphoneanwendung (für diesen Fall) gearbeitet.
   Innerhalb des Shared Economy haben sie eventuell schon Erfahrung in alternative Varianten wie Couchsurfing.

   \item
   \textbf{Technologieerfahrung:} Kenntnisse im Umgang mit dem Smartphone und der grundlegenden Anwendung von Applikationen kann vorausgesetzt werden.

   \item
  \textbf{verfügbare Technologien:} Besitzen ein Android Handy mit entsprechender Anwendung und Internetzugang. Dabei über Netzanbieter oder lokalen Hotspots.

   \item
   \textbf{Motivation:} Komfortabilität des Smartphones nutzen, Suchanfrage vereinfachen, Kosteneinsparung gegenüber gängigen Varianten, sozialer Kontakt-

   \item
   \textbf{Aufgabe:} Suchen einer geeigneten Unterkunft mit Anreise, Absprache und Bezahlung.
   
\end{itemize}
Wahrscheinlichkeit das diese Anwendergruppe auftritt: sicher


\newpage
\subsubsection{User Profile: Vermieter (primary user)}
\begin{itemize}
   \item 
   \textbf{Demografisch:} Das Mindestalter beträgt 18 Jahre, es gibt kein obere Altersgrenze. Kernzielgruppe wird auf 25 - 55 Jahre geschätzt. Wohnort innerhalb Deutschlands (bzw. Lage des Grundstückes) und ein bedeutender Teil lebt mit ihrer Familie (mit Kindern). Als Hilfsperson des secondary User auch mit jüngerem Alter möglich, dabei jedoch mit Registrierung des gesetzlichen Eigenstümers.

   \item 
  \textbf{Sozialökonomisch:} Jede Einkommensklasse möglich, sind im Besitz von Eigentum und stehen wahrscheinlich im Berufsleben. (Wenn sie als Zwischenperson agieren, Familienmitglied oder Bekannte.)

   \item 
   \textbf{Kultureller Hintergrund:} Großteil in Deutschland aufgewachsen und an Werte angepasst. Deutsche Sprachkenntnisse garantiert vorhanden, Englisch bei vielen.

   \item
  \textbf{Physiologisch:} Verschiedenste Beeinträchtigungen können auftreten, auch hier kein spezieller Fokus auf einzelne Einschränkungen.

   \item 
   \textbf{Einstellung:} Können unterschiedliche Einstellungen gegenüber dem Sharing Prinzip besitzen. Eventuell kein Interesse an sozialem Kontakt, auf geschäftlicher Ebene angesiedelt. Prinzipiell keine Angst im Umgang mit der Technologie, wobei Erfahrung nicht voraussgesetzt werden kann. Legen besonders Wert auf Datensicherheit.

   \item 
   \textbf{Domänenkenntnisse:} Können Erfahrung durch ähnliche Konzepte haben, grundsätzlich aber auch der erste Kontakt innerhalb der Anwendung möglich. Kenntnisse zur Vermietung mit rechtlichen Richtlinien sind ebenfalls nicht zwingend vorhanden, auch Erfahrung beim Camping/ Sharing Konzepten kann nicht vorausgesetzt werden.

   \item
   \textbf{Technologieerfahrung:} Erfahrung in der Benutzung eines Smartphones, tieferfreifendes, technisches Verständnis nicht zwangsläufig ausgeprägt.

   \item
   \textbf{verfügbare Technologien:} Smartphone mit Android und Anwendung, Internetzugang.

   \item
   \textbf{Motivation:} Finanzieller Anreiz wie Deckung möglicher Grundstückskosten oder für zusätzliches Einkommen. Dazu kommt soziale Erfahrung und Motivation zum Teilen/ Unterstützen.

   \item
   \textbf{Aufgabe:} Bekanntmachen des Angebotes für Mieter, Erfüllen der angedachten Vorraussetzungen des Reisenden.
   
\end{itemize}
Wahrscheinlichkeit das diese Anwendergruppe auftritt: sicher


\newpage
Folgendes User Profil befasst sich mit Grundstücksbesitzern, die selbst nicht mit der Technologie umgehen können oder wollen, dennoch bereit sind ihr Grundstück zu vermieten.
Die Mietobjektanzeige und Kontaktaufnahme wird dabei von einem weiteren Familienmiglied oder Bekannten in der stellvertretenden Rolle des primary users durchgeführt.\\

\subsubsection{User Profile: Zwischenperson für Vermieter (secondary user)}
\begin{itemize}
   \item 
   \textbf{Demografisch:} Mindestalter über 18 Jahre, Kernzielgruppe: 50 - 80 Jahre.  

   \item 
  \textbf{Sozialökonomisch:} Finanziell gefestigt, müssen nicht mehr im Berufsleben stehen. 

   \item 
   \textbf{Einstellung:} Erfahrene Menschen mit gefestigten Ansichten und Lebenserfahrungen. Einstellung oftmals fixiert und schwer beeinflussbar, daher Motivation schwer zu vermitteln.
   Stehen Technologie oftmals desinteressiert oder kritisch gegenüber. Technologieängste bei Vielzahl vorhanden.

   \item 
   \textbf{Domänenkenntnisse:} Haben eventuell Erfahrung in der Vergangenheit durch ähnliche Konzepte gehabt und konnten daher Interesse als Vermieter gewinnen.

   \item
   \textbf{Technologieerfahrung:} Nicht zwingend vorhanden. 

   \item
   \textbf{verfügbare Technologien:} Sind nicht im direkten Besitzer der Technologien, daher auf andere Besitzer angewiesen.

   \item
   \textbf{Motivation:} Anbieten des Grundstücks zur Vermietung.

   \item
   \textbf{Aufgabe:} Weitergabe der Angaben und Informationen an primary user, stellen den Input bereit.
   
\end{itemize}
Wahrscheinlichkeit das diese Anwendergruppe auftritt: sehr gering


\newpage
Bedeutend für das angedachte Geschäftsmodell\footnote{Konzeptseite 37} sind potentielle Kooperationspartner in Form durch Werbeverträge oder bei bestimmten Events.

\subsubsection{User Profile: Kooperationspartner (Werbe- oder Event-)}
\begin{itemize}
   \item 
   \textbf{Kultureller Hintergrund:} Kooperationspartner kann aus unterschiedlichen Ländern stammen, hat aber Bezug zum deutschen Markt oder dort stattfindende Events. Standort oder Unternehmen ist (zum Teil) in Deutschland angesiedelt oder wirbt auf dem deutschen Markt.

   \item
   \textbf{Auftreten:} Vorrangig Unternehmen die domänenspezifische Produkte verkaufen. Zusätzlich Eventpartner unterschiedlicher Art, wobei es nicht zwangsläufig der bestimmte Veranstaltungsort sein kann, sondern der Veranstalter. Organisiert in mittelgroßen bis großen Maßstäben.

   \item 
   \textbf{Domänenbezogenes:} Werbeprodukt mit Relevanz zur Campingdomäne, Eventpartner mit Relevanz für entsprechenden Standort. Können bereits Kooperationserfahrung mit anderen Anbietern der Domäne haben. Definitive Erfahrung im relevanten Marketing.

   \item
   \textbf{Einstellung:} Sehen Applikationen und In-App Werbung als lokrative Werbemöglichkeit und haben die Anwender und Technologie als Werbeplattform erkannt. Setzen auf Werbung innerhalb dieser Technologie und haben keine Angst davor.

   \item
   \textbf{Motivation:} Werben bei einer starken Kernzielgruppe; Wahrscheinlichkeit, dass Anwender eher angesprochen werden als durch den "öffentlichen" Werbeweg ist relativ hoch. Zielgruppe jedoch bedeutend kleiner als bei Fernseh- oder Straßenwerbung
   
\end{itemize}
Wahrscheinlichkeit das diese Anwendergruppe auftritt: mittel - hoch

Neben diesen User Profiles, wäre der Schritt in einer vertiefenden Iterationsphase relevante Entwickler und mögliche Mitarbeiter der Firma, wie Kundensupport, genauer zu analysieren. Für den gesetzten Projektfokus und gestellte Minimalziele, ist dies aber von keiner Relevanz und wird deshalb nicht weiter betrachtet.\\



\newpage
\subsection{Entwickelte Personae}
Aufbauend auf den entwickelten User Profiles, wurden im nächsten Schritt Beispielpersonae entwickelt, die Vetreter der einzelnen Stakeholdergruppen darstellen. Ziel war der Aufbau eines Portfolios, in welchem die zuvor identifizierten Charakteristiken weitestgehen abgedeckt werden.\footnote{Nach Prof. Hartmann Draft zur MCI ab S.381 mit Referenz auf Courage, Cathrine; Baxter, Kathy,Understanding Your Users. A practical guide to user requirements} 
Üblicherweise wird jede Persona mit einem Bild ausgestattet, um die Vorstellungskraft zu steigern. Aus datenschutzrechlichen Gründen, sollten an dieser Stelle nicht einfach passende Bilder von weiteren Quellen bezogen werden und sind in folgender Darstellung nicht mit inbegriffen.\\

\textbf{Persona 1: Mathias Schmidt}\\
Status: primäry user (Mieter)\\

Alter: 40 \\
Wohnort: München, verheiratet\\
Familenstand: verheiratet, 1 Sohn Tim Schmidt (15)\\
Beruf: Mechaniker\\
Beeinträchtigungen: Sehschwäche (Brillenträger) \\

Mathias ist 40 Jahre alt und leidenschaftlicher Mechaniker. Neben seiner Arbeit betreibt er zum Ausgleich viel Sport und ist ein naturgebundener Mensch.
Er ist sehr offen und kontaktfreudig gegenüber anderen Leuten. In seinen 20er Jahren ist er oft verreist und investierte sein Einkommen in diese Beschäftigung. Dadurch konnte er sich die Englische Sprache beibringen, die im Laufe der Zeit jedoch etwas an Übung verloren hat. Seit dem muss er für seine Familie sorgen und schafft es aus beruflichen Gründen oftmals nicht längerfristig Urlaub zu bekommen und viel zu verreisen.\\

Er hat keine Erfahrung mit Couchsurfing oder Ähnlichem, diese Art des Reisens ist ihm jedoch nicht fremd. Er selbst konnte auf diese Art bisher keine Reise durchführen, da er den Fokus auf seine Familie legt.
Was Technologie angeht ist er weitestgehend interessiert am derzeitigen Stand, verfolgt jedoch keine spezifischen Nachricht und nimmt Informationen eher durch Zufall auf. Mit Computern und seinem Smartphone kann er sicher umgehen. Sein berufliches Ziel für die Zukunft ist weiterhin gesichertes Einkommen, eine höhere Karrie strebt er nicht an, da er mit seinen aktuellen Umständen soweit zufrieden ist. Privat würde er gerne mehr Zeit mit seinem Sohn investieren, schafft oftmals aber nur spontane Kurzausflüge, die finanziell nicht zuviel fordern.\\

Da er kurzfristig Überstunden absetzen konnte, plant er für die nächste Woche spontan eine Fahrradtour mit seinem Sohn. Ein Arbeitskollege berichtete ihm von seiner vergangenen Deutschlandreise, bei der er günstige Schlafplätze bei privaten Vermietern gefunden hat. Nach einer Suche im Internet stellt er fest, dass gängige Hotels und Unterkünfte zu teuer sind und kein Campingplatz auf seiner Reiseroute zu finden ist. Nach kurzer Recherche stößt er auf die Find your Camp Anwendung. Er registriert sich dort als Mieter und verifiziert sich im Anschluss.\\




\textbf{Persona 2: Christine Bayer }\\
Status: primary user (Mieter)\\

Alter: 21\\
Wohnort: Köln\\
Familenstand: ledig\\
Beruf: Journalismus Studentin\\
Beeinträchtigungen: - \\

Christine ist 21 Jahre, stammt aus Köln und möchte in ihren Semesterferien als Backpackerin durch ganz Europa reise.
Neben ihrer Tätigkeit als Student, verdient sie ihr Geld mit einem Nebenjob als Kellnerin und erhält zudem Bafög.
Sie interessiert sich besonders für Kulturen und Fotografie. Technik nimmt sie eher nebenläufig war, ist sicher im Umgang mit ihrem Computer und besitzt ein aktuelles Smartphone.
Privat erhofft sie sich nach dem Studium eine Karriere als freue Journalistin und sieht ihr Berufsfeld vorallem im Ausland.

Bereits im letzten Jahr unternahm sie einen Kurzaufenthalt in Berlin und nutzte dabei das Couchsurfing Portal um eine Unterkunft zu finden. Auch wenn sie sich mit ihrem Gastgeber gut verstand, empfand sie es etwas befremdlich permanent auf jemand anderen angewiesen zu sein. Dieses Mal ist sie mit einer guten Freundin unterwegs und gemeinsam beginnen sie ihre Reise Richtung Süddeutschland. Da sie für einen längeren Zeitraum unterwegs sind, sind sie auf günstige Angebote gewiesen. Hotels kommen daher nicht in Frage und Hostels nur in Ausnahmefällen. Da sie im Sommer verreisen nehmen sie leichte Zeltausrüstung mit und können zur Not auch im freien im Übernachten. Bei der Suche nach Anwendungen für ihre Reise, stößt sie durch Zufall auf die Find your Camp Anwendung.



\newpage
\textbf{Persona 3: Karin Schultz}\\
Status:primary user (Vermieter)\\

Alter: 32\\
Wohnort: Gummersbach\\
Familenstand: vergeben, 1 Kind\\
Beruf: Grundschulehrerin\\
Beeinträchtigungen: Sehschwäche\\

Karin ist 32 Jahre und lebt seit ihrer Geburt in Gummersbach. Sie lebt gemeinsam mit ihrem Freund und deren gemeinsamen Tochter im Haus ihrer Eltern. Aus Platzgründen stellt diese kein Problem dar.
Als Grundschullehrin ist sie sehr offen und kontaktfreudig. Während ihrer Studienzeit verreiste sie gerne und absolvierte ein Teil ihres Studium im Ausland, wodurch sie viele Bekanntschaften schloss.
Seit der Geburt ihrer Tochter ist sie nicht mehr verreist und verbrachte viel Zeit mit ihrer Familie. \\

Über eine Kollegin erfährt sie von der Find your Camp Anwendung. Seit 1 Monat seie sie dort angemeldet und bietet ihren Garten zur Untermiete an. Neulich hatte sie Besuch von 2 Studenten die auf der Durchreise von Leipzig nach Amsterdam waren und die ganze Vermietung bereitete ihr, bis auf eine einmalige registrierung, kaum Aufwand.\\

Da Karins Familie ein Haus mit großem Grundstück besitzt, dass sie aus zeitlichen Gründen eher vernachlässigte, beschließt sie sich, ebenfalls als Vermieterin anzumelden. Sie erhofft sich damit, neben einem kleinen finanziellen Gewinn vorallem Kontakt zu netten Leuten und das Gewinnen neuer Erfahrungen.

\newpage
\textbf{Persona 4: Wolfgang Ehrlichmann}\\
Status: primary user (Vermieter)\\

Alter: 52\\
Wohnort: Gummersbach\\
Familenstand: verheiratet, 2 Kinder\\
Beruf: Zahnarzt\\
Beeinträchtigungen: Farbschwäche\\

Wolfgang ist 52 Jahre alt, studierte Zahnmedizin in Hamburg und heiratete mit 36 seine Frau Tina. Als gebürtige Gummersbacherin, wollte sie dort weiterhin leben, wodurch er in ihre Heimat zog.
Dort baute er sich eine kleine Praxis auf und lebt mit gesichertem Einkommen in einem Einfamilienhaus. Ihre beiden Kinder studieren derzeit, sind sehr aktiv und an neuen Erfahrungen interessiert.
Über sie erfährt von den Reisearten der Shared Economy und steht dem ganzen eher kritisch gegenüber. Er selbst könnte sich nicht vorstellen bei Fremden in der Wohnung zu übernachten. \\
Da er mit dem Computer nur die wesentlichen Aufgaben für seinen Alltag bewältigt und ansonsten zufrieden ist mit den Dingen an die er gewohnt ist, sieht er keinen direkten Mehrwert am Internet und diversen Social Media Portalen.
Auch auf seinem Smartphone reichen ihm die üblichen Anwendungen wie Kalender und Email, um den Alltag zu bewältigen.\\

Nach mehreren Gesprächen mit seinen Kindern, konnten sie ihn davon überzogen, ihren Garten einmalig zur Probe zu vermieten. Er konnte letztendlich davon überzeugt werden, dass er seine Daten nicht öffentlich ins Internet stellen muss und vorher sieht wer zu ihm kommen möchte. Er selbst muss lediglich das Inserat erstellen, die Töchter übernehmen die eigentliche Vermietung und somit läuft für ihn alles ab, "wie bei einem Besuch von Freunden seiner Kinder".
Er erwartet sich als Gegenleistung für die Besuch jedoch eine Übernachtungsgebühr und das er den Garten in einem einwandfreien Zustand vorfindet.

\newpage
\textbf{Persona 5: Clara Hütt}\\
Status: primary user (für Vermieter)\\

Alter: 16\\
Wohnort: Neuwied\\
Beruf: Schülerin, 11. Klasse \\
Beeinträchtigungen: - \\

Clara ist 16 Jahre alt und Schülerin aus Neuwied. 
Sie lebt mit ihrer Familie in einem Mehrfamilienhaus. Ihr Mutter ist Bankkauffrau und ihr Vater Geschäftsmann. Beruflich ist er unterhalb der Woche oft auf Reise, wodurch sie mit ihrer Mutter oftmals alleine ist.
Sie ist sehr zurückhaltend, Interessiert sich für fremde Länder und alles was mit Medien zu tun hat. Sie ist mit dem Internet aufgewachsen und kann mit Technik problemlos umgehen.
Nach der Schule möchte sie gerne in Köln studieren und irgenwann mal im Ausland wohnen.\\
In ihrere Freizeit unternimmt sie viel mit Freunden. Früher als ihr Vater weniger verreisen musste, war sie mit ihrer Familie am Wochenende meistens im Garten und zeltete dort ab und an mit ihren Freundinnen.
Aus zeitlichen Gründen wird der Garten derzeit weniger genutzt und die Mutter steht daher vor der Entscheidung den Garten abzugeben, da ansonsten unnötiger finanzieller Aufwand ansteht.
Als sie von Find your Camp erfährt möchte sie versuchen die Kosten darüber zu decken. Da sie selbst kein Smartphone besitzt, bittet sie Clara darum, das Inserat für sie zu erstellen. Sie soll lediglich bei Angeboten bescheid geben und die Mutter übernimmt die weiteren Verpflichtungen. Als Belohnung erhält Clara einen Anteil der Gebühren.\\


Diese fünf Beispielpersonae wurden innerhalb der ersten Iterationsphase entwickelt und sollten als Anwender dienen. 
Sollte sich im Projektverlauf bei der Evaluation der Bedarf an weiteren Sichtweisen ergeben, so wäre an dieser Stelle die Entwicklung weiterer Persona vorgesehen. Dieser Schritt war letztendlich kein Teil des Projektes.

\newpage
\subsection{Role Modeling}
\subsubsection{User Roles}
Die Einordnung der Anwender in Bezug auf ihre Beziehung zum System, ist eine der ersten Modellierungsschritte im usage centered design. Bereits im Vorfeld fand eine ausführliche Beschäftigung mit den Anwendern innerhalb der Stakeholderanalyse des Konzeptes\footnote{Konzeptseite 14} und dieser Dokumentation statt. Aus diesem Grund soll das Definieren der User Roles nicht in dem Maße stattfinden, wie es das Vorgehensmodell ursprünglich vorsieht.
Da bereits User Profiles erarbeitet wurde und sich diese den Focal Roles widmen, würde die Entwicklung von User Roles zu diesem Projektzeitpunkt zu keinen neuen Ergebnissen führen, als die in den angesprochenen Ausarbeitungsstellen. Anhand der erarbeiteten Rollen soll dennoch die User Role Map, als Bestandteil des Role Modelings entwickelt werden.

\subsubsection{User Role Map}
Die Beziehungen der einzelnen Rollen untereinander und die Ausprägungen, werden mit Hilfe der entwickelten User Role Map in Abbildung \ref{fig:rolemap} veranschaulicht.
\begin{figure}[H]
\includegraphics[width=0.75\textwidth]{./images/rolemodeling.png}
\caption{Entwurf der User Role Map}
\label{fig:rolemap}
\end{figure}

Grundsätzlich gibt es 3 Rollentypen, die sich wiederrum in Untertypen Einteilen lassen. Zum einen die Anwender, die als direkte Interessenten mit der Appplikation interagieren. Hierzu zählen die Personen in der Rolle des Mieters und Vermieters. Dabei ist es auch möglich, dass eine Personen zu bestimmten Zeiten eine der Rollen wechselt. Zu den Vermietern zählen die Grundstückbesitzer, die als primary user selber die Vermietung anbieten oder vertretende Vermieter. Damit sind die Personen gemeint, die das Grundstück nicht als Eigentümer besitzen, jedoch von diesem zur Vermietung beauftragt werden.
Eine zweiter Haupttyp ist der Mitarbeiter. In dargestellter Abbildung steht er in direkter Kommunikation mit Anwender, sei es als Supportkraft, Entwickler oder Eigentümer. In dieser Hinsicht ist die Bezeichnung des "Anwenders" eventuell ungünstig gewählt, da auch ein Mitarbeiter bei der Benutzung zum Anwender wird. Diese Trennung spiegelt dabei die Einstufung des internen und externen Stakeholders wieder. Zusätzlich dazu gibt es die Rolle des Kooperationspartners. Er steht in Kommunikation mit Mitarbeiter zur Organisation der Verträge und Absprachen, muss auf Dauer aber nicht zwangsläufig selbst der Benutzer der Applikation sein.






\newpage

%!TEX root = ../dokumentation.tex

\section{Task Modeling}
Nach Abschluss der Benutzermodellierung ging es darum, sich mit den vorhandenen Aufgaben der Anwender zu beschäftigen. Das bedeutet, sie zu identifizieren, die Struktur zu verstehen und die Beziehungen der einzelnen Aufgaben in einem konkreten Zusammenhang zu bringen.\\

Die Mensch Computer Interaktion bietet im Bereich der Aufgabenmodellierung unterschiedlichste Methodiken an, die entsprechend der Modellierungsziele und des Vorgehensmodells abgewogen wurden.\\
Im usage centered Design wird dieser Aufgabenbereich als Task Modeling bezeichnet und setzt sich aus der Entwicklung von essential use cases mit zugehöriger use case map zusammen.\\
Wie an vorheriger Stelle erläutert, sieht das Konzept des usage centered Design vor, die Modelle auf einer abstrakteren Ebene zu entwickeln, da der allgemeinere Fokus sich nicht in Details verirrt und zu viele Abhängigkeiten formuliert. Der Vorteil für das Projekt besteht darin, dass sich erarbeitet Aufgabenstrukturen nicht zu sehr an technische Details verhaften und damit bei Rückfällen oder Änderungen der Systemstruktur weiterhin benutzbar bleiben. \\
Ein Ansatz der Aufgabenmodellierung ist das szenarienbasierte Vorgehen, bei dem durch die Entwicklung konkreter Szenarien einzelne Interaktionssbeispiele nachvollziehbar dargestellt werden. Dies eignet besonders dann, wenn der Fokus auf den Benutzer liegt und beschrieben wird, wie und warum sie mit dem System agieren. Speziell im user centered Design ist diese Methode von relevanz, spielt aber in geplantem Vorgehensmodell keine konkretere Rolle.\\

Zur Darlegung der Aufgabenstrukturen gibt es unterschiedliche Darstellungsarten. Die Hierarchical Task Analysis eignet sich gerade bei komplexen Aufgaben, da kognitive Entscheidungen nachvollziehbar dargestellt sind. Ein ähnlicher Ansatz, der meist am Anfang der Aufgabenmodellierung erfolgt, ist die Entwicklung von Work Breakdown Charts, welche einzelne Aufgabenstufen durch hierachische Zusammenhänge veranschaulichen. In Bezug auf die anfängliche Anforderungsanalyse\footnote{Konzeptseite 24} der funktionalen Aspekte, konnte der Eindruck gewonnen werden, dass auftretende Aufgaben keine komplexeren Strukturen aufweisen, die durch eine grafische Darstellung nachvollziehbarer gemacht werden müssten. Daher wurden essential use cases als erster Methode der Aufgabenmodellierung entwickelt.

\newpage
\subsection{Essential Use Cases}
Essential use cases liefern einen grober Überblick über die Struktur der Aufgabe und ist dabei in der Formulierung frei von Annahmen und Technologiebehaftungen.\\
Orientiert am Aufbau der Beispiele in der Publikum von Constantine \& Lockwood\footnote{Constantine \& Lockwood: Software for Use S.105} wurden die funktionalen Anforderungen\footnote{Konzeptseite 34}, die zu diesem Zeitpunkt vorhanden waren, umgesetzt.
Eine Herausforderung stellte dabei die Wahl einer geeigneten Sprache und der Detailierungsgrad dar. Daher wurden sie innerhalb des Projektverlaufes mehrfach überarbeitet und erweitert. Im weiteren Verlauf der ersten Iterationsphase der MCI Auseinandersetzung, führte die Evaluation des ersten abstract prototype (wird im nächsten Kapitel 3.4 genauer beschrieben), dazu, dass die vorhandene nessential use cases einer erneuten Überarbeitung unterzogen werden mussten, da sie nicht alle funktionalen Aspekte berücksichtigten.\\

Im Folgenden sind die erarbeiteten use cases dargestellt. Diese Übersicht stellt dabei noch keine Gewährleistung auf Vollständigkeit dar, da im späteren Projektverlauf neue Anforderungen erkannt wurden, die aus zeitlichen Gründen keiner weiteren Iterationsphase unterstanden.\\ 

Essential Use Case \#1 (Tabelle \ref{tab:registrieren}) sieht die Registrierung eines Anwenders vor, welcher beim erstmaligen Start der Anwendung ein Benutzeraccount anlegen muss.
\begin{table}[H]
\caption{Essential Use Case \#1 registriereUser }
\centering
\begin{tabular}{l l}
\\ [-0.5ex]

\hline\hline
\\ [-0.5ex]
user intention & system responsibility
\\ [1.5ex]
\hline
\\ [-0.5ex]
Registrierung initiieren            &                                \\[1ex]
                              & Aktion anbieten und bei Wahl ausführen  \\[1ex]
Userinformationen angeben           &                                \\[1ex] 
                              & Userinformationen aufnehmen          \\[1ex]
                              & Userinformationen präsentieren       \\[1ex]
eingegebene Informationen bestätigen   &                                \\[1ex]
                              & Bestätigung abfragen, bei Auswahl akzeptieren  \\[1ex]
                              & Registrierung durchführen, Aktion beenden  \\[1ex]


\hline
\end{tabular}
\label{tab:registrieren}
\end{table}

Essential Use Case \#2 (Tabelle \ref{tab:profilverwalten}) befasst sich mit dem Anwenderziel, sein bereits angelegtes Profil zu verwalten, indem angegeben Funktionen geändert oder das Profil gelöscht wird.
\begin{table}[H]
\caption{Essential Use Case \#2 verwalteProfil }
\centering
\begin{tabular}{l l}
\\ [-0.5ex]

\hline\hline
\\ [-0.5ex]
user intention & system responsibility
\\ [1.5ex]
\hline
\\ [-0.5ex]
Verwaltung initiieren      &                                 \\[1ex]
                     & Aktion anbieten und bei Wahl ausführen     \\[1ex]
Profilart wählen        &                                 \\[1ex]
                     & Auswahl anbieten und bei Wahl ausführen  \\[1ex]             
vorhandene Informationen   &                                 \\[1ex]
angezeigt bekommen         &                                 \\[1ex]
                     & vorhandene Informationen präsentieren      \\[1ex] 
Informationen ändern       &                                 \\[1ex] 
                     & Aufnahme neuer Informationen ermöglichen    \\[1ex]
                     & löschen des Profils ermöglichen          \\[1ex]
Änderungen bestätigen      &                                 \\[1ex]
                     & Bestätigung anbieten, bei Wahl ausführen   \\[1ex]
                     & neue Informationen sichern, Aktion beenden \\[1ex]

\hline
\end{tabular}
\label{tab:profilverwalten}
\end{table}

\newpage
Essential Use Case \#3 (Tabelle \ref{tab:reiseprofil}) beschreibt die Erstellung eines "Reiseprofils". Im Rahmen des Projektes wurde diese Bezeichnung dafür vergeben, wenn der Reisende für eine Suchanfrage die gewünschte Ausstattung des Grundstückes und relevante Informationen zur Gruppengröße, Reisezeit und Preisvorstellung angibt.
\begin{table}[H]
\caption{Essential Use Case \#3 erstelleReiseprofil }
\centering
\begin{tabular}{l l}
\\ [-0.5ex]

\hline\hline
\\ [-0.5ex]
user intention & system responsibility
\\ [1.5ex]
\hline
\\ [-0.5ex]
Erstellung initiieren      &                                 \\[1ex]
                     & Aktion anbieten und bei Wahl ausführen   \\[1ex]
Reiseinformationen anlegen    &                                 \\[1ex] 
                     & neue Reiseinformationen aufnehmen        \\[1ex]
                     & Reiseinformationen präsentieren          \\[1ex]
eingegebene Informationen bestätigen   &                                 \\[1ex]
                     & Bestätigung anbieten, bei Wahl akzeptieren \\[1ex]
                     & Reiseprofil sichern, Aktion beenden      \\[1ex]

\hline
\end{tabular}
\label{tab:reiseprofil}
\end{table}


\newpage
Essential Use Case \#4 (Tabelle \ref{tab:mietobjekt}) beschreibt die Durchführung einer neuen Suchanfrage des Reisenden, um einen passenden Unterkunftsanbieter zu finden.

\begin{table}[H]
\caption{Essential Use Case \#4 sucheMietobjekt }
\centering
\begin{tabular}{l l}
\\ [-0.5ex]

\hline\hline
\\ [-0.5ex]
user intention & system responsibility
\\ [1.5ex]
\hline
\\ [-0.5ex]
Suche initiieren        &                                 \\[1ex]
                     & Aktion anbieten und bei Wahl ausführen   \\[1ex]
Mietanfrage spezifizieren  &                                 \\[1ex]
                     & vorhandene Mietanfrage präsentieren      \\[1ex]
                     & und Option für neuen Mietanfrage anbieten  \\[1ex]
Option auswählen           &                                 \\[1ex] 
                     & gewählte Option ausführen                \\[1ex]
                     & Mietanfrage präsentieren                 \\[1ex]
Anfrage starten         &                                 \\[1ex] 
                     & Aktion ausführen und Anfrage bearbeiten  \\[1ex]
Rückmeldung erhalten    &                                 \\[1ex]
                     & Ergebnis präsentieren                 \\[1ex]
                     & weitere Aktionen anbieten                \\[1ex]


\hline
\end{tabular}
\label{tab:mietobjekt}
\end{table}

\newpage
Essential Use Case \#5 (Tabelle \ref{tab:mietanfrage}) beschreibt die Aufgabe des Vermieters, auf eine erhaltene Mietanfrage zu reagieren und sie zu anzunehmen oder abzulehnen.
\begin{table}[H]
\caption{Essential Use Case \#5 beantworteMietanfrage }
\centering
\begin{tabular}{l l}
\\ [-0.5ex]

\hline\hline
\\ [-0.5ex]
user intention & system responsibility
\\ [1.5ex]
\hline
\\ [-0.5ex]
Mietanfrage erhalten       &                                 \\[1ex]
                     & neue Mietanfrage präsentieren            \\[1ex]
Mietanfrage beantworten    &                                 \\[1ex] 
                     & Reiseprofil präsentieren              \\[1ex]
                     & Aktionen anbieten und ausführen          \\[1ex]
Rückmeldung erhalten    &                                 \\[1ex]
                     & über Ausführung informieren           \\[1ex]
                     & Aktion beenden                     \\[1ex]

\hline
\end{tabular}
\label{tab:mietanfrage}
\end{table}


\newpage
Essential Use Case \#6 (Tabelle \ref{tab:bezahlung}) stellt die Aufgabe dar, die Bezahlung eines getätigten Mietauftrages durchzuführen. Diese Aufgabe sollte spätestens vor Weiterreise geschehen.
\begin{table}[H]
\caption{Essential Use Case \#6 bezahleMiete }
\centering
\begin{tabular}{l l}
\\ [-0.5ex]

\hline\hline
\\ [-0.5ex]
user intention & system responsibility
\\ [1.5ex]
\hline
\\ [-0.5ex]
Bezahlung initiieren    &                                 \\[1ex]
                     & Aktion anbieten und bei Wahl ausführen   \\[1ex]
Mietauftrag wählen         &                                 \\[1ex] 
                     & Mietaufträge präsentieren und Wahl ermöglichen \\[1ex]
Bezahlung durchführen      &                                 \\[1ex]
                     & Auftragsinformationen präsentieren       \\[1ex]
                     & Informationsaufnahme anbieten            \\[1ex]
Auftrag bestätigen         &                                 \\[1ex]
                     & Zusammenfassung der Informationen präsentieren \\[1ex]
                     & Optionen anbieten  und ausführen         \\[1ex]
Rückmeldung erhalten    &                                       \\[1ex]
                     & Ergebnis präsentieren                  \\[1ex]
                     & weitere Aktionen anbieten                 \\[1ex]


\hline
\end{tabular}
\label{tab:bezahlung}
\end{table}


\newpage
Essential Use Case \#7 (Tabelle \ref{tab:bewertung}) beschreibt die Absicht des Benutzers, den Verleihpartner nach durchgeführtem Vermietprozess zu bewerten.
\begin{table}[H]
\caption{Essential Use Case \#7 bewerteUser }
\centering
\begin{tabular}{l l}
\\ [-0.5ex]

\hline\hline
\\ [-0.5ex]
user intention & system responsibility
\\ [1.5ex]
\hline
\\ [-0.5ex]
Bewertung initiieren    &                                 \\[1ex]
                     & Aktion anbieten und bei Wahl ausführen   \\[1ex]
Mietauftrag wählen         &                                 \\[1ex] 
                     & Mietaufträge präsentieren und Wahl ermöglichen \\[1ex]
Bewertung abgeben          &                                 \\[1ex]
                     & Bewertungsinformationen aufnehmen        \\[1ex]
                     & Aktion durchführen                 \\[1ex]
Rückmeldung erhalten    &                                 \\[1ex]
                     & Ergebnis präsentieren                    \\[1ex]
                     & Aktion beenden                     \\[1ex]

\hline
\end{tabular}
\label{tab:bewertung}
\end{table}


\newpage
Essential Use Case \#8 (Tabelle \ref{tab:kontaktaufnahme}) sieht die Kontaktaufnahme zweier Anwender vor, nachdem eine Mietanfrage erfolgreich durchgeführt und angenommen wurde. Hierbei sind Textnachrichten zur Kommunikation zwischen Vermieter und Mieter gemeint.
\begin{table}[H]
\caption{Essential Use Case \#8 kontaktiereUser }
\centering
\begin{tabular}{l l}
\\ [-0.5ex]

\hline\hline
\\ [-0.5ex]
user intention & system responsibility
\\ [1.5ex]
\hline
\\ [-0.5ex]
Kontaktaufnahme initiieren &                                 \\[1ex]
                     & Aktion anbieten und bei Wahl ausführen   \\[1ex]
relevanten Mietauftrag     &                                 \\[1ex] 
anzeigen             &                                 \\[1ex] 
                     & vorhandene Mietaufträge präsentieren        \\[1ex]
                     & gewählten Auftrag präsentieren        \\[1ex]
Nachricht verfassen        &                                 \\[1ex]
                     & Nachrichtenaufnahme anbieten             \\[1ex]
                     & Aktion durchführen                 \\[1ex]
\hline
\end{tabular}
\label{tab:kontaktaufnahme}
\end{table}


\newpage
Essential Use Case \#9 (Tabelle \ref{tab:ausloggen}) beschreibt das Ausloggen vom verbundenen Account eines Nutzers.
\begin{table}[H]
\caption{Essential Use Case \#9 userAusloggen }
\centering
\begin{tabular}{l l}
\\ [-0.5ex]

\hline\hline
\\ [-0.5ex]
user intention & system responsibility
\\ [1.5ex]
\hline
\\ [-0.5ex]
Ausloggen initiieren    &                                 \\[1ex]
                     & Aktion anbieten und bei Wahl ausführen   \\[1ex]
Rückmeldung erhalten    &                                 \\[1ex]
                     & Ergebnis präsentieren                 \\[1ex]
                     & Aktion beenden                     \\[1ex]

\hline
\end{tabular}
\label{tab:ausloggen}
\end{table}



Essential Use Case \#10 (Tabelle \ref{tab:aktionen}) beschreibt das Ziel des Anwenders, eine der vorhandenen Funktionen wie  "Suchanfrage starten" zu finden und auszuführen. Dieser use case entstand im Rahmen der letzten Überarbeitung, als Ergebnis der Auswertung des ersten abstrakten Prototypen. Die Tätigkeit die innerhalb dieses Anwendungsfall dargestellt ist, stellt den ersten Schritt jedes vorherigen use cases dar, wurde für ein weiteres content model jedoch in Betracht gezogen.

\begin{table}[H]
\caption{Essential Use Case \#10 wähleAktion }
\centering
\begin{tabular}{l l}
\\ [-0.5ex]

\hline\hline
\\ [-0.5ex]
user intention & system responsibility
\\ [1.5ex]
\hline
\\ [-0.5ex]
mögliche Aktion finden        &                                \\[1ex]
                        & vorhandene Aktionen präsentieren        \\[1ex]
Aktion ausführen           &                                \\[1ex] 
                        & Auswahl der Aktion ermöglichen       \\[1ex]
                        & Aktion initiieren                    \\[1ex]
\hline
\end{tabular}
\label{tab:aktionen}
\end{table}

\newpage
\subsection{Use Case Map}
Die zuvor dargestellten use cases, stehen bei der Verwendung des Systems in unmittelbarem Zusammenhang. Die Beziehung untereinander ist in der use case map (ref) gezeigt. Dieses Modell stellte für weitere Ausarbeitungen den Vorteil dar, dass beim Prototyping Verbindungen zwischen den Funktionen und den möglichen Working Spaces der Anwendung veranschaulicht werden.

TODO Einbindung Use Case Map

\newpage
\subsection{Concrete Use Case}

Zusätzlich dazu konnte der abstrakte Charakter nicht immer zum genauen Verständnis der Struktur führen.
use case methoden: allistar cockburn template \\
concrete use case, lockwood\\

Szenario auch möglich, verweis zu teil 1
Nachteil im Design durch detailierungsgrad S. 101 allgemeine organisation vernachlässigen

In kurzer Form wie Beispiel Software for Use Seite 100 unter Einbezug der Persona
Ein use case erfasst eine Beziehung zwischen stakeholder und dem technischen Subsystem bzgl. des Verhaltens des technischen Subsys- tems. Es beschreibt das Verhalten des technischen Subsystems unter bestimmten Rahmenbedingungen als Antwort auf eine Anfrage des stakeholders (primary actor). Der primary actor iniziiert die Interak- tion, um ein Ziel zu erreichen.


Use Case \#1 (Tabelle \ref{tab:anmeldenUC})
\begin{table}[H]
\caption{Use Case\#1 registriereUser }
\centering
\begin{tabular}{l l}
\\ [-0.5ex]

\hline\hline
\\ [-0.5ex]
user intention & system responsibility
\\ [1.5ex]
\hline
\\ [-0.5ex]
Funktion zur Neuanmeldung auswählen             &                                   \\[1ex]
                                       & Funktion bereitstellen, die dies ermöglicht   \\[1ex]
                                       & Über Richtlinien hinweisen              \\[1ex]
Kenntnissnahme zu Richtlinien bestätigen        &                                   \\[1ex]
                                       & Bestätigung ermöglichen                 \\[1ex]
Informationen eingeben                       &                                   \\[1ex] 
Personeninformationen Name, Vorname, Gebdat etc.   &                                   \\[1ex] 
                                       & Eingabefelder bereitstellen, Informationen    \\[1ex]
                                       & aufnehmen, weitere Schritte ermöglichen    \\[1ex]
Korrektheit der Daten prüfen und bestätigen        &                                   \\[1ex]
                                       & Funktion zur Bestätigung anbieten          \\[1ex]
                                       & weitere Funktionalitäten anbieten, Mietobjekt \\[1ex]
                                       & anlegen oder Mietauftrag anlegen           \\[1ex]
                                       & Informationen zur Verifizierung senden     \\[1ex]
Profil verifizieren                          &                                   \\[1ex]
                                       & TODO verifizierungsschritte (extra Use Case?) \\[1ex]


\hline
\end{tabular}
\label{tab:anmeldenUC}
\end{table}


Use Case \#2 (Tabelle \ref{tab:profilBearbeitenUC})
\begin{table}[H]
\caption{Use Case\#2 verwalteBenutzerprofil }
\centering
\begin{tabular}{l l}
\\ [-0.5ex]

\hline\hline
\\ [-0.5ex]
user intention & system responsibility
\\ [1.5ex]
\hline
\\ [-0.5ex]
Funktion zur Verwaltung auswählen   &                                      \\[1ex]
                           & Funktion bereitstellen, die dies ermöglicht      \\[1ex]
Identität bestätigen          &                                      \\[1ex]
                           & Identitätsabfrage einleiten und mit Eingabe      \\[1ex]
                           & prüfen, anschließend bereits vorhandene          \\[1ex] 
                           & Informationen anzeigen                     \\[1ex] 
Neuen Informationen eingeben     &                                      \\[1ex] 
                           & neue Informationen über Eingabefelder aufnehmen,    \\[1ex]
                           & auf syntaktische Korrektheit prüfen           \\[1ex]
Neue Informationen abspeichern      &                                      \\[1ex]
                           & Funktion zum abspeichern anbieten             \\[1ex]
\hline
\end{tabular}
\label{tab:profilbearbeitenUC}
\end{table}


Use Case \#3 (Tabelle \ref{tab:mietauftragUC})
\begin{table}[H]
\caption{Use Case\#3 erstelleReiseprofil }
\centering
\begin{tabular}{l l}
\\ [-0.5ex]

\hline\hline
\\ [-0.5ex]
user intention & system responsibility
\\ [1.5ex]
\hline
\\ [-0.5ex]
Funktion zum neuen Mietauftrag wählen     &                                   \\[1ex]
                                 & Funktion bereitstellen, die dies ermöglicht   \\[1ex]
Profil des Reisenden auswählen über den      &                                   \\[1ex]
gemietet werden soll                      &                                   \\[1ex]
                                 & Anmeldung mit Benutzerdaten ermöglichen    \\[1ex]
Organisatorische Informationen eintragen  &                                   \\[1ex] 
                                 & Eingabefelder bereitstellen und            \\[1ex]
                                 & Informationen aufnehmen                 \\[1ex]
Gewünschte Austattung auswählen           &                                   \\[1ex]
                                 & Austattungsmerkmale anzeigen und           \\[1ex]
                                 & Auswahl über Menü ermöglichen           \\[1ex]
Informationen bestätigen und abspeichern  &                                   \\[1ex]
                                 & Funktion zum abspeichern anbieten          \\[1ex]

\hline
\end{tabular}
\label{tab:mietauftragUC}
\end{table}

\newpage
Use Case \#4 (Tabelle \ref{tab:mietobejktUC})
\begin{table}[H]
\caption{Use Case\#4 findeMietobjekt }
\centering
\begin{tabular}{l l}
\\ [-0.5ex]

\hline\hline
\\ [-0.5ex]
user intention & system responsibility
\\ [1.5ex]
\hline
\\ [-0.5ex]
Funktion zur neuen Suchanfrage starten    &                                   \\[1ex]
                              & Funktion bereitstellen, die dies ermöglicht   \\[1ex]
Passenden Mietauftrag zur Reise wählen &                                   \\[1ex]
                              & Vorhandene angelegte Mietaufträge anzeigen \\[1ex]
                              & und Auswahl ermöglichen, Option zum anlegen   \\[1ex]
                              & eines neuen Auftrags anzeigen              \\[1ex]
Suchanfrage ausführen               &                                   \\[1ex] 
                              & Ausgewählten Auftrag anzeigen  und Suchanfrage   \\[1ex]
                              & funktional ermöglichen, nach Bestätigung      \\[1ex]
                              & GPS Daten bestimmen und Anfrage weiterleiten. \\[1ex]
                              & Benutzer über getätigte Suche informieren     \\[1ex]
Rückmeldung zu vorhandenen Angeboten   &                                   \\[1ex]
erhalten                      &                                   \\[1ex]
                              & Wenn kein Angebot vorhanden ist, direkte      \\[1ex]
                              & Rückmeldung. Ansonsten regelmäßige Abfrage    \\[1ex]
                              & an Server, ob Vermieter angenommen hat.    \\[1ex]
                              & Meldung an Benutzer schicken bei Treffer      \\[1ex]
                              & nach Ablauf einer kritischen Zeit.            \\[1ex]
\hline
\end{tabular}
\label{tab:mietobjektUC}
\end{table}

\newpage
Use Case \#5 (Tabelle \ref{tab:mietanfrageUC})
\begin{table}[H]
\caption{Use Case\#5 beantworteMietanfrage }
\centering
\begin{tabular}{l l}
\\ [-0.5ex]

\hline\hline
\\ [-0.5ex]
user intention & system responsibility
\\ [1.5ex]
\hline
\\ [-0.5ex]
Meldung über relevante Mietanfrage  &                                   \\[1ex]
erhalten                      &                                   \\[1ex]
                           & Mietauftrag empfangen und lokal matchen    \\[1ex]
                           & bei Bestätigung Meldung mit Informationen     \\[1ex]
                           & auf dem Display hervorheben, Benachrichtigen  \\[1ex]
Mieterdaten ansehen              &                                   \\[1ex]
                           & Reisedaten anzeigen: Datum Gruppengröße    \\[1ex]
                           & Preisvorstellung, Anfragezeit              \\[1ex]
Mietanfrage beantworten          &                                   \\[1ex] 
                           & Funktion anbieten zur Annahme oder Ablehnung  \\[1ex]
                           & der Anfrage, Meldung an Mieter schicken       \\[1ex]
Annahme des Auftrags erhalten    &                                   \\[1ex]
                           & Bestätigung des Mieters empfangen, Auswahl \\[1ex]
                           & dem Vermieter anzeigen und ggf. weitere       \\[1ex]
                           & Funktion zum Senden der Objektinformationen   \\[1ex]
Objektinformationen verschicken     &                                   \\[1ex]
Funktionsausführung              &                                   \\[1ex]



\hline
\end{tabular}
\label{tab:mietanfrageUC}
\end{table}


\newpage
Use Case \#6 (Tabelle \ref{tab:mietobjektAUC})

\begin{table}[H]
\caption{Use Case\#6 registriereMietobjekt }
\centering
\begin{tabular}{l l}
\\ [-0.5ex]

\hline\hline
\\ [-0.5ex]
user intention & system responsibility
\\ [1.5ex]
\hline
\\ [-0.5ex]
Funktion zur Objektverwaltung auswählen      &                                   \\[1ex]
                                 & Funktion bereitstellen, die dies ermöglicht   \\[1ex]
Auswahl ob altes Objekt bearbeiten oder      &                                   \\[1ex]
neues Anlegen                          &                                   \\[1ex]
                                 & Funktion bereitstellen, die dies ermöglicht   \\[1ex]
Neues Mietobjekt registrieren          &                                   \\[1ex] 
Besitzer festlegen                     &                                   \\[1ex] 
                                 & Verifizierung des Besitzers durch Accountdaten \\[1ex]
                                 & Eingabefelder bereitstellen und prüfen     \\[1ex]
Objektstandort bestimmten              &                                   \\[1ex]
                                 & Funktion zur Standortbestimmung bereitstellen  \\[1ex] 
                                 & Kartenanbindung mit Stecknadel?              \\[1ex]   
Objektmerkmale wie Größe, Preis eingeben  &                                   \\[1ex]
                                 & Eingabefelder anzeigen und Eingabe über       \\[1ex]
                                 & vordefinierte Optionen ermöglichen         \\[1ex]
Gewünschte Austattung auswählen           &                                   \\[1ex]
                                 & Austattungsmerkmale anzeigen und           \\[1ex]
                                 & Auswahl über Menü ermöglichen           \\[1ex]
Informationen kontrollieren, bearbeiten   &                                   \\[1ex]
und abspeichern                     &                                   \\[1ex]
                                 & Funktionalität zur Navigation bereitstellen   \\[1ex]
                                 & eingegebene Daten zeigen und speichern     \\[1ex]
                                 & funktional ermöglichen                  \\[1ex]



\hline
\end{tabular}
\label{tab:mietobjektAUC}
\end{table}



\subsection{Szenarien aus Concrete durch Alternativen}
TODO








\newpage

%!TEX root = ../dokumentation.tex

\section{Content Modeling}

\subsection{Abstract Prototype 1}

User environment design Schaffen eines Architekturmodells, das zeigt, wie die einzelnen Systemkomponenten zusammen in Beziehung stehen. Repräsentieren der Struktur- und der Funktions- Cluster des Systems unabhängig von Aspekten des Benutzerinterfaces 
oder der Implementierung

Sicherstellen, dass der Arbeitsfluss innerhalb des Systems die Arbeit unterstützt wird. Unterstützen der Designer in der Konzentration auf Arbeitsfluss und Funktion in dem System anstelle der Betrachtung von Benutzer Interface und Implementierung. Schaffen von System-Repräsentationen, die Planung unterstützen

 Abstract Prototyping\\
 content of user interface ohne zu zeigen wie es aussieht\\

 Content Modeling \\
 inhalte des user interface ohne details zum aussehen
 content model = sammlung von material (was den user interessiert zu sehen oder zu ändern), tools (ermöglichen den user dies zu tun) und working spaces(teile die tools und materials verbinden)\\

 Papier = working spaces (post it = tools and materials)
 content model + navigation map

 conceptual scenarios£

\subsubsection{Interface Contexts 1}

Hartmann S442
Abb. \ref{interfaceContents1}

\begin{figure}[H]
\centering
\hfill
\subfloat[aktionAuswählen \label{pic:aktionAuswaehlen}]{\includegraphics[width=.5\textwidth]{./images/abstract/version1/aktionAuswaehlen.JPG}}
\hfill % alternativ auch \hspace{1cm} für genaue Angaben
\subfloat[sucheMietobjekt \label{pic:sucheMietobjekt}]{\includegraphics[width=.5\textwidth]{./images/abstract/version1/sucheMietobjekt.JPG}}
\hfill %
\caption{Inter. Context AP1: aktionAuswählen und sucheMietobjekt }
\label{interfaceContents1}
\end{figure}

\subsubsection{Context Navigation Map 1}

Abb. \ref{fig:navigationmap1}
\begin{figure}[H]
\includegraphics[width=1\textwidth]{./images/navigationmap1.png}
\caption{Context Navigation Map Version 1}
\label{fig:navigationmap1}
\end{figure}


\newpage
\subsubsection{Evaluation}

\subsection{Abstract Prototype 2}


\subsubsection{Interface Contexts 2}
Abb. \ref{interfaceContents2}

\begin{figure}[H]
\centering
\hfill
\subfloat[anzeigenVorhandenerAktionen \label{pic:anzeigenVorhandenerAktionen}]{\includegraphics[width=.5\textwidth]{./images/abstract/version2/anzeigenVorhandenerAktionen.JPG}}
\hfill % alternativ auch \hspace{1cm} für genaue Angaben
\subfloat[suchanfrageStarten \label{pic:suchanfrageStarten}]{\includegraphics[width=.5\textwidth]{./images/abstract/version2/suchanfragenStarten.JPG}}
\hfill %
\caption{Inter. Context AP2: anzeigenVorhandenerAktionen und suchanfrageStarten }
\label{interfaceContents2}
\end{figure}

\subsubsection{Context Navigation Map 2}

Abb. \ref{fig:navigationmap2}
\begin{figure}[H]
\includegraphics[width=1\textwidth]{./images/navigationmap2.png}
\caption{Context Navigation Map Version 2}
\label{fig:navigationmap2}
\end{figure}

\newpage
\subsubsection{Evaluation}




\newpage

%!TEX root = ../dokumentation.tex

\section{Gestaltungslösungen}


\subsection{Prototyping}

Hartmann Seite 442

\subsection{Paperbased Prototyping}

Testen und Modifizieren des neuen Systems in Partnerschaft mit den Benutzern, indem papierbasierte „mock-ups“ des Benutzerinterfaces geschaffen werden. Wirkliche Arbeitsaufgaben mit „Papier- System“ bearbeiten lasseen und nicht nur „reviewen“ sammeln von Einschätzungen

Sicherstellen eines verlässlichen Weges um mit Benutzern über das zu schaffende System zu sprechen. Verifizieren des Entwurfs bevor der codiert wird. Vertiefen der Anforderungen an das System. Frühes Testen von Benutzer- Interface Ideen und Produktkonzeptionen

\subsection{Evaluation}

empirischer Ansatz
Vertreter der Stakeholder ,analytisch nicht möglich, da keine MCI Experten
Hartmann Seite 
Techniken
Think aloud (formativ) während, dient zur Entdeckung, mit benutzerbeteiligung, qualitativ\\
testperson testet system und spricht gedanken laut aus + protokoll
tester, moderator
user testing\\
test mit probanden
problem rahmenbedingungen beeinflussen
observation\\
interpretation des Analysten
survey\\
schwierig da psychologisch
interview\\
cognitive walkthrough\\
	gedankliches Nachverfolgen der Aufgabenerledigung, erlernbarkeit, gebrauchstauglichkeit
expert review\\

Auswahl nach Zielsetzung, Ressourcen, , domänenspezifischer Kontext\\
Zeitpunkt
formative vor oder während vs summative E abschließend\\
hinsichtlich erhobener Daten
qualitative sprachliche Basis vs quantitative E auf Zahlenbasis\\
wissenschaftstheoretische ausrichtung
induktive bottom Einzelbeobachtungen verallgemeinern up vs dediktive E vom allgemeinen zum konkreten top down\\

Auswahl nach Zeitpunkt im EP, Grad der Benutzerbeteiligungm Aufwand, Untersuchungsfokus

Interaktionsphase\\
Durchführungsphase\\
Abschlussphase\\
high fidelty prototype
passiver Prototyp
working prototyp 

Abb. \ref{fig:evaluation11}
\begin{figure}[H]
\includegraphics[width=1\textwidth]{./images/evaluation/eva11.JPG}
\caption{Evaluationsprotokoll 1.1}
\label{fig:evaluation11}
\end{figure}

Abb. \ref{fig:evaluation12}
\begin{figure}[H]
\includegraphics[width=1\textwidth]{./images/evaluation/eva12.JPG}
\caption{Evaluationsprotokoll 1.2}
\label{fig:evaluation12}
\end{figure}

Abb. \ref{fig:evaluation21}
\begin{figure}[H]
\includegraphics[width=1\textwidth]{./images/evaluation/eva21.JPG}
\caption{Evaluationsprotokoll 2.1}
\label{fig:evaluation21}
\end{figure}

Abb. \ref{fig:evaluation22}
\begin{figure}[H]
\includegraphics[width=1\textwidth]{./images/evaluation/eva22.JPG}
\caption{Evaluationsprotokoll 2.2}
\label{fig:evaluation22}
\end{figure}

Abb. \ref{fig:mainneu}
\begin{figure}[H]
\includegraphics[width=1\textwidth]{./images/mainneu.JPG}
\caption{Evaluationsprotokoll 2.2}
\label{fig:mainneu}
\end{figure}

\subsection{Grundsätze der Dialoggestaltung}
 ISO 9241 Teil 110
 Gestaltungsempfehlungen ISO Teil 12-17
\\
 1. Aufgabenangemessenheit\\
 	Fokus auf Aufgabe, nicht auf Technik\\
 	Dialogsystem: nur Informationen für Aufgabe, Hilfefunktion, \\automatische Ausführung, Eingabe Cursor im ersten Eingabe Widget, bei wiederkehrenden Aufgaben unterstützen = Speichern von aufeinander folgenden Interaktionsschritten, Standartwerte als Vorgabewerte\\
\\
 2. Selbstbeschreibungsfähigkeit\\
    Verständnis zum aktuellen Standpunkt im Dialog, was gemacht werden kann, wie man es ausführt\\
    Rückmeldung durch unmittelbares Anzeigen der Eingaben, bei schwerwiegenden Folgen Bestätigung einbauen, keine Fachterminologie, Begriffe erläutern und Benutzerhandbuch, Informationen zum aktuellen Stand, zB prozentualer Beabreitungsstand, Überblick über zukünftige Dialogschritte, Informationen zum Eingabedatentyp

 3. Steuerbarkeit\\
 	Geschwindigkeit unter der Kontrolle des Benutzers, Dialogsystem setzt immer auf nächsten Eingabecursor, sollte aber freie Interaktion ermöglichen. Interaktionsschritte sollten zurücknehmbar sein, auch auf das zurückgreifen zuvor gelöschter Objekte, Short Cuts?

 4. Erwartungskonformität\\
 	konsistent, Merkmalen der Benutzern entsprechend, Zustandsmeldungen immer an der selben Stelle, Interaktionssequenz wird immer durch selbe Taste beendet, ähnlichkeiten, Antwortzeiten mitteilen, zB über ladebalken, status widget, prozentsatz des datenvolumens

 5. Fehlertoleranz\\
 	testen des Datentypens nach Eingabe, Fehler erläutern mit Fehlermeldung, farbige hervorhebung 

 6. Individualisierbarkeit\\
 	Anpassung an individuelle Fähigkeit und Bedürfnisse, Schriftgröße und Farbe, Sprache

 7. Lernförderlichkeit\\
 	Fehlerbeschreibung, learning by doing durch hohe Fehlertoleranz, Übungsszenarien\\

 	in weiteren Iso nachschauen
 	das Erkennen und Spezifizieren von Dialoganforderungen auf der Grundlage der verschiedenen Dialogtechniken, die in den Teilen der Norm ISO 9241-14 bis ISO 9241-17 beschrieben sind;
 den Entwurf von Gestaltungslösungen unter Einhaltung von ISO 9241-12 bis ISO 9241-17\\

 aus Benutzerperspektive gucken

 Gestaltungsanforderungen\\
 Gestaltungslösungen

 Nutzungskontext ist Quelle für DIaloganfordernungenTODO physische und soziale Arbeitsumgebung

 Gestaltung berücksichtigt gesamte User Experience\\
 Darstellung\\
 Funktionalität\\
 Systemleistung\\
 interaktiven Verhalten\\
 unterstützenden Ressourcen (Hardware und Software)\\
 + bisherigen Erfahrungen, Einstellungen, Fähigkeiten, Gewohnheiten und der Persönlichkeit des Benutzers\\

 organisatorische Auswirkungen, Benutzerdokumentation, Online-Hilfe, unterstützende Betreuung und Instandhaltung (einschliesslich Bera- tung und Kundenkontaktstellen), Schulung, langfristiger Gebrauch, und Produktverpackung (einschliesslich der Eindrücke bei der ers- ten Inbetriebnahme) sollten die Erfahrungen des Benutzers mit vor- herigen oder anderen Systemen und Probleme wie Markenkennzeich- nung und Werbung bedacht werden.\\

 Kontext sozialer kontext, alleine in der gruppe, ton an oder aus
 organisationaler kontext, zeit, ort

 Abstract Prototyping\\
 content of user interface ohne zu zeigen wie es aussieht\\

 Content Modeling \\
 inhalte des user interface ohne details zum aussehen
 content model = sammlung von material (was den user interessiert zu sehen oder zu ändern), tools (ermöglichen den user dies zu tun) und working spaces(teile die tools und materials verbinden)\\

 Papier = working spaces (post it = tools and materials)
 content model + navigation map

 conceptual scenarios£

\subsection{Sonstiges}

\subsubsection{Metaphern}
Icons können Metapher aktivieren und mentale Modelle erzeugen



\newpage

\input{chapter/kapitel3_6_projektplan.tex}

\newpage


%!TEX root = ../dokumentation.tex

\section{Aussicht}
Mentale Modelle: funktional, strukturell, wie es benutzt wird, wie es funktioniert\\

prototyping 

\newpage




	%Die Zähler für Tabellen und Abbildungen werden zurückgesetzt, damit
	%in jedem Kapitel die Nummerierung neu beginnt
	\setcounter{table}{1}
	\setcounter{figure}{1}
	%Einbinden des zweiten Kapitels
	
	%Systemdokumentation
	%!TEX root = ../dokumentation.tex

\chapter{Systemdokumentation}

%!TEX root = ../dokumentation.tex

\section{Ergebnisse des anwendungszentrierten Gestaltungsprozesses}

%!TEX root = ../dokumentation.tex

\section{Zielsetzung}


\newpage

%!TEX root = ../dokumentation.tex

\subsection{Nutzungskontext}
TODO ausformulieren
 Kontext sozialer kontext, alleine in der gruppe, ton an oder aus
 organisationaler kontext, zeit, ort

organisatorische Auswirkungen, Benutzerdokumentation, Online-Hilfe, unterstützende Betreuung und Instandhaltung (einschliesslich Bera- tung und Kundenkontaktstellen), Schulung, langfristiger Gebrauch, und Produktverpackung (einschliesslich der Eindrücke bei der ers- ten Inbetriebnahme) sollten die Erfahrungen des Benutzers mit vor- herigen oder anderen Systemen und Probleme wie Markenkennzeich- nung und Werbung bedacht werden.\\

zu Nutzungskontext\\
die Ziele und Arbeitsaufgaben der Benutzer. Die Beschreibung sollte Gesamtziele für die Verwendung des Systems enthalten. Die Merkmale jener Aufgaben, die Gebrauchstauglichkeit be- einflussen können, sollten beschrieben werden ebenso wie ggf. vorhandene Auswirkungen auf die Gesundheit und Sicherheit. Beschreibung sollte die Verteilung der Aktivitäten und Arbeits- schritte zwischen Mensch und technischen Hilfsmitteln einsch- liessen. Die Aufgaben sollten nicht nur bzgl. Funktionen oder Leistungsmerkmale beschrieben werden,\\

die Umgebung, in der die Benutzer das System benutzen sollen. Die Umgebung schliesst die hardware, software und die zu ver- wendenden Materialien ein. Deren Beschreibung kann eine Aus- wahl von Produkten darstellen, von denen eines oder mehre-re den Schwerpunkt der menschzentrierten Spezifikation oder Beur- teilung bilden kann, oder sie kann aus einer Auswahl von Merk- malen oder Leistungseigenschaften der Hardware, Software oder sonstiger Materialien bestehen. Darüber hinaus gehören zu der Umgebung die physischen, organisatorischen und sozialen Rah- menbedingungen und Einflussgrössen.\\

Benutzer
\begin{itemize}
   \item 
   Altersklassen: Nicht exakt definierbar, grundsätzlich jeder der ein Smartphone besitzt und auf diese Art und Weise Reisen will. Kernzielgruppe wird zwischen 16 - 50 Jahre geschätzt. Prinzipiell könnten auch Personen darüber hinaus Interesse an der Applikation haben. Vorraussetzung ist lediglich die vorhandene Technologie. Vermutlich spricht der sozialere Aspekt diese aber eher weniger an und sind wahrscheinlich nicht mit Campingausrüstung längere Zeit unterwegs. Physische Merkmale befähigen sie als Mieter dazu, eine Reise aufzunehmen. Als Vermieter gibt es dahingehend keine Einschränkung.

   \item 
   Einkommen: Grundsätzlich sind Personen aller Einkommensklassen möglich. Aufgrund des Kostenfaktors wird die Kernzielgruppe jedoch hauptsächlich aus einer einkommensschwächeren oder besonders finanzbewusste Schicht stammen.

   \item 
   Erfahrung der Benutzer: Mit Erfahrung der Campingdomäne kann gerechnet werden, muss aber nicht zwangsläufig vorhanden sein.
   Erfahrene Leute in diesem Bereich, haben eventuell schon einige Apps ausprobiert und in Benutzung und können mit solchen Systemen sicher umgehen.
   Grundsätzlich wird davon ausgegangen, dass die Stakeholder mit einem Smartphone umgehen können, aber eventuell zum ersten Mal auf diesem Weg Reservierungen und Bezahlungen durchführen. 

   \item
   Fähigkeiten: Sprachkenntnisse können variieren, da auch ausländische Touristen die Software benutzen könnten.

   \item 
   Einstellung der Benutzer: In der Regel sozial offenere Menschen, eventuell naturgebundene Leute die auf der Durchreise sind bzw. an kürzeren Aufenthalten interessiert sind. 

   
\end{itemize}

\newpage

Aufgaben und Ziele
\begin{itemize}
   \item 
   Mieter: Finden eines geeigneten Schlafplatzes mit Hilfe des Smartphones. Dabei geht die Suchanfrage vom Reisenden aus und wird mit Hilfe des Systems verarbeitet. 
   \item 
   Vermieter: Erfolgreiches Vermieten eines Schlafplatzes. Relevante Anfragen vom System empfangen und nach eigenem Ermessen beantworten. 
   \item
   Sichere Informationsübertagung zwischen Kontaktpersonen.
   \item
   Sichere Abwicklung des (Ver-) Mietvorgangs mit anschließender Bezahlung.\\  


\end{itemize}


Arbeitsmittel
\begin{itemize}
   \item 
   Smartphone mit Applikation und Android Betriebssystem. (Im Sonderfall wäre eine Registrierung über Webpräsenz am Computer eine denkbare Option.) 
   \item  
   Internetanbindung im Vetrag mit Telefonanbieter oder wahlweise über Hotspots und verfügabren WLAN Netze.\\
   

\end{itemize}


physikalische Umfeld
\begin{itemize}

   \item 
   Umgebung: Reisende befinden sich an unterschiedlichen Gegenden. Die Verwendung kann im Freien oder auch in vorhandenen Gebäuden stattfinden. Dabei ist es nicht vorhersehbar, ob sie sich in einer Großstadt oder ländlicher Gegend befinden.

   \item
   Anbindung: Da Benutzer auch fernab von Hauptstraßen und angebundenen Orten reisen können, ist mit Netzproblemen zu rechnen. Zusätzlich sollte Rücksicht auf Akkuleistung genommen werden, da die Stromversorgung selten vorhanden sein wird. 

   \item
   Gepäck: Reisende führen relativ viel Gepäck mit sich und benötigen ihre Campingausrüstung. Sie sind daher in ihrer körperlichen Flexibilität etwas eingeschränkt.
\end{itemize}




\newpage

%!TEX root = ../dokumentation.tex

\subsection{Nutzungsanforderungen}
TODO

\subsubsection{Funktionale Anforderungen}        	
Im Vergleich zu obigem Beispiel sollte dabei beachtet werden, dass die Software auf einem Smartphone läuft und von dem Kontext ausgegangen wird, dass ein Benutzer auf seiner Strecke die entsprechenden Suchfunktionen nutzt. Zusätzlich dazu wurden auch Grundfunktionalitäten betrachtet, die nicht mit dem direkten Suchen zusammenhängen.

\begin{itemize}
   \item 
   Registrierung der Anwender und Speichern von relevanten Informationen.
   \item
   Erstellung und Bearbeitung eines Benutzerprofils/ Camp Profils.
   \item
   Erstellen eines Reiseprofils: geplante Reisezeit, gewünschte Ausstattung und Gruppengröße, anhand dessen das Matching stattfindet.
   \item
   Lokalisierung über GPS.
   \item 
   Suchen nach vorhanden Grundstücksanbietern und Rückmeldung über Suchtreffer.
   \item 
   Anzeige der Anfrage beim potentiellen Vermieter.
   \item
   Kontaktaufnahme und Kommunikation des Mieters und Vermieters (Google Cloud Messaging) über Nachrichten.
   \item
   Übersenden von Kontaktinformationen: Reiseinformationen des Mieters, sowie Grundstücksinformationen des Vermieters. 
   \item 
   Matchingfunktion anhand derer relevante Anfragen gefiltert werden.
   \item 
   (Eventuell) Lokale Speicherung der Kontaktinformationen zur GPS Benutzung oder bei Verbindungsabbruch.
   \item
   Bezahlfunktion über Applikation.
   \item
   Bewertung des Benutzers.
   \item
   Einblenden von Werbeaktionen

\end{itemize}

\newpage 

\subsubsection{Qualitative Anforderungen} 
Neben den funktionalen Anforderungen, ergaben sich auch erste nonfunktionale Anforderungen an das System.


Qualitätsattribute der gewünschten Funktionen,
Anforderungen an das implementierte System als 
Ausführungsverhalten (Verarbeitung unter Echtzeitbedingungen, Auslastung von Ressourcen, Genauigkeit, Antwortzeiten, Durchsatz, Speicherbedarf)

\begin{itemize}
   \item 
   \textbf{Zuverlässigkeit}: Funktionen müssen zuverlässig arbeiten und die Aktivitäten zielgerichtet unterstützen. Dazu gehört vorallem die genaue GPS Ortung, ein passendes Matchingsystem und die effektive Nachrichtenweiterleitung. Auch der Bezahlvorgang sollte problemlos funktionieren, da hierbei nachhaltiger Schaden bei Fehlern entstehen kann. 

   \item
   \textbf{Ausfallsicherheit}: Da die Anwendung verstärkt im freien Verwendung findet, muss die Anwendung auf Verbindungsausfälle reagieren. Dazu kann bei leerem Akku eine Unterbrechung während einer Aktivität stattfinden. In folge dessen, mussen die Information weiterhin unversehr bleiben.

   \item
   \textbf{Robustheit}: Die Anwendung muss, speziell für das Matching, sehr robust bei fehlerhaften Eingaben des Anwenders sein. Bei fehlerhafter Benutzung (z.B. durch Unerfahrenheit oder durch Auswahl einer falschen Option aufgrund von Sonneneinstrahlung/Sichteinschränkung), sollen für den Anwender keine Folgeschäden auftreten.

   \item
   \textbf{Usability}: Geforderte Funktionen müssen ausführbar sein und möglichst effektiv zum Ziel führen. Dabei sollte auch das physikalische Umfeld betrachtet werden, da der Anwender sich verstärkt im freien Aufhalten wird und unter zeitlichen Faktoren stehen kann.
   
   \item 
   \textbf{Effiziens}: Funktionen müssen die Suche gezielt unterstützen und eine sinnvolle Allokation der Arbeitsschritte muss vorhanden sein, um dem Benutzer möglichst viel Arbeit abzunehmen. (Mit möglichst geringem Aufwand an gewünschtes Ziel führen.)
   Die einzelnen Aktivitäten müssen möglichst schnell durchgeführt und Daten schnell weitergeleitet werden. Da mit Verbindungsabbrüchen zu rechnen ist, sollen einzelne Schritte während der Kommunikation zwischen Mieter und Vermieter zwischengespeichert werden.
   
   \item
   \textbf{Sicherheit}: Permanent gespeicherte Daten müssen sicher gespeichert und übertragen werden. 

   \item
   \textbf{Barrierefrei und Zugänglichkeit}: Applikation sollte so gestaltet sein, dass eine möglichst große Interessentengruppe angesprochen wird. Sie sollte daher einfach und effizient zu benutzen sein, leicht erlernbar und den User unterstützen. Von der  Gestaltung sollten auch Anwender mit visuellen Beeinträchtigungen unterstützt werden.
\end{itemize}



\subsubsection{Organisatorische Anforderungen}          
\begin{itemize}
   \item
   \textbf{TODO 1}

   \item 
   \textbf{TODO 2}

   \item
   \textbf{TODO 3}

\end{itemize}


\newpage



\newpage

%!TEX root = ../dokumentation.tex

\section{Bearbeitung der Proof of Concepts}


\newpage

%!TEX root = ../dokumentation.tex

\section{Überarbeitete Systemarchitektur}

Im Konzept wurde bereits ein Kommunikationsablauf\footnote{Konzeptkapitel 4.1 Seite 28} konstruiert. Aufgrund der weiteren Recherche zum Thema Google Cloud Messaging und dem erfolgreichen Abschluss der Proof-of-Concepts, konnte die angedachte Systemarchitektur überarbeitet werden.

Der Kommunikationsablauf ging davon aus, dass kein Rückkanal bei Google Cloud Messaging verfügbar ist. Auf Grund dieser Annahme wurde neben GCM eine weitere Schnittstelle eingeplant, welche auf dem REST Paradigma basierte. Die dadurch nur mögliche synchrone Interaktion ist jedoch eine großer Nachteil für die geplanten Features gewesen. Außerdem kann die Wartung von zwei Architekturen auf längere Zeit zu Problemem führen.\\

Durch die bidirektionale Verbindung (siehe Abb. \ref{fg:gcmccsarchitecture}) zwischen Client und Server bzw Server und Client kann der Google Dienst „GCM Cloud Connection Server“ die aufgeführte Problematik lösen.

\begin{figure}[H]
	\centering
	\includegraphics[width=0.85\textwidth]{./images/architekturneu.png}
	\caption{Systemarchitekur mit „GCM Cloud Connection Server“}
	\label{fg:gcmccsarchitecture}
\end{figure}


\newpage

%!TEX root = ../dokumentation.tex

\section{Entwickeltes System}

\subsection{Android Client}

\subsubsection{Activities}

\begin{figure}[H]
	\centering
	\includegraphics[width=0.85\textwidth]{./images/activitiesoverview.png}
	\caption{Übersicht der Activities und Navigationselementen}
	\label{fg:activitiesoverview}
\end{figure}

Activities sind die Komponenten, die dem Benutzer visuell Interaktionen anbieten können. Die Activities sollten untereinander verbunden werden, um so einen gewissen Workflow zu garantieren und „Sackgassen“ zu vermeiden. Die mit Android 3.0 eingeführte „Action Bar“ liefert eine gute Grundlage. Für die erste Version wurde die „Up Navigation“ implementiert. In der „Action Bar“ wird bei Aufruf einer untergeordneten Activity ein Pfeil neben dem Appicon angezeigt. Hierb ist es notwendig beim Deklarieren der Activity im Manifest die übergeordnete Activity mit anzugeben (Code \ref{ls:activities}).

\begin{lstlisting}[label=ls:activities,caption=Deklaration zweier Activities mit Referenz zu einer übergeordneten Activity,language=xml]
<application>
    <activity
        android:name="de.fhkoeln.gm.findyourcamp.app.SettingsActivity"
        android:label="@string/action_settings"
        android:parentActivityName="de.fhkoeln.gm.findyourcamp.app.MainActivity" >
        <meta-data
            android:name="android.support.PARENT_ACTIVITY"
            android:value="de.fhkoeln.gm.findyourcamp.app.MainActivity" />
    </activity>
    <activity
        android:name="de.fhkoeln.gm.findyourcamp.app.InsertRentalPropertyActivity"
        android:label="@string/action_insert_rental_property"
        android:parentActivityName="de.fhkoeln.gm.findyourcamp.app.MainActivity" >
        <meta-data
            android:name="android.support.PARENT_ACTIVITY"
            android:value="de.fhkoeln.gm.findyourcamp.app.MainActivity" />
    </activity>
</application>
\end{lstlisting}

\subsubsection{Datenbank}

Wie im Proof-of-Concept „Lokale Datenbankanbindung“ (vgl Kapitel 3.2.4) bereits dokumentiert, wird das System mit dem Datenbanksystem SQLite arbeiten. Im Folgenden wird nun auf die endgültige Implementierung im System eingegangen.

Zunächst wurde eine abstrakte Klasse \texttt{Table} entwickelt (Code \ref{ls:abstracttableclass}), welches nichts anders macht, als den Datenbanknamen sowie die aktuelle Datenbankversion beinhaltet. Die Klasse erweitert die Klasse \texttt{SQLiteOpenHelper}.

\begin{lstlisting}[label=ls:abstracttableclass,caption=Abstrakte Klasse \texttt{Table}]
public abstract class Table extends SQLiteOpenHelper {
	private static final String DATABASE_NAME = "data.db";
	private static final int DATABASE_VERSION = 1;

	public Table(Context context) {
		super(context, DATABASE_NAME, null, DATABASE_VERSION);
	}
}
\end{lstlisting}

Der Ausschnitt (Code \ref{ls:userstableclass}) der Implementation der \texttt{UserTable} demonstriert die Vererbung der Klasse \texttt{Table}.

\begin{lstlisting}[label=ls:userstableclass,caption=Ausschnitt aus der Klasse \texttt{UsersTable}]
public class UsersTable extends Table {

	public static final String TABLE_NAME = "users";
	public static final String COLUMN_NAME_ID = "id";
	public static final String COLUMN_NAME_USER_NAME = "name";

	private static final String TABLE_CREATE =
		"CREATE TABLE " + TABLE_NAME + " (" +
			COLUMN_NAME_ID + " integer primary key autoincrement," +
			COLUMN_NAME_USER_NAME + " text NOT NULL" +
		");";

	private static final String TABLE_DROP =
		"DROP TABLE IF EXISTS " + TABLE_NAME;

	public UsersTable(Context context) {
		super(context);
	}

	public void onCreate(SQLiteDatabase db) {
		db.execSQL(TABLE_CREATE);
	}
}
\end{lstlisting}

Hierbei hat sich als hilfreich erwiesen, Tabellenspezifische Angaben wie Tabellenname oder Spaltenamen in Konstanten auszulagern, um diese wiederverwertbar zu gestalten. Dies spiegelt sich zum Beispiel bei einem Insert-Befehl wieder (Code \ref{ls:userstableinsert}).

\begin{lstlisting}[label=ls:userstableinsert,caption=Exemplarisches Insert in die „UsersTable“]
UsersTable usersTable = new UsersTable(this);
SQLiteDatabase usersTableDb = usersTable.getWritableDatabase();

ContentValues values = new ContentValues();
values.put(UsersTable.COLUMN_NAME_ID, id);
values.put(UsersTable.COLUMN_NAME_USER_NAME, username);

long userId = usersTableDb.insert(
	UsersTable.COLUMN_NAME_ID,
	UsersTable.COLUMN_NAME_USER_NAME,
	values);
\end{lstlisting}

\subsubsection{Google Cloud Messaging}

Die Implementierung der GCM Schnittstelle hatte sich bereits im Proof-of-Concept als kompliziert erwiesen. Im Entwicklungssystem wurde dies wiederum bestätigt.

Ein aufgetretenes Problem war das fehlende Auslösen des Hintergrundprozesses, welches es unmöglich machte, die Push Notifcations zu bearbeiten. Eine Fehlerausgabe suchten wir vergeblich. Nach längeren Debugging wurde der recht banale Fehler gefunden: Es gab eine Komplikation bei den Berechtigungen in der Datei \texttt{AndroidManifest.xml} (Code \ref{ls:androidmanifest}).

\begin{lstlisting}[label=ls:androidmanifest,caption=Auszug aus der AndroidManifest.xml,language=xml]
<?xml version="1.0" encoding="utf-8"?>
<manifest xmlns:android="http://schemas.android.com/apk/res/android"
    package="de.fhkoeln.gm.findyourcamp.app"
    android:versionCode="1"
    android:versionName="1.0" >

    <permission android:name="de.fhkoeln.gm.findyourcamp.app.permission.C2D_MESSAGE" android:protectionLevel="signature" />
    <uses-permission android:name="de.fhkoeln.gm.findyourcamp.app.permission.C2D_MESSAGE" />
\end{lstlisting}

Der Fehler lag darin, dass der Wert für \texttt{package} nicht dem Präfix in Zeile 7 und 8 entsprach. Der korrekte Name der Berechtigung baut sich somit über \texttt{\{Paketname\}.permission.C2D\_MESSAGE} auf.


\subsubsection{Message Broker}

Über den GCM Dienst werden alle Daten als Nachrichten transportiert. Die einzelnen Nachrichten müssen aus diesem Grund einer eindeutigen Aktion zugeordnert werden können. Diese Aktion bestimmt dann die weitere Verarbeitung der Nachricht. Die Aktion wird im Payload der Nachricht unter dem Key \texttt{action} aufgeführt. Der Message Broker prüft dieses Feld und routet die Nachricht an den jeweiligen Controller weiter. Bei Bedarf kann dieser auch Nachrichten zunächst sammeln und als Ganzes weiterleiten.

Der Message Broker kommt jeweils in der Clientanwendung sowie in der Serveranwendung zum Einsatz. Das folgende Beispiel (Code \ref{ls:messagebroker}). zeigt einen Ausschnitt des Message Broker, welcher im Client zum Einsatz kommt.

\begin{lstlisting}[label=ls:messagebroker,caption=Intent Service mit Message Broker]
public class GcmIntentService extends IntentService {

    public GcmIntentService() {
        super("GcmIntentService");
    }

    @Override
    protected void onHandleIntent(Intent intent) {
        Bundle extras = intent.getExtras();
        GoogleCloudMessaging gcm = GoogleCloudMessaging.getInstance(this);
        String messageType = gcm.getMessageType(intent);

        if (!extras.isEmpty()) {  // has effect of unparcelling Bundle
            if (GoogleCloudMessaging.MESSAGE_TYPE_SEND_ERROR.equals(messageType)) {
               // ...
            } else if (GoogleCloudMessaging.MESSAGE_TYPE_DELETED.equals(messageType)) {
               // ...
            } else if (GoogleCloudMessaging.MESSAGE_TYPE_MESSAGE.equals(messageType)) {
            	MessageBroker mb = new MessageBroker(extras);
            	mb.handleRequest();
            }
        }
        GcmBroadcastReceiver.completeWakefulIntent(intent);
    }
}

public class MessageBroker {

	public static final int ACTION_RESPONSE_USER_REGISTER  = 1;
	public static final int ACTION_RESPONSE_SEARCH_REQUEST = 2;

	Bundle data;

	public MessageBroker(Bundle data) {
		this.data = data;
	}

	public void handleRequest() {
		int action = (Integer) data.get("action");

		switch (action) {
			case ACTION_RESPONSE_USER_REGISTER:
				// ...
				break;
			case ACTION_RESPONSE_SEARCH_REQUEST:
				// ...
				break;
			default:
				Logger.error("Unsupported action: " + action);
				break;
		}
	}
}
\end{lstlisting}

Dem Message Broker wird der Payload übergeben. Dieser prüft zunächst das Feld \texttt{action}, welches mit einem int-Wert belegt ist und leitet dann die weitere Verarbeitung ein.

\subsection{Serveranwendung}

\subsubsection{Datenbank}

Für die Speicherung von Daten wurden die Datenbanksysteme SQLite, MySQL und MongoDB in Erwägung gezogen. Die Einbindung in Java geschieht dabei über Treiber. Der zugehörige Treiber  muss dafür geladen werden und dem Projekt als Bibliothek zugewiesen werden.

Bei der Implementierung ist auf einen Fehler einzugehen: Das Laden des Laden des Treibers wird über einen dynamischen Klassenaufruf gesteuert. In den jeweiligen Anleitungen war folgender Schnipsel (Code \ref{ls:wrongdriveruse}) zu sehen (vgl. Verbindung mit MySQL über die Schnittstelle DriverManager\footnote{\url{http://dev.mysql.com/doc/refman/5.1/de/connector-j-usagenotes-basic.html\#connector-j-usagenotes-connect-drivermanager}, zuletzt gesichtet am 13. Januar 2014})

\begin{lstlisting}[label=ls:wrongdriveruse,caption=Fehlerhafter dynamischer Klassenaufruf]
try {
	Class.forName("com.mysql.jdbc.Driver").newInstance();
} catch (Exception ex) {
	// Fehler behandeln
}
\end{lstlisting}

Die Nutzung des Schnipsels gab jedoch nicht das gewünschte Ergebnis aus, genauer gesagt gar nichts. Die Lösung des Problems bestand darin, den Aufruf \textit{.newInstance()} zu entfernen (Code \ref{ls:correctdriveruse}).

\begin{lstlisting}[label=ls:correctdriveruse,caption=Korrekter dynamischer Klassenaufruf]
try {
	Class.forName("com.mysql.jdbc.Driver")
} catch (Exception ex) {
	// Fehler behandeln
}
\end{lstlisting}

In der folgenden Übersicht werden die drei Datenbanksysteme anhand eines Beispiels - anlegen einer Tabelle und einfügen eines Wertes - gegenübergestellt:

\paragraph{SQLite}

Treiber: \url{https://bitbucket.org/xerial/sqlite-jdbc}
\begin{lstlisting}[label=ls:sqliteexample,caption=Exemplarische Darstellung der Nutzung des SQLite Treibers]
Class.forName("org.sqlite.JDBC");
Connection connection = DriverManager.getConnection("jdbc:sqlite:findyourcamp.db");
Statement statement = con.createStatement();
statement.executeUpdate("drop table if exists users");
statement.executeUpdate("create table users (id integer, name string)");
statement.executeUpdate("insert into person values(1, 'max')");
\end{lstlisting}

\paragraph{MySQL}

Treiber: \url{http://dev.mysql.com/downloads/connector/j/}
\begin{lstlisting}[label=ls:mysqlexample,caption=Exemplarische Darstellung der Nutzung des MySQL Treibers]
Class.forName("com.mysql.jdbc.Driver");
Connection connection = DriverManager.getConnection("jdbc:mysql://localhost:3306/findyourcamp", "username", "password");
Statement statement = con.createStatement();
statement.executeUpdate("drop table if exists users");
statement.executeUpdate("create table users (id int, name text)");
statement.executeUpdate("insert into person values(1, 'max')");
\end{lstlisting}

\paragraph{MongoDB}

Treiber: \url{http://docs.mongodb.org/ecosystem/drivers/java/}
\begin{lstlisting}[label=ls:mongoexample,caption=Exemplarische Darstellung der Nutzung des MongoDB Treibers]
MongoClient mongoClient = new MongoClient();
DB db = mongoClient.getDB("findyourcamp");
DBCollection collection = db.getCollection("users");
collection.insert(new BasicDBObject("name", "max"));
\end{lstlisting}

SQLite kennen wir schon aus der Entwicklung der Android Applikation. Als weiteres relationales Datenbanksystem ist MySQL gewählt worden. Der Unterschied zwischen beiden System liegt darin, dass MySQL auf eine exterene Datenbasis setzt, wo SQLite sich direkt in die Anwendungen integrieren lässt. MongoDB hingegen kommt aus der NoSQL Sparte. Die Daten werden im JSON Format hinterlegt, wodurch das System gut skalierbar ist. Die Nachteile liegen allerdings in der fehlenden Relationen oder einer Volltext-Suche.

In den Tests (Code \ref{ls:performancetest}) zeigte sich außerdem, dass NoSQL, sowie auch SQLite, um 50\% langsamer als die MySQL Implementation war.

\begin{lstlisting}[label=ls:performancetest,caption=Perfomancetest zwischen MySQL und SQLite,language=bash]
$ time java -jar test_sqlite.jar
java -jar test_sqlite.jar 2,24s user 0,46s system  97% cpu 2,760 total
$ time java -jar test_mysql.jar
java -jar test_mysql.jar  1,47s user 0,17s system 133% cpu 1,225 total
\end{lstlisting}

Da der Zeitfaktor doch eine große Rolle in unserem System spielt, wurde entschieden für die Serveranwendung auf MySQL zu setzen. Jedoch sollte, sofern es die Zeit zulässt, eine gewisse Abstraktion geschaffen werden, wodurch ein späteres Ändern des Datenbanksystems durchführbar ist.

\subsection{Matching Algorithmus}

In Konzept wurde bereits kurz darauf eingegangen: Um den Vermieter vor einer Flut an Mietanfragen zu schützen, bekommt dieser nur eine Benachrichtigung bei relevanten Anfragen. Wann ist eine Anfrage releveant?

\begin{enumerate}
	\item \textbf{Ort:} Der Ort wird über einen serverseitigen Abgleich auf Relevanz geprüft. Zur Erinnerung: In den Datenbanken auf dem Server werden nur die Orte mit der jeweiligen Vermieter ID gespeichert. Entspricht ein Ort der Anfrage so wird dem Device des Vermieters die Anfrage weitergeleitet.
	\item \textbf{Gruppengröße:} Die Gruppengröße dient als Grundlage für den nächsten Abgleich. Ist die geforderte Gruppengröße kleiner oder gleich der verfügbaren Größe, ist das Objekt als relevant zu betrachten.
	\item \textbf{Preis:} Bevor die Ausstattung geprüft wird, wird zunächst ein Abgleich zwischen gewünschter und geforderter Preis durchgeführt. Das Objekt ist weiterhin relevant, wenn der gewünschte Preis dem des Vermieters entspricht.
	\item \textbf{Ausstattung:} Die Ausstattungsmerkmale werden als Boolean Werte verarbeitet. Dadurch ergeben sich nur die Werte 1 und 0.
	Die Relevanz kann daher auf Basis des Skalarproduktes ermittelt werden. Vektor A entspricht dabei den Daten des Vermieters, Vektor B dem des potentiellen Mieters. Wünscht der Mieter 4 Ausstattungsmerkmale und das Skalarprodukt ergibt zwei oder mehr, so ist das Objekt relevant und der Vermieter wird endgültig über die Anfrage benachrichtigt.
\end{enumerate}

Der relevante Code hierfür befindet sich in der Klasse \texttt{LocationMatch} im Paket \texttt{de.fhkoeln.gm.findyourcamp.server.matching} für das Server-Matching und das Client-Matching wird in der Klasse \texttt{RentalPropertyMatch} im Paket \texttt{de.fhkoeln.gm.findyourcamp.app.matching} sichtbar.



\newpage







	%Die Zähler für Tabellen und Abbildungen werden zurückgesetzt, damit
	%in jedem Kapitel die Nummerierung neu beginnt
	\setcounter{table}{1}
	\setcounter{figure}{1}
	%Einbinden des zweiten Kapitels
	
	%Installationsdokuemntation
	%!TEX root = ../dokumentation.tex

\chapter{Installationsdoku}

Anforderungen an ein System
• Installationsschritte
• Was muss Betreiber/Admin/etc. tun?

\newpage

%!TEX root = ../dokumentation.tex

\section{Anforderungen an ein System}



\newpage

%!TEX root = ../dokumentation.tex

\section{Installationsschritte}

Die Projekte für „Client“ und „Server“ liegen jeweils als Eclipse Projekt vor.\\

\paragraph{Client - Entwicklungsumgebung}

\begin{enumerate}
	\item Für die Entwicklungsumgebung der „Client“ Anwendung muss zunächst Eclipse IDE for Java Developers\footnote{\url{http://www.eclipse.org/downloads/moreinfo/java.php}, zuletzt gesichtet am 13. Januar 2014} installiert werden.
	\item Das aktuelle Software Development Kit\footnote{\url{http://developer.android.com/sdk/index.html\#ExistingIDE (Menü aufklappen)}, zuletzt gesichtet am 13. Januar 2014} muss geladen und entpackt werden
	\item Anschließend muss das „Android Development Tools“ Plugin\footnote{\url{http://developer.android.com/sdk/installing/installing-adt.html}, zuletzt gesichtet am 13. Januar 2014} für Eclipse heruntergeladen und installiert werden.
	\item Der im Plugin enthaltene SDK Manager muss dann so konfiguriert werden, dass er den Pfad zu dem bereits heruntergeladen SDK kennt.
	\item Nach der Konfigration müssen weitere Tools und Pakete geladen werden, siehe Abb. \ref{fg:android-sdk-manager}.
	\begin{figure}[H]
		\centering
		\includegraphics[width=0.85\textwidth]{./images/install/android-sdk-manager.png}
		\caption{SDK Manager: Pakete, die installiert werden müssen.}
		\label{fg:android-sdk-manager}
	\end{figure}
	\item Optional: Falls kein Testgerät vorliegt, welches den Anforderungen entspricht, kann ein „Android Virtual Device“ (AVD)\footnote{\url{http://developer.android.com/tools/devices/index.html}, zuletzt gesichtet am 13. Januar 2014} eingerichtet werden. In unseren Tests hat sich ein AVD auf Basis des Nexus One (Abb. \ref{fg:android-adv}) als hilfreich erwiesen. Wichtig ist, dass bei \textit{Target} ein Wert mit \textit{Google APIs} als Präfix ausgewählt wird.
	\begin{figure}[H]
		\centering
		\includegraphics[width=0.85\textwidth]{./images/install/android-avd.png}
		\caption{AVD auf Basis des Nexus One}
		\label{fg:android-adv}
	\end{figure}
	\item Die Applikation kann über das Kontextmenü des Projektes unter \textit{Run As -> Android Application} ausgeführt werden.
\end{enumerate}


\paragraph{Client - Device}

\begin{enumerate}
	\item Nachdem das Projekt einmal als Android-Anwendung im Emulator oder auf dem Testgerät ausgeführt wurde, befindet sich im \texttt{bin} Verzeichnis des Projekts die Datei \texttt{FindYourCamp.apk}. Diese kann kopiert werden und beispielsweise über Dropbox auf das Gerät kopiert werden
	\item Die APK kann dann auf dem Endgerät installiert werden. Zunächst muss dazu, falls nicht schon aktiviert, die Funktion zur manuellen Installation aktiviert werden. Dazu im Menu Einstellungen unter den Punkt \textit{Sicherheit} den Haken hinter \textit{Unbekannte Herkunft} setzen.
	\item Hinweis: Der Server muss für die vollständige Funktionalität laufen.
\end{enumerate}

\paragraph{Server}

\begin{enumerate}
	\item Die Serveranwendung benötigt einen MySQL Instanz. Die Installationsanweisung für ihre Systemungebung entnehmen Sie bitte dem offiziellen Handbuch: \url{http://dev.mysql.com/doc/refman/5.1/de/installing.html}
	\item Die Zugangsdaten sowie der Datenbankname ist in der Datei \texttt{DbConfig.java} im Paket \texttt{de.fhkoeln.gm.findyourcamp.server.db} zu hinterlegen.
	\item Für die Serveranwendung muss nun die \texttt{Main.java} im \texttt{de.fhkoeln.gm.findyourcamp.server} Paket als Java-Anwendung ausgeführt werden.
	\item Hinweis: Die Anwendung ist ohne GUI konzipiert. Die Ausgabe wird über die Konsole gesteuert.
\end{enumerate}


\newpage

	%Die Zähler für Tabellen und Abbildungen werden zurückgesetzt, damit
	%in jedem Kapitel die Nummerierung neu beginnt
	\setcounter{table}{1}
	\setcounter{figure}{1}
	%Einbinden des zweiten Kapitels

	%Fazit
	%!TEX root = ../dokumentation.tex

\chapter{Projektfazit}

Erfüllungsgrad
Bewertung des eigenen Prozesses




 
%Zeilenabstand 1 fach für die Verzeichnisse
\singlespacing
%Einbindne der Verzeichnisse
\include{inc/verzeichnisse}

%Zeilenabstand 1,5 fach für den Eid
\onehalfspacing
%Einbinden des Eides
%\include{inc/eid}

%Konzept als Anhang einbinden
%!TEX root = ../dokumentation.tex

\addcontentsline{toc}{chapter}{Anhang}

\section*{Anhang}
Folgende Artefakte liegen dem Dokument bei:

	überarbeitetes Konzept

  \newpage
  %Konzeptpdf einbinden
  \includepdf[pages=1-, scale=1]{../konzept/konzept.pdf}

	


\end{document}
