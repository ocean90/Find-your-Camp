%!TEX root = ../dokumentation.tex

\section{Aussicht der weiteren MCI Auseinandersetzung}
Innerhalb des durchgeführten Projektrahmens konnten zeitlich nicht alle MCI Vorgehen durchgeführt werden, die in der Regel zu einem erfolgreichen Entwicklungsprojekt beitragen sollen.\\

Das Projekt endete mit der Ausarbeitung eines überarbeiteten Prototypen. Da dieser nicht mehr mit realen Benutzern evaluiert werden konnte, wäre dieser Schritt die nächste Durchführung. Aufbauend auf den Ergebnissen die sich daraus ergeben, sollte der Übergang vom papierbasierten Prototypen zum computerbasierten Protoypen erfolgen. Dabei werden die Ergebnisse der Evaluation in einem neuen Redesign verarbeitet und mit stärkerer Betrachtung von Dialoggestaltungsgrundsätzen entwickelt.\\
Die konkrete Auseinandersetzung mit Aspekten zur Interfacegestaltung und vorhandenen Interaktionsparadigmen, -stilen, und modi, stellt ebenfalls eine Relevanz für die Entwicklung dar, konnte aber ebenfalls nicht mehr  realisiert werden.\\
Da die Entwicklung einer Smartphoneanwendung mit Touchscreen vorgesehen ist, sollte im weiteren Verlauf auch die Beschäftigung mit Touchscreenspezifischen Anforderungen geschehen.\\

Darüber hinaus sollte der Projektverlauf soweit gehen, bis die Gestaltungslösungen alle Nutzeranforderungen unterstützen und dies durch Evaluation mit realen Anwendern und anhand von definierten Messkriterien bestätigt wurde. 
Die Häufigkeit der Iterationsphasen kann dabei nicht genau vorhergesagt werden. 