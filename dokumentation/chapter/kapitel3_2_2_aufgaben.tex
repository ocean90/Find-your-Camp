%!TEX root = ../dokumentation.tex

\section{Aufgabenmodellierung}

Was sind die Aufgaben die sie erreichen wollen?\\
Was benötigen sie vom System und wie sollte das organisiert sein?\\
Welches sind die Aufgaben, welche Struktur weisen sie auf und in welchen Beziehungen stehen die Aufgaben zueinander?\\
Wie sieht ein valides Modell des Nutzungskontextes aus?\\

use case methoden: allistar cockburn template \\
concrete use case, lockwood\\


-> User Role/ Use Cases speziell für usage centered	\\
task model essentiell use cases, use case map ale aufeinander\\

task Model: struktur der Aufgaben\\


\subsection{Essentiel Use Cases}
Funktionale Anforderungsermittlung, interaktion zwischen Anwender und System, sehr abstrakt, grober Überblick, technologie frei\\
Formal und an Reichhaltigkeit orientiert an Software for Use S. 105

\begin{table}[H]
\caption{Essential Use Case \#1 registriereUser }
\centering
\begin{tabular}{l l}
\\ [-0.5ex]

\hline\hline
\\ [-0.5ex]
user intention & system responsibility
\\ [1.5ex]
\hline
\\ [-0.5ex]
Aktion initiieren			& 											\\[1ex]
							& Aktion anbieten							\\[1ex]
Informationen angeben 		& 											\\[1ex] 
							& Informationen aufnehmen					\\[1ex]
							& Informationen präsentieren				\\[1ex]
Informationen bestätigen	& 											\\[1ex]
							& Bestätigung anbieten, bei Auswahl akzeptieren  \\[1ex]


\hline
\end{tabular}
\label{tab:anmelden}
\end{table}

\begin{table}[H]
\caption{Essential Use Case \#2 verwalteProfil }
\centering
\begin{tabular}{l l}
\\ [-0.5ex]

\hline\hline
\\ [-0.5ex]
user intention & system responsibility
\\ [1.5ex]
\hline
\\ [-0.5ex]
Aktion initiieren			& 											 \\[1ex]
							& Aktion anbieten							 \\[1ex]
Profilart wählen			& 											 \\[1ex]
							& Auswahl anbieten							 \\[1ex]							
eingetragene Informationen	& 											 \\[1ex]
anzeigen					& 											 \\[1ex]
							& vorhandene Informationen präsentieren      \\[1ex] 
Informationen ändern 		& 											 \\[1ex] 
							& Aufnahme neuer Informationen ermöglichen	 \\[1ex]
							& alte Informationen entfernen				 \\[1ex]
Änderungen bestätigen		& 											 \\[1ex]
							& Bestätigung anbieten, bei Wahl akzeptieren \\[1ex]
\hline
\end{tabular}
\label{tab:profilbearbeiten}
\end{table}



\begin{table}[H]
\caption{Essential Use Case \#4 erstelleReiseprofil }
\centering
\begin{tabular}{l l}
\\ [-0.5ex]

\hline\hline
\\ [-0.5ex]
user intention & system responsibility
\\ [1.5ex]
\hline
\\ [-0.5ex]
Aktion auswählen 			& 											 \\[1ex]
							& Optionen bekanntmachen, Wahl ermöglichen	 \\[1ex]
Identität bestätigen		& 											 \\[1ex]
							& Identität prüfen							 \\[1ex]
Informationen anlegen 		& 											 \\[1ex] 
							& neue Informationen aufnehmen				 \\[1ex]
							& Informationen präsentieren				 \\[1ex]
Informationen bestätigen	& 											 \\[1ex]
							& Bestätigung anbieten, bei Wahl akzeptieren \\[1ex]

\hline
\end{tabular}
\label{tab:mietauftrag}
\end{table}

\begin{table}[H]
\caption{Essential Use Case \#5 sucheMietobjekt }
\centering
\begin{tabular}{l l}
\\ [-0.5ex]

\hline\hline
\\ [-0.5ex]
user intention & system responsibility
\\ [1.5ex]
\hline
\\ [-0.5ex]
Aktion auswählen 			& 											 \\[1ex]
							& Optionen bekanntmachen, Wahl ermöglichen	 \\[1ex]
Mietauftrag bekanntmachen	& 											 \\[1ex]
							& vorhandene Mietaufträge präsentieren		 \\[1ex]
							& Option für neuen Mietauftrag anbieten      \\[1ex]
Anfrage bestätigen   		& 											 \\[1ex] 
							& Anfrage akzeptieren						 \\[1ex]
							& Anfrage bearbeiten \\[1ex]
Antwort erhalten			& 											 \\[1ex]
							& Akteur über Antwort informieren			 \\[1ex]

\hline
\end{tabular}
\label{tab:mietobjekt}
\end{table}



\begin{table}[H]
\caption{Essential Use Case \#6 beantworteMietanfrage }
\centering
\begin{tabular}{l l}
\\ [-0.5ex]

\hline\hline
\\ [-0.5ex]
user intention & system responsibility
\\ [1.5ex]
\hline
\\ [-0.5ex]
Mietanfrage bekommen 		& 											 \\[1ex]
							& neue Mietanfrage präsentieren				 \\[1ex]
Mietanfrage einsehen		& 											 \\[1ex]
							& Informationen präsentieren				 \\[1ex]
Mietanfrage beantworten  	& 											 \\[1ex] 
							& Auswahl anbieten							 \\[1ex]
							& nächsten Schritt anbieten					 \\[1ex]
Informationen mitteilen		& 											 \\[1ex]
							& vorhandene Informationen anzeigen			 \\[1ex]
							& Informationen vermitteln					 \\[1ex]


\hline
\end{tabular}
\label{tab:mietanfrage}
\end{table}

\begin{table}[H]
\caption{Essential Use Case \#7 bezahleMiete }
\centering
\begin{tabular}{l l}
\\ [-0.5ex]

\hline\hline
\\ [-0.5ex]
user intention & system responsibility
\\ [1.5ex]
\hline
\\ [-0.5ex]
Aktion auswählen	 		& 											 \\[1ex]
							& Option anbieten							 \\[1ex]
Identität bestätigen		& 											 \\[1ex]
							& Identität prüfen							 \\[1ex]
Mietauftrag wählen		  	& 											 \\[1ex] 
							& Mietaufträge präsentieren, Wahl ermöglichen\\[1ex]
Bezahlauftrag starten		& 											 \\[1ex]
							& relevante Informationen präsentieren		 \\[1ex]
							& Informationsaufnahme anbieten	     		 \\[1ex]
Auftrag bestätigen			&	     									 \\[1ex]
							& Bestätigung anbieten				   		 \\[1ex]
							& Ergebnis präsentieren			    		 \\[1ex]

\hline
\end{tabular}
\label{tab:statuscodes}
\end{table}

\begin{table}[H]
\caption{Essential Use Case \#8 bewerteUser }
\centering
\begin{tabular}{l l}
\\ [-0.5ex]

\hline\hline
\\ [-0.5ex]
user intention & system responsibility
\\ [1.5ex]
\hline
\\ [-0.5ex]
Aktion auswählen	 		& 											 \\[1ex]
							& Option anbieten							 \\[1ex]
Identität bestätigen		& 											 \\[1ex]
							& Identität prüfen							 \\[1ex]
Mietauftrag wählen		  	& 											 \\[1ex] 
							& Mietaufträge präsentieren, Wahl ermöglichen\\[1ex]
Bezahlauftrag starten		& 											 \\[1ex]
							& relevante Informationen präsentieren		 \\[1ex]
							& Informationsaufnahme anbieten	     		 \\[1ex]
Auftrag bestätigen			&	     									 \\[1ex]
							& Bestätigung anbieten				   		 \\[1ex]
							& Ergebnis präsentieren			    		 \\[1ex]

\hline
\end{tabular}
\label{tab:statuscodes}
\end{table}

\begin{table}[H]
\caption{Essential Use Case \#9 kontaktiereUser }
\centering
\begin{tabular}{l l}
\\ [-0.5ex]

\hline\hline
\\ [-0.5ex]
user intention & system responsibility
\\ [1.5ex]
\hline
\\ [-0.5ex]
Aktion auswählen	 		& 											 \\[1ex]
							& Option anbieten							 \\[1ex]
Identität bestätigen		& 											 \\[1ex]
							& Identität prüfen							 \\[1ex]
Mietauftrag wählen		  	& 											 \\[1ex] 
							& Mietaufträge präsentieren, Wahl ermöglichen\\[1ex]
Bezahlauftrag starten		& 											 \\[1ex]
							& relevante Informationen präsentieren		 \\[1ex]
							& Informationsaufnahme anbieten	     		 \\[1ex]
Auftrag bestätigen			&	     									 \\[1ex]
							& Bestätigung anbieten				   		 \\[1ex]
							& Ergebnis präsentieren			    		 \\[1ex]

\hline
\end{tabular}
\label{tab:statuscodes}
\end{table}

\newpage
\subsection{Concrete Use Case}
Ein use case erfasst eine Beziehung zwischen stakeholder und dem technischen Subsystem bzgl. des Verhaltens des technischen Subsys- tems. Es beschreibt das Verhalten des technischen Subsystems unter bestimmten Rahmenbedingungen als Antwort auf eine Anfrage des stakeholders (primary actor). Der primary actor iniziiert die Interak- tion, um ein Ziel zu erreichen.

\begin{table}[H]
\caption{Use Case\#1 registriereUser }
\centering
\begin{tabular}{l l}
\\ [-0.5ex]

\hline\hline
\\ [-0.5ex]
user intention & system responsibility
\\ [1.5ex]
\hline
\\ [-0.5ex]
Funktion zur Neuanmeldung auswählen 				& 												\\[1ex]
													& Funktion bereitstellen, die dies ermöglicht	\\[1ex]
													& Über Richtlinien hinweisen 					\\[1ex]
Kenntnissnahme zu Richtlinien bestätigen			& 												\\[1ex]
													& Bestätigung ermöglichen						\\[1ex]
Informationen eingeben 								& 												\\[1ex] 
Personeninformationen Name, Vorname, Gebdat etc. 	& 												\\[1ex] 
													& Eingabefelder bereitstellen, Informationen    \\[1ex]
													& aufnehmen, weitere Schritte ermöglichen		\\[1ex]
Korrektheit der Daten prüfen und bestätigen			& 												\\[1ex]
													& Funktion zur Bestätigung anbieten 			\\[1ex]
													& weitere Funktionalitäten anbieten, Mietobjekt \\[1ex]
													& anlegen oder Mietauftrag anlegen				\\[1ex]
													& Informationen zur Verifizierung senden		\\[1ex]
Profil verifizieren									& 												\\[1ex]
													& TODO verifizierungsschritte (extra Use Case?)	\\[1ex]


\hline
\end{tabular}
\label{tab:anmeldenUC}
\end{table}

\begin{table}[H]
\caption{Use Case\#2 verwalteBenutzerprofil }
\centering
\begin{tabular}{l l}
\\ [-0.5ex]

\hline\hline
\\ [-0.5ex]
user intention & system responsibility
\\ [1.5ex]
\hline
\\ [-0.5ex]
Funktion zur Verwaltung auswählen  	& 												 	\\[1ex]
									& Funktion bereitstellen, die dies ermöglicht	 	\\[1ex]
Identität bestätigen				& 											     	\\[1ex]
									& Identitätsabfrage einleiten und mit Eingabe    	\\[1ex]
									& prüfen, anschließend bereits vorhandene 		 	\\[1ex] 
									& Informationen anzeigen     				     	\\[1ex] 
Neuen Informationen eingeben 		& 											     	\\[1ex] 
									& neue Informationen über Eingabefelder aufnehmen, 	\\[1ex]
									& auf syntaktische Korrektheit prüfen				\\[1ex]
Neue Informationen abspeichern		& 											 		\\[1ex]
									& Funktion zum abspeichern anbieten					\\[1ex]
\hline
\end{tabular}
\label{tab:profilbearbeitenUC}
\end{table}

\begin{table}[H]
\caption{Use Case\#3 erstelleReiseprofil }
\centering
\begin{tabular}{l l}
\\ [-0.5ex]

\hline\hline
\\ [-0.5ex]
user intention & system responsibility
\\ [1.5ex]
\hline
\\ [-0.5ex]
Funktion zum neuen Mietauftrag wählen 		& 												\\[1ex]
											& Funktion bereitstellen, die dies ermöglicht	\\[1ex]
Profil des Reisenden auswählen über den		& 												\\[1ex]
gemietet werden soll          				& 												\\[1ex]
											& Anmeldung mit Benutzerdaten ermöglichen		\\[1ex]
Organisatorische Informationen eintragen	& 											 	\\[1ex] 
											& Eingabefelder bereitstellen und 				\\[1ex]
											& Informationen aufnehmen 						\\[1ex]
Gewünschte Austattung auswählen				& 					 							\\[1ex]
											& Austattungsmerkmale anzeigen und 				\\[1ex]
											& Auswahl über Menü ermöglichen 				\\[1ex]
Informationen bestätigen und abspeichern 	& 												\\[1ex]
										 	& Funktion zum abspeichern anbieten 			\\[1ex]

\hline
\end{tabular}
\label{tab:mietauftragUC}
\end{table}

\begin{table}[H]
\caption{Use Case\#4 findeMietobjekt }
\centering
\begin{tabular}{l l}
\\ [-0.5ex]

\hline\hline
\\ [-0.5ex]
user intention & system responsibility
\\ [1.5ex]
\hline
\\ [-0.5ex]
Funktion zur neuen Suchanfrage starten 	& 											 	\\[1ex]
										& Funktion bereitstellen, die dies ermöglicht	\\[1ex]
Passenden Mietauftrag zur Reise wählen	& 												\\[1ex]
										& Vorhandene angelegte Mietaufträge anzeigen	\\[1ex]
										& und Auswahl ermöglichen, Option zum anlegen   \\[1ex]
										& eines neuen Auftrags anzeigen					\\[1ex]
Suchanfrage ausführen 					& 												\\[1ex] 
										& Ausgewählten Auftrag anzeigen	und Suchanfrage	\\[1ex]
										& funktional ermöglichen, nach Bestätigung 		\\[1ex]
										& GPS Daten bestimmen und Anfrage weiterleiten.	\\[1ex]
										& Benutzer über getätigte Suche informieren		\\[1ex]
Rückmeldung zu vorhandenen Angeboten	& 												\\[1ex]
erhalten								& 												\\[1ex]
										& Wenn kein Angebot vorhanden ist, direkte		\\[1ex]
										& Rückmeldung. Ansonsten regelmäßige Abfrage 	\\[1ex]
										& an Server, ob Vermieter angenommen hat.		\\[1ex]
										& Meldung an Benutzer schicken bei Treffer 		\\[1ex]
										& nach Ablauf einer kritischen Zeit.				\\[1ex]
\hline
\end{tabular}
\label{tab:mietobjektUC}
\end{table}

\begin{table}[H]
\caption{Use Case\#5 beantworteMietanfrage }
\centering
\begin{tabular}{l l}
\\ [-0.5ex]

\hline\hline
\\ [-0.5ex]
user intention & system responsibility
\\ [1.5ex]
\hline
\\ [-0.5ex]
Meldung über relevante Mietanfrage 	& 												\\[1ex]
erhalten 							& 												\\[1ex]
									& Mietauftrag empfangen und lokal matchen		\\[1ex]
									& bei Bestätigung Meldung mit Informationen		\\[1ex]
									& auf dem Display hervorheben, Benachrichtigen	\\[1ex]
Mieterdaten ansehen					& 											 	\\[1ex]
									& Reisedaten anzeigen: Datum Gruppengröße		\\[1ex]
									& Preisvorstellung, Anfragezeit					\\[1ex]
Mietanfrage beantworten  			& 												\\[1ex] 
									& Funktion anbieten zur Annahme oder Ablehnung 	\\[1ex]
									& der Anfrage, Meldung an Mieter schicken      	\\[1ex]
Annahme des Auftrags erhalten		& 											 	\\[1ex]
									& Bestätigung des Mieters empfangen, Auswahl	\\[1ex]
									& dem Vermieter anzeigen und ggf. weitere 		\\[1ex]
									& Funktion zum Senden der Objektinformationen	\\[1ex]
Objektinformationen verschicken		& 											 	\\[1ex]
Funktionsausführung					& 											 	\\[1ex]



\hline
\end{tabular}
\label{tab:mietanfrageUC}
\end{table}

\begin{table}[H]
\caption{Use Case\#6 registriereMietobjekt }
\centering
\begin{tabular}{l l}
\\ [-0.5ex]

\hline\hline
\\ [-0.5ex]
user intention & system responsibility
\\ [1.5ex]
\hline
\\ [-0.5ex]
Funktion zur Objektverwaltung auswählen		& 												\\[1ex]
											& Funktion bereitstellen, die dies ermöglicht	\\[1ex]
Auswahl ob altes Objekt bearbeiten oder		& 												\\[1ex]
neues Anlegen         						& 												\\[1ex]
											& Funktion bereitstellen, die dies ermöglicht	\\[1ex]
Neues Mietobjekt registrieren				& 											 	\\[1ex] 
Besitzer festlegen							& 											 	\\[1ex] 
											& Verifizierung des Besitzers durch Accountdaten \\[1ex]
											& Eingabefelder bereitstellen und prüfen 		\\[1ex]
Objektstandort bestimmten					& 					 							\\[1ex]
											& Funktion zur Standortbestimmung bereitstellen  \\[1ex]	
											& Kartenanbindung mit Stecknadel?			     \\[1ex]	
Objektmerkmale wie Größe, Preis eingeben	& 					 							\\[1ex]
											& Eingabefelder anzeigen und Eingabe über  		\\[1ex]
											& vordefinierte Optionen ermöglichen			\\[1ex]
Gewünschte Austattung auswählen				& 					 							\\[1ex]
											& Austattungsmerkmale anzeigen und 				\\[1ex]
											& Auswahl über Menü ermöglichen 				\\[1ex]
Informationen kontrollieren, bearbeiten 	& 												\\[1ex]
und abspeichern 							& 												\\[1ex]
										 	& Funktionalität zur Navigation bereitstellen	\\[1ex]
										 	& eingegebene Daten zeigen und speichern 		\\[1ex]
										 	& funktional ermöglichen						\\[1ex]



\hline
\end{tabular}
\label{tab:mietobjektAUC}
\end{table}


\subsection{HTA?}

zu Nutzungskontext\\
die Ziele und Arbeitsaufgaben der Benutzer. Die Beschreibung sollte Gesamtziele für die Verwendung des Systems enthalten. Die Merkmale jener Aufgaben, die Gebrauchstauglichkeit be- einflussen können, sollten beschrieben werden ebenso wie ggf. vorhandene Auswirkungen auf die Gesundheit und Sicherheit. Beschreibung sollte die Verteilung der Aktivitäten und Arbeits- schritte zwischen Mensch und technischen Hilfsmitteln einsch- liessen. Die Aufgaben sollten nicht nur bzgl. Funktionen oder Leistungsmerkmale beschrieben werden,\\

die Umgebung, in der die Benutzer das System benutzen sollen. Die Umgebung schliesst die hardware, software und die zu ver- wendenden Materialien ein. Deren Beschreibung kann eine Aus- wahl von Produkten darstellen, von denen eines oder mehre-re den Schwerpunkt der menschzentrierten Spezifikation oder Beur- teilung bilden kann, oder sie kann aus einer Auswahl von Merk- malen oder Leistungseigenschaften der Hardware, Software oder sonstiger Materialien bestehen. Darüber hinaus gehören zu der Umgebung die physischen, organisatorischen und sozialen Rah- menbedingungen und Einflussgrössen.\\


Anforderungsdokument:\\
Funktionale Anforderungen\\
Eingaben und deren Einschränkungen\\
Funktionen\\
Ausgabe\\

Nichtfunktionale Anforderungen\\
Qualitätsattribute an Funktionen\\
Anforderungen an implementiertes System als Ganzes\\





