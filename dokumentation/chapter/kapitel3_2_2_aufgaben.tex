%!TEX root = ../dokumentation.tex

\section{Aufgabenmodellierung}

Was sind die Aufgaben die sie erreichen wollen?\\
Was benötigen sie vom System und wie sollte das organisiert sein?\\

use case methoden: allistar cockburn template \\
concrete use case, lockwood\\


-> User Role/ Use Cases speziell für usage centered	\\
task model essentiell use cases, use case map ale aufeinander\\

task Model: struktur der Aufgaben\\


\subsection{Essentiel Use Cases}
Funktionale Anforderungsermittlung, interaktion zwischen Anwender und System, sehr abstrakt, grober Überblick, technologie frei\\
Formal und an Reichhaltigkeit orientiert an Software for Use S. 105

\begin{table}[H]
\caption{\#1 Anwender bekanntmachen }
\centering
\begin{tabular}{l l}
\\ [-0.5ex]

\hline\hline
\\ [-0.5ex]
user intention & system responsibility
\\ [1.5ex]
\hline
\\ [-0.5ex]
Aktion auswählen 			& 											\\[1ex]
							& Optionen bekanntmachen, Wahl ermöglichen	\\[1ex]
Informationen angeben 		& 											\\[1ex] 
							& Informationen aufnehmen					\\[1ex]
							& Informationen präsentieren				\\[1ex]
Informationen bestätigen	& 											\\[1ex]
							& Bestätigung anbieten, bei Wahl akzeptieren \\[1ex]


\hline
\end{tabular}
\label{tab:anmelden}
\end{table}

\begin{table}[H]
\caption{\#2 Benutzerprofil verwalten }
\centering
\begin{tabular}{l l}
\\ [-0.5ex]

\hline\hline
\\ [-0.5ex]
user intention & system responsibility
\\ [1.5ex]
\hline
\\ [-0.5ex]
Aktion auswählen 			& 											 \\[1ex]
							& Optionen bekanntmachen, Wahl ermöglichen	 \\[1ex]
Identität bestätigen		& 											 \\[1ex]
							& Identität prüfen							 \\[1ex]
							& vorhandene Informationen präsentieren      \\[1ex] 
Informationen ändern 		& 											 \\[1ex] 
							& neue Informationen aufnehmen				 \\[1ex]
							& alte Informationen entfernen				 \\[1ex]
Informationen bestätigen	& 											 \\[1ex]
							& Bestätigung anbieten, bei Wahl akzeptieren \\[1ex]
\hline
\end{tabular}
\label{tab:profilbearbeiten}
\end{table}

\begin{table}[H]
\caption{\#3 Mietauftrag erstellen }
\centering
\begin{tabular}{l l}
\\ [-0.5ex]

\hline\hline
\\ [-0.5ex]
user intention & system responsibility
\\ [1.5ex]
\hline
\\ [-0.5ex]
Aktion auswählen 			& 											 \\[1ex]
							& Optionen bekanntmachen, Wahl ermöglichen	 \\[1ex]
Identität bestätigen		& 											 \\[1ex]
							& Identität prüfen							 \\[1ex]
Informationen anlegen 		& 											 \\[1ex] 
							& neue Informationen aufnehmen				 \\[1ex]
							& Informationen präsentieren				 \\[1ex]
Informationen bestätigen	& 											 \\[1ex]
							& Bestätigung anbieten, bei Wahl akzeptieren \\[1ex]

\hline
\end{tabular}
\label{tab:mietauftrag}
\end{table}

\begin{table}[H]
\caption{\#4 Mietobjekt finden }
\centering
\begin{tabular}{l l}
\\ [-0.5ex]

\hline\hline
\\ [-0.5ex]
user intention & system responsibility
\\ [1.5ex]
\hline
\\ [-0.5ex]
Aktion auswählen 			& 											 \\[1ex]
							& Optionen bekanntmachen, Wahl ermöglichen	 \\[1ex]
Mietauftrag bekanntmachen	& 											 \\[1ex]
							& vorhandene Mietaufträge präsentieren		 \\[1ex]
							& Option für neuen Mietauftrag anbieten      \\[1ex]
Anfrage bestätigen   		& 											 \\[1ex] 
							& Anfrage akzeptieren						 \\[1ex]
							& Anfrage bearbeiten \\[1ex]
Antwort erhalten			& 											 \\[1ex]
							& Akteur über Antwort informieren			 \\[1ex]

\hline
\end{tabular}
\label{tab:mietobjekt}
\end{table}

\begin{table}[H]
\caption{\#5 Mietanfrage beantworten }
\centering
\begin{tabular}{l l}
\\ [-0.5ex]

\hline\hline
\\ [-0.5ex]
user intention & system responsibility
\\ [1.5ex]
\hline
\\ [-0.5ex]
Mietanfrage bekommen 		& 											 \\[1ex]
							& neue Mietanfrage präsentieren				 \\[1ex]
Mietanfrage einsehen		& 											 \\[1ex]
							& Informationen präsentieren				 \\[1ex]
Mietanfrage beantworten  	& 											 \\[1ex] 
							& Auswahl anbieten							 \\[1ex]
							& nächsten Schritt anbieten					 \\[1ex]
Informationen mitteilen		& 											 \\[1ex]
							& vorhandene Informationen anzeigen			 \\[1ex]
							& Informationen vermitteln					 \\[1ex]


\hline
\end{tabular}
\label{tab:mietanfrage}
\end{table}

\begin{table}[H]
\caption{\#6 Bezahlung }
\centering
\begin{tabular}{l l}
\\ [-0.5ex]

\hline\hline
\\ [-0.5ex]
user intention & system responsibility
\\ [1.5ex]
\hline
\\ [-0.5ex]
Aktion auswählen	 		& 											 \\[1ex]
							& Option anbieten							 \\[1ex]
Identität bestätigen		& 											 \\[1ex]
							& Identität prüfen							 \\[1ex]
Mietauftrag wählen		  	& 											 \\[1ex] 
							& Mietaufträge präsentieren, Wahl ermöglichen\\[1ex]
Bezahlauftrag starten		& 											 \\[1ex]
							& relevante Informationen präsentieren		 \\[1ex]
							& Informationsaufnahme anbieten	     		 \\[1ex]
Auftrag bestätigen			&	     									 \\[1ex]
							& Bestätigung anbieten				   		 \\[1ex]
							& Ergebnis präsentieren			    		 \\[1ex]

\hline
\end{tabular}
\label{tab:statuscodes}
\end{table}

\newpage
\subsection{Concrete Use Case}
Ein use case erfasst eine Beziehung zwischen stakeholder und dem technischen Subsystem bzgl. des Verhaltens des technischen Subsys- tems. Es beschreibt das Verhalten des technischen Subsystems unter bestimmten Rahmenbedingungen als Antwort auf eine Anfrage des stakeholders (primary actor). Der primary actor iniziiert die Interak- tion, um ein Ziel zu erreichen.

\begin{table}[H]
\caption{Use Case\#1 Benutzer registrieren }
\centering
\begin{tabular}{l l}
\\ [-0.5ex]

\hline\hline
\\ [-0.5ex]
user intention & system responsibility
\\ [1.5ex]
\hline
\\ [-0.5ex]
Funktion zur Neuanmeldung auswählen 				& 												\\[1ex]
													& Funktion bereitstellen, die dies ermöglicht	\\[1ex]
													& Über Richtlinien hinweisen 					\\[1ex]
Kenntnissnahme zu Richtlinien bestätigen			& 												\\[1ex]
													& Bestätigung ermöglichen						\\[1ex]
Informationen eingeben 								& 												\\[1ex] 
Personeninformationen Name, Vorname, Gebdat etc. 	& 												\\[1ex] 
													& Eingabefelder bereitstellen, Informationen    \\[1ex]
													& aufnehmen, weitere Schritte ermöglichen		\\[1ex]
Korrektheit der Daten prüfen und bestätigen			& 												\\[1ex]
													& Funktion zur Bestätigung anbieten 			\\[1ex]
													& weitere Funktionalitäten anbieten, Mietobjekt \\[1ex]
													& anlegen oder Mietauftrag anlegen				\\[1ex]


\hline
\end{tabular}
\label{tab:anmeldenUC}
\end{table}

\begin{table}[H]
\caption{Use Case\#2 Benutzerprofil verwalten }
\centering
\begin{tabular}{l l}
\\ [-0.5ex]

\hline\hline
\\ [-0.5ex]
user intention & system responsibility
\\ [1.5ex]
\hline
\\ [-0.5ex]
Funktion zur Verwaltung auswählen  	& 												 	\\[1ex]
									& Funktion bereitstellen, die dies ermöglicht	 	\\[1ex]
Identität bestätigen				& 											     	\\[1ex]
									& Identitätsabfrage einleiten und mit Eingabe    	\\[1ex]
									& prüfen, anschließend bereits vorhandene 		 	\\[1ex] 
									& Informationen anzeigen     				     	\\[1ex] 
Neuen Informationen eingeben 		& 											     	\\[1ex] 
									& neue Informationen über Eingabefelder aufnehmen, 	\\[1ex]
									& auf syntaktische Korrektheit prüfen				\\[1ex]
Neue Informationen abspeichern		& 											 		\\[1ex]
									& Funktion zum abspeichern anbieten					\\[1ex]
\hline
\end{tabular}
\label{tab:profilbearbeitenUC}
\end{table}

\begin{table}[H]
\caption{Use Case\#3 Mietauftrag erstellen }
\centering
\begin{tabular}{l l}
\\ [-0.5ex]

\hline\hline
\\ [-0.5ex]
user intention & system responsibility
\\ [1.5ex]
\hline
\\ [-0.5ex]
Funktion zum neuen Mietauftrag wählen 		& 												\\[1ex]
											& Funktion bereitstellen, die dies ermöglicht	\\[1ex]
Profil des Reisenden auswählen über den		& 												\\[1ex]
gemietet werden soll          				& 												\\[1ex]
											& Anmeldung mit Benutzerdaten ermöglichen		\\[1ex]
Organisatorische Informationen eintragen	& 											 	\\[1ex] 
											& Eingabefelder bereitstellen und 				\\[1ex]
											& Informationen aufnehmen 						\\[1ex]
Gewünschte Austattung auswählen				& 					 							\\[1ex]
											& Austattungsmerkmale anzeigen und 				\\[1ex]
											& Auswahl über Menü ermöglichen 				\\[1ex]
Informationen bestätigen und abspeichern 	& 												\\[1ex]
										 	& Funktion zum abspeichern anbieten 			\\[1ex]

\hline
\end{tabular}
\label{tab:mietauftragUC}
\end{table}

\begin{table}[H]
\caption{Use Case\#4 Mietobjekt finden }
\centering
\begin{tabular}{l l}
\\ [-0.5ex]

\hline\hline
\\ [-0.5ex]
user intention & system responsibility
\\ [1.5ex]
\hline
\\ [-0.5ex]
Aktion auswählen 			& 											 \\[1ex]
							& Optionen bekanntmachen, Wahl ermöglichen	 \\[1ex]
Mietauftrag bekanntmachen	& 											 \\[1ex]
							& vorhandene Mietaufträge präsentieren		 \\[1ex]
							& Option für neuen Mietauftrag anbieten      \\[1ex]
Anfrage bestätigen   		& 											 \\[1ex] 
							& Anfrage akzeptieren						 \\[1ex]
							& Anfrage bearbeiten 						\\[1ex]
Antwort erhalten			& 											 \\[1ex]
							& Akteur über Antwort informieren			 \\[1ex]

\hline
\end{tabular}
\label{tab:mietobjektUC}
\end{table}

\begin{table}[H]
\caption{Use Case\#5 Mietanfrage beantworten }
\centering
\begin{tabular}{l l}
\\ [-0.5ex]

\hline\hline
\\ [-0.5ex]
user intention & system responsibility
\\ [1.5ex]
\hline
\\ [-0.5ex]
Mietanfrage bekommen 		& 											 \\[1ex]
							& neue Mietanfrage präsentieren				 \\[1ex]
Mietanfrage einsehen		& 											 \\[1ex]
							& Informationen präsentieren				 \\[1ex]
Mietanfrage beantworten  	& 											 \\[1ex] 
							& Auswahl anbieten							 \\[1ex]
							& nächsten Schritt anbieten					 \\[1ex]
Informationen mitteilen		& 											 \\[1ex]
							& vorhandene Informationen anzeigen			 \\[1ex]
							& Informationen vermitteln					 \\[1ex]


\hline
\end{tabular}
\label{tab:mietanfrageUC}
\end{table}

\begin{table}[H]
\caption{Use Case\#6 Bezahlung }
\centering
\begin{tabular}{l l}
\\ [-0.5ex]

\hline\hline
\\ [-0.5ex]
user intention & system responsibility
\\ [1.5ex]
\hline
\\ [-0.5ex]
Aktion auswählen	 		& 											 \\[1ex]
							& Option anbieten							 \\[1ex]
Identität bestätigen		& 											 \\[1ex]
							& Identität prüfen							 \\[1ex]
Mietauftrag wählen		  	& 											 \\[1ex] 
							& Mietaufträge präsentieren, Wahl ermöglichen\\[1ex]
Bezahlauftrag starten		& 											 \\[1ex]
							& relevante Informationen präsentieren		 \\[1ex]
							& Informationsaufnahme anbieten	     		 \\[1ex]
Auftrag bestätigen			&	     									 \\[1ex]
							& Bestätigung anbieten				   		 \\[1ex]
							& Ergebnis präsentieren			    		 \\[1ex]

\hline
\end{tabular}
\label{tab:statuscodesUC}
\end{table}



\subsection{Persona Use Cases ?}
\subsection{HTA?}

