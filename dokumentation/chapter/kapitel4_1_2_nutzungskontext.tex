%!TEX root = ../dokumentation.tex

\subsection{Nutzungskontext}
TODO

Benutzer
\begin{itemize}
   \item 
   Altersklassen: Nicht exakt definierbar, grundsätzlich jeder der ein Smartphone besitzt und auf diese Art und Weise Reisen will. Kernzielgruppe wird zwischen 16 - 50 Jahre geschätzt. Prinzipiell könnten auch Personen darüber hinaus Interesse an der Applikation haben. Vorraussetzung ist lediglich die vorhandene Technologie. Vermutlich spricht der sozialere Aspekt diese aber eher weniger an und sind wahrscheinlich nicht mit Campingausrüstung längere Zeit unterwegs. Physische Merkmale befähigen sie als Mieter dazu, eine Reise aufzunehmen. Als Vermieter gibt es dahingehend keine Einschränkung.

   \item 
   Einkommen: Grundsätzlich sind Personen aller Einkommensklassen möglich. Aufgrund des Kostenfaktors wird die Kernzielgruppe jedoch hauptsächlich aus einer einkommensschwächeren oder besonders finanzbewusste Schicht stammen.

   \item 
   Erfahrung der Benutzer: Mit Erfahrung der Campingdomäne kann gerechnet werden, muss aber nicht zwangsläufig vorhanden sein.
   Erfahrene Leute in diesem Bereich, haben eventuell schon einige Apps ausprobiert und in Benutzung und können mit solchen Systemen sicher umgehen.
   Grundsätzlich wird davon ausgegangen, dass die Stakeholder mit einem Smartphone umgehen können, aber eventuell zum ersten Mal auf diesem Weg Reservierungen und Bezahlungen durchführen. 

   \item
   Fähigkeiten: Sprachkenntnisse können variieren, da auch ausländische Touristen die Software benutzen könnten.

   \item 
   Einstellung der Benutzer: In der Regel sozial offenere Menschen, eventuell naturgebundene Leute die auf der Durchreise sind bzw. an kürzeren Aufenthalten interessiert sind. 

   
\end{itemize}

\newpage

Aufgaben und Ziele
\begin{itemize}
   \item 
   Mieter: Finden eines geeigneten Schlafplatzes mit Hilfe des Smartphones. Dabei geht die Suchanfrage vom Reisenden aus und wird mit Hilfe des Systems verarbeitet. 
   \item 
   Vermieter: Erfolgreiches Vermieten eines Schlafplatzes. Relevante Anfragen vom System empfangen und nach eigenem Ermessen beantworten. 
   \item
   Sichere Informationsübertagung zwischen Kontaktpersonen.
   \item
   Sichere Abwicklung des (Ver-) Mietvorgangs mit anschließender Bezahlung.\\  


\end{itemize}


Arbeitsmittel
\begin{itemize}
   \item 
   Smartphone mit Applikation und Android Betriebssystem. (Im Sonderfall wäre eine Registrierung über Webpräsenz am Computer eine denkbare Option.) 
   \item  
   Internetanbindung im Vetrag mit Telefonanbieter oder wahlweise über Hotspots und verfügabren WLAN Netze.\\
   

\end{itemize}


physikalische Umfeld
\begin{itemize}

   \item 
   Umgebung: Reisende befinden sich an unterschiedlichen Gegenden. Die Verwendung kann im Freien oder auch in vorhandenen Gebäuden stattfinden. Dabei ist es nicht vorhersehbar, ob sie sich in einer Großstadt oder ländlicher Gegend befinden.

   \item
   Anbindung: Da Benutzer auch fernab von Hauptstraßen und angebundenen Orten reisen können, ist mit Netzproblemen zu rechnen. Zusätzlich sollte Rücksicht auf Akkuleistung genommen werden, da die Stromversorgung selten vorhanden sein wird. 

   \item
   Gepäck: Reisende führen relativ viel Gepäck mit sich und benötigen ihre Campingausrüstung. Sie sind daher in ihrer körperlichen Flexibilität etwas eingeschränkt.
\end{itemize}


