%!TEX root = ../dokumentation.tex

\chapter{Projektreflektion}
TODO

Erfüllungsgrad
Bewertung des eigenen Prozesses


Erfüllungsgrad detailliert diskutiert\\ 
beispielsweise die Entwurfsziele, die Systemarchitektur und -technik und die Nutzbarkeit, zu berücksichtigen. Zudem soll ein Fazit gezogen und ein Ausblick gegeben werden, was evtl. an dem System zu ändern oder ergänzen wäre\\
inwiefern sich der reale Zeitplan von dem aus der Konzeptionsphase unterscheidet

Zum Zeitpunkt der Konzeptabgabe stand ein Projektplan mit groben Vorstellungen zum weiteren Projektverlauf.\footnote{Konzeptseite 43} Neben den organisatorischen Meilensteinen zu Abgaben der Artefakte, wurden 2 projektspezifische Meilensteine definiert. Der erste Meilenstein sollte am 25. November erreicht werden und die Bearbeitung des MCI Vorgehens und die Bearbeitung der Proof-of-Concepts beinhalten. 
Anschließend war geplant die Implementation der reinen Funktionalitäten durchzuführen und bis zum 2. Meilenstein am 20. Dezember damit fertig zu werden. Die anschließende Zeit sollte der Gestaltung des Interface gewidmet werden, was wiederrum Gestaltungslösungen der MCI beinhaltet hätte. \\
Bereits nach kurzer Zeit innerhalb der Dokumentationsphase wurde bewusst, dass speziell die Planung der MCI Beschäftigung viel zu knapp bemessen war und sich über den gesamten Projektzeitraum erstrecken sollte. Aufgrund des iterativen Charakters, sollte die Phase bis zum ersten Meilenstein eher als erste Iterationsstufe angesehen werden. Die Überarbeitung des Konzeptes beanspruchte mehr Zeit als anfänglich eingeplant, was zur Folge hatte, dass die eingeplante Zeit für MCI Aspekte weiterhin verkürzt wurde. Als Reaktion darauf, wurde die erste Iterationsphase der MCI Auseinandersetzung bis zum ersten Meilenstein geplant. 

