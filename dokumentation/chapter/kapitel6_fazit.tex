%!TEX root = ../dokumentation.tex

\chapter{Projektreflektion}
Zum Zeitpunkt der Konzeptabgabe stand ein Projektplan mit groben Vorstellungen zum weiteren Projektverlauf.\footnote{Konzeptseite 43} Neben den organisatorischen Meilensteinen zu Abgaben der Artefakte, wurden 2 projektspezifische Meilensteine definiert. Der erste Meilenstein sollte am 25. November erreicht werden und die Bearbeitung des MCI Vorgehens und die Bearbeitung der Proof-of-Concepts beinhalten. 
Anschließend war geplant die Implementation der reinen Funktionalitäten durchzuführen und bis zum 2. Meilenstein am 20. Dezember damit fertig zu werden. Die anschließende Zeit sollte der Gestaltung des Interface gewidmet werden, was wiederrum Gestaltungslösungen der MCI beinhaltet hätte. \\
Bereits nach kurzer Zeit innerhalb der Dokumentationsphase wurde bewusst, dass speziell die Planung der MCI Beschäftigung viel zu knapp bemessen war und sich über den gesamten Projektzeitraum erstrecken sollte. Aufgrund des iterativen Charakters, sollte die Phase bis zum ersten Meilenstein eher als erste Iterationsstufe angesehen werden. Die Überarbeitung des Konzeptes beanspruchte mehr Zeit als anfänglich eingeplant, was zur Folge hatte, dass die eingeplante Zeit für MCI Aspekte weiterhin verkürzt wurde. Als Reaktion darauf, wurde die erste Iterationsphase der MCI Auseinandersetzung bis zum ersten Meilenstein geplant. Zum ersten Test der konzipierten Systemarchitektur, war die Umsetzung der Proof-of-Concepts der erste Schritt, der bis zum ersten Meilenstein ohne Komplikationen durchgeführt werden konnte.\\
Im Laufe des Projektes zeigte sich, dass vorallem zu Beginn einzelne Arbeitsphasen zu grob geplant wurden, was zur Folge hatte das die Auseinandersetzung mit einzelnen Themen mehr Zeit beanspruchte, als ursprünglich gedacht. Bei der organisation des Projektplans wurde auf Excel zurückgegriffen, was zur Folge hatte, dass der Plan bei zunehmender Projektdauer und Verfeinerung der unübersichtlich wurde, letzendlich aber keinen Einfluss auf das Zeitmanagement an sich hat.\\

Rückblickend bedeutete sowohl die MCI Auseinandersetzung als auch die Implementation des Systems mehr Arbeit, als in der gegeben Zeit investiert werden konnte. 
Auf zeitliche Rückschläge wurde an gegeben Stellen reagiert, jedoch hätte eine konkretere Fokusierung auf wesentliche Aspekte die Entwicklungszeit eventuell kontrollierbarer gestaltet.\\ 

Zahlreiche Techniken der MCI wurden durchgeführt, letztendlich reichte es aber zeitlich nicht den finalen Prototypen zu evaluieren und auf diese Art in ein Interface umzusetzen. Während des Projektverlaufs wäre eine gezieltere Wahl der Methoden an gegeben Stellen von Vorteil gewesen, um zusätzlichen Aufwand und Überarbeitung zu vermeiden. Die Auseinandersetzung mit den Benutzern und ihren Aufgaben fand in einem umfangreichen Rahmen statt und sind nach eigenem Ermessen eine gute Grundlage für weitere Iterationsphasen.\\

Zu Beginn der Konzeptphase, wurden eigene Ziele an das Projekt in Form einer Zielhierachie definiert.\footnote{Konzeptkapitel 6}
In Hinblick auf das Alleinstellungsmerkmal wurde dabei das Minimalziel beschrieben, dass auf der Vermittlung zwischen Mieter und Vermieter lag und die Kontaktaufnahme beider Parteien ermöglicht. Da in Anbetracht der Systemdokumentation, wesentliche Aspekten umgesetzt werden konnten, gilt zumindest dieses Ziel als weitestgehend erfüllt.