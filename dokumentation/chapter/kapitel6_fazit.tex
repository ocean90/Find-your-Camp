%!TEX root = ../dokumentation.tex

\chapter{Projektreflektion}

Erfüllungsgrad
Bewertung des eigenen Prozesses

Hoher Detailierungsgrad\\
Voraus planen\\
Iterationen vorgegebene und eigene Meilensteine definieren\\
Vorgehensmodell berücksichtigen/ in Literatur Methoden, Techniken, Gestaltungsaktivitäten aus der ISO... nachschlagen\\
Pufferzeiten\\
Architektur und PoC berücksichtigen\\
Aufteilung der Arbeitszeit zwischen den Gruppenmitgliedern\\
Geplanter Aufwand vs. eingetretener Aufwand in Stunden\\
Wichtig: Anhand der Zielpriorisierung bei Zeitmangel begründetes Weglassen einzelner\\


Erfüllungsgrad detailliert diskutiert\\ 
beispielsweise die Entwurfsziele, die Systemarchitektur und -technik und die Nutzbarkeit, zu berücksichtigen. Zudem soll ein Fazit gezogen und ein Ausblick gegeben werden, was evtl. an dem System zu ändern oder ergänzen wäre\\
inwiefern sich der reale Zeitplan von dem aus der Konzeptionsphase unterscheidet

Zum Zeitpunkt der Konzeptabgabe stand ein Projektplan mit groben Vorstellungen zum weiteren Projektverlauf. Neben den organisatorischen Meilensteinen zu Abgaben der Artefakte, wurden 2 projektspezifische Meilenstein definiert. Der erste Meilenstein sollte am 25. November erreicht werden und die Bearbeitung des MCI Vorgehens und die Bearbeitung der Proof-of-Concepts beinhalten. 
Anschließend war geplant die Implementation der reinen Funktionalitäten durchzuführen und bis zum 2. Meilenstein am 20. Dezember damit fertig zu werden. Die anschließende Zeit sollte der Gestaltung des Interface gewidmet werden, was Gestaltungslösungen der MCI beinhaltet hätte. \\

