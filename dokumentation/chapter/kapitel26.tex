%!TEX root = ../konzept.tex

\section{Geschäftsmodell}
Um das Projekt langfristig umsetzen zu können ist es notwendig, die finanziellen Kosten zu decken und im Idealfall auch Gewinn zu erzielen. 
Daher wurden erste Ansätze eines möglichen Geschäftsmodells überlegt.

\begin{itemize}
   \item \textbf{Kostenlose Nutzung, dafür einmalige Zahlung für App}. Keine Zutsatsgebühren für Inserate, Verleihprozesse etc.\\
   Vorteil: Nutzer werden nur einmalig mit Appkosten konfrontiert, geringe Folgekosten für Anwender.

   Nachteil: Schwer kritische Menge zu erreichen, direkte Kosten können Einstiegshürde darstellen, keine langfristigen Einnahmen für Serverkosten etc. Finanzierung müsste mit anderen Mitteln wie Werbung erreicht werden. In diesem Fall wäre eine kostenpflichtige Applikation eher unpassend. Es ergebe sich die Option einer werbefinanzierten kosntelosen Variante und dem kostenpflichtigen, dafür werbefreien Gegenstück.

   \item \textbf{In-App Käufe oder Guthaben aufladen} 
   Die Bezahlung der Mietvorgänge wird komplett über die Applikation geregelt. Aufladen des entsprechenden Guthaben gibt Anteilmäßige Einnahmen. Eine weitere Möglichkeit wären In-App Käufe mit zusätzlichen Funktionen. Dabei sollten diese jedoch im Kontext Sinn ergeben und auch ohne diese noch zum gewünschten Ziel führen. 

   \item \textbf{Werbung + Eventkooperationen}.
   Innerhalb der Anwendungen gibt es die Möglichkeit mit Partnern zu kooperieren. Vorstellbar wären z.B. Eventpartner, die bei bestimmten Veranstaltungungen dafür werben und entsprechende Angebote darauf auslegen. Dies ist ein greifender Punkt, da auch Events als Motivationsgrund der Anwender gelten können und zu solchen vermehrt eingesetzt werden. Zusätzlich dazu innerhalb einer (werbepflichten) Applikation Werbung von Partnern der entsprechenden Domäne.

   \item \textbf{Bezahlung läuft über App ab (2)} 
   Dadurch haben die Benutzer eine Bestätigung und Sicherheit, da sie zum einen bargeldlos bezahlen können und eine Absicherung über den Mietvorgang haben. (Keiner kann sich nach angenommener Leistung aus der Bezahlung abwenden). Durch den Bezahlprozess wird eine Anteilmäßige Provision draufgerechnet. 

\end{itemize}
 
Ein vorläufiges Modell konnte an diesem Punkt noch nicht genauer ausgemacht werden, da erst mit der genaueren funktionalen Betrachtung und Nutzungsmotivation einige dieser Aspekte abgehandelt werden können. Weitere Schritte werden im Rahmen der Dokumentation abgehandelt.