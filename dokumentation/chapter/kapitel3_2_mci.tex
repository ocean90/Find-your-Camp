%!TEX root = ../dokumentation.tex

\section{Ausarbeitung MCI}

Innerhalb der Konzeptphase wurde bereits der Grundstein zur Benutzermodellierung gelegt. Innerhalb der Zieldomäne  fand eine Identifikation vorhandener Stakeholdergruppen samt Einordnung statt. Für die weitere Auseinandersetzung geht es nun darum, die einzelnen Benutzergruppen genauer zu analysieren und entsprechende User Profiles zu entwickeln, um die Anwender genauer bestimmten zu können. Daraufhin aufbauend sollen Beispiel Persona erstellt werden, da diese innerhalb des Design Prozesse geeigneter sind um Interaktionsprozesse nachvollziehbarer zu gestalten. Der Vorteil von Persona gegenüber User Profiles liegt darin, dass kognitive Vorgänge durch die “Modellierung” einer Person mit Hintergrundinformationen präziser zu erreichen sind. 
Auch wenn der Fokus des Usage Center Designs auf der eigentlichen Interaktion liegt, sollte diese Modellierungstiefe erreicht werden, da innerhalb des Verleihprozesses unterschiedliche Anwendertypen auftreten können mit eigenen (moralischen) Ansichten zum Shared Economy Konzept. Speziell bezogen auf eine Zielgruppe, die ihr Grundstück nur unter bestimmten Sicherheitsaspekten verleihen würde, kann die Auseinandersetzung neue Erkenntnisse zu geforderten Funktionalitäten und Restriktionen liefern. (Seite 17 3.1 Benutzermodellierung)	


-> danach Entwicklung von Szenarien

-> User Role/ Use Cases speziell für usage centered	

\newpage

%!TEX root = ../dokumentation.tex

\section{MCI1}




\newpage

%!TEX root = ../dokumentation.tex

\section{MCI2}




\newpage

%!TEX root = ../dokumentation.tex

\section{MCI3}





\newpage