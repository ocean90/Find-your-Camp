%!TEX root = ../dokumentation.tex

\chapter{Einführung}
"Im Rahmen des Moduls Entwicklungsprojekt interaktive Systeme, geht es um die Konzipierung und Umsetzung einer verteilten, multimedialen Anwendung unter Verwendung zuvor erlernter Grundlagen. Innerhalb des Projektes sollen Methoden und Techniken der Veranstaltungen der Mensch Computer Interaktion und Web-basierte Anwendungen 2: Verteilte Systeme selbstständig geplant und durchgeführt werden."\footnote{Auszug der Konzepteinleitung, Seite 1}\\

Nach Ausarbeitung des Konzeptes innerhalb der ersten Projektphase, befasst sich diese Dokumentation mit der weiterführenden Projektarbeit bis hin zur finalen Abgabe der entstandenen Projektergebnisse. Die bereits begonnene Auseinandersetzung mit wesentlichen Aspekten der Mensch Computer Interaktion und Webbasierte Anwendungen 2, wird weiter vertieft und gewonnene Ergebnisse dargelegt.\\

Die vorliegende Projektdokumentation ist in mehrere Teile strukturiert.\\
Nach der allgemeinen Einführung folgt als erster Hauptbestandteil die Prozessdokumentation, die den zeitlichen Ablauf des Projektes fokusiert. 
Darin wird die Entwicklung nach Konzeptabgabe, einzelne Arbeitsschritte und angewendete MCI Methoden mit entstandenen Ergebnissen beschrieben.\\ Es folgt die Systemdokumentation, welche das entwickelte System genauer erklärt und die Alternativen und Abwägungen während der Entwicklungsphase herausstellt.\\ 
Zur Verwendung wird anschließend innerhalb der Installationsdokumentation beschrieben, welche Anforderungen an ein System gestellt werden, um die entwickelte Anwendung benutzen zu können. Dazu werden entsprechende Installationsschritte vorgestellt.\\
Zum Abschluss folgt die Projektreflektion in welcher die anfangs gesteckten und letztendlich erreichten Ziele verglichen und ein Fazit zum Projekt gezogen wird.\\

Die überarbeitete Version des Konzept befindet sich im Anhang dieses Dokuments und wird an gegebenen Stellen referenziert.

